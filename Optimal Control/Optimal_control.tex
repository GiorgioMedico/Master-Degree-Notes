%%%%%%%%%%%%%%%%%%%%%%%%%%%%%%%%%%%%%%%%%%%%%%%%%%%%%%%%%%%%%%%
% Document Class Configuration
%%%%%%%%%%%%%%%%%%%%%%%%%%%%%%%%%%%%%%%%%%%%%%%%%%%%%%%%%%%%%%%
\documentclass[openany]{book}

%%%%%%%%%%%%%%%%%%%%%%%%%%%%%%%%%%%%%%%%%%%%%%%%%%%%%%%%%%%%%%%
% Basic Packages
%%%%%%%%%%%%%%%%%%%%%%%%%%%%%%%%%%%%%%%%%%%%%%%%%%%%%%%%%%%%%%%
% Language settings
\usepackage[english]{babel}

% Page geometry and layout
\usepackage{geometry}
\geometry{
    a4paper,           % Paper size
    total={170mm,237mm}, % Total usable area
    left=20mm,         % Left margin
    top=30mm          % Top margin
}

% Line spacing control
\usepackage{setspace}
\onehalfspacing       % Set 1.5 line spacing for the document

%%%%%%%%%%%%%%%%%%%%%%%%%%%%%%%%%%%%%%%%%%%%%%%%%%%%%%%%%%%%%%%
% Mathematical Packages and Settings
%%%%%%%%%%%%%%%%%%%%%%%%%%%%%%%%%%%%%%%%%%%%%%%%%%%%%%%%%%%%%%%
\usepackage{
    amsmath,    % Advanced math formatting
    amsthm,     % Theorem environments
    amsfonts,   % Mathematical fonts
    bm          % Bold math symbols
}

\usepackage{algorithm}
\usepackage{algpseudocode}

%%%%%%%%%%%%%%%%%%%%%%%%%%%%%%%%%%%%%%%%%%%%%%%%%%%%%%%%%%%%%%%
% Graphics and Figures
%%%%%%%%%%%%%%%%%%%%%%%%%%%%%%%%%%%%%%%%%%%%%%%%%%%%%%%%%%%%%%%
\usepackage{graphicx}  % Include graphics
\usepackage{float}     % Improved float environment
\usepackage{tikz}      % Drawing tools
\graphicspath{{./images/}}  % Path for images

%%%%%%%%%%%%%%%%%%%%%%%%%%%%%%%%%%%%%%%%%%%%%%%%%%%%%%%%%%%%%%%
% Header, Footer, and Hyperlinks
%%%%%%%%%%%%%%%%%%%%%%%%%%%%%%%%%%%%%%%%%%%%%%%%%%%%%%%%%%%%%%%
\usepackage[hidelinks]{hyperref}  % Hyperlinks without colored borders
\usepackage{fancyhdr}             % Custom headers and footers

% Header and footer configuration
\pagestyle{fancy}
\fancyhf{}                        % Clear all header/footer fields
\renewcommand{\headrulewidth}{0pt}% Remove header rule
\fancyfoot[C]{\thepage}          % Center page number in footer
\setlength{\footskip}{50pt}      % Footer distance from text

%%%%%%%%%%%%%%%%%%%%%%%%%%%%%%%%%%%%%%%%%%%%%%%%%%%%%%%%%%%%%%%
% Custom Mathematical Operators and Commands
%%%%%%%%%%%%%%%%%%%%%%%%%%%%%%%%%%%%%%%%%%%%%%%%%%%%%%%%%%%%%%%
% Function evaluation
\newcommand\at[2]{\left.#1\right|_{#2}}

% Mathematical operators
\DeclareMathOperator{\sgn}{sgn}           % Sign function
\DeclareMathOperator{\col}{col}           % Column operator
\DeclareMathOperator{\des}{des}           % Descent operator
\DeclareMathOperator*{\argmax}{arg\,max}  % Argument of maximum
\DeclareMathOperator*{\argmin}{arg\,min}  % Argument of minimum

% Special symbols and notations
\newcommand{\notimplies}{%
    \mathrel{{\ooalign{\hidewidth$\not\phantom{=}$\hidewidth\cr$\implies$}}}}
\newcommand{\R}{\mathbb{R}}               % Real numbers
\newcommand{\N}{\mathbb{N}}               % Natural numbers
\newcommand{\deriv}[1]{\displaystyle\frac{d}{d #1}}  % Derivative
\newcommand{\traj}{(\bar{\mathbf{x}},\bar{\mathbf{u}})}  % Trajectory

%%%%%%%%%%%%%%%%%%%%%%%%%%%%%%%%%%%%%%%%%%%%%%%%%%%%%%%%%%%%%%%
% Theorem Environments
%%%%%%%%%%%%%%%%%%%%%%%%%%%%%%%%%%%%%%%%%%%%%%%%%%%%%%%%%%%%%%%
% Definition-style theorems
\theoremstyle{definition}
\newtheorem{definition}{Definition}[section]
\newtheorem{theorem}{Theorem}[section]
\newtheorem{proposition}{Proposition}[section]
\newtheorem*{corollary}{Corollary}

% Remark-style theorems
\theoremstyle{remark}
\newtheorem*{remark}{Remark}
\newtheorem*{notation}{Notation}
\newtheorem*{assumption}{Assumption}


%%%%%%%%%%%%%%%%%%%%%%%%%%%%%%%%%%%%%%%%%%%%%%%%%%%%%%%%%%%%%%%
% Document Information
%%%%%%%%%%%%%%%%%%%%%%%%%%%%%%%%%%%%%%%%%%%%%%%%%%%%%%%%%%%%%%%
\title{Optimal Control M}
\author{Giorgio Medico based on Dante Piotto's notes}
\date{fall semester 2024}


\begin{document}
\maketitle
\tableofcontents

\chapter*{Course Overview}
This course focuses on optimization-based control of dynamical systems. It provides theoretical and numerical methods to compute control system trajectories that minimize a performance index, with particular emphasis on their application to trajectory optimization and maneuvering of Autonomous Systems.

\section*{Learning Objectives}
Upon completion of this course, students will be able to:
\begin{enumerate}
    \item Set up optimal control problems and characterize optimality conditions
    \item Develop numerical optimization methods to compute optimal, feasible trajectories
    \item Design optimization-based receding-horizon control schemes for nonlinear systems
\end{enumerate}

\section*{Course Applications}
To bridge the gap between theory and application, students will apply the proposed techniques to trajectory optimization and maneuvering of Autonomous Systems across various domains including:
\begin{itemize}
    \item Autonomous Vehicles
    \item Robotic Systems (e.g., Aerial Robots)
    \item Other Mechatronic Systems
\end{itemize}

\chapter{Introduction to optimal control}

\section{Optimal control problem formulation}
Consider the continuous-time system (State-Space representation) ($t\in\R$)

\begin{equation}\label{system}
    \begin{gathered}\
        \dot{x}(t) = f(x(t),u(t),t) \\
        y(t) = h(x(t),u(t),t)
    \end{gathered}
\end{equation}

\begin{itemize}
    \item $x(t)\in\R^n$ state of the system at time $t$ 
    \item $u(t)\in\R^m$ input of the system at time $t$ 
    \item $y(t)\in\R^p$ output of the system at time $t$
\end{itemize}
We will mainly work with time invariant systems, $\dot{x}(t)=f(x(t),u(t))$. \\[0.5cm]
We consider nonlinear, discrete-time systems described by Finite-Difference Equations (FDE):
\[
    x(t+1)=f_t(x(t),u(t)) \quad t\in\N_0
\]
but from now on we will use the compact notation
\[
    x_{t+1}=f_t(x_t,u_t) \quad t\in\N_0
\]
where $x_t\in\R^n$ and $u_t\in\R^m$ are the state and the input of the system at time $t$.

Consider a nonlinear, discrete-time system on a finite time horizon 
\[
    x_{t+1} = f_t(x_t,u_t) \quad t=0,\dots,T-1
\]
We use $\mathbf{x}\in\R^{nT}$ and $\mathbf{u}\in\R^{mT}$ to denote, respectively, the stack of the states $x_t$ for all $t\in\{1,\dots,T\}$ and the inputs $u_t$ for all $t\in\{0,\dots,T-1\}$, that is:
\begin{gather*}
    \mathbf{x} := \begin{bmatrix}
        x_1 \\ \vdots \\ x_T
    \end{bmatrix} \qquad
    \mathbf{u} := \begin{bmatrix}
        u_0 \\ \vdots \\ u_{T-1}
    \end{bmatrix}
\end{gather*}

\subsubsection{Trajectory of a system}

\textbf{Definition:} A pair ($\bar{\mathbf{x}},\bar{\mathbf{u}})\in\R^{nT} \times \R^{mT}$ is called a trajectory of system \eqref{system} 
if $\bar{x}_{t+1}=f_t(\bar{x}_t,\bar{u}_t)$ for all $t\in\{0,\dots,T-1\}$. That is, if $(\bar{\mathbf{x}},\bar{\mathbf{u}})$ satisfies the system dynamics (the same holds for continuous time systems with proper adjustments). In particular, $\bar{\mathbf{x}}$ is the state trajectory, while $\bar{\mathbf{u}}$ is the input trajectory.

\subsubsection{Equilibrium}

\textbf{Definition:} A state-input pair $(x_e,u_e)\in\R^n\times\R^m$ is called an equilibrium pair of \eqref{system} if $(x_t,u_t)=(x_e,u_e),\forall t\in\N_0$ is a trajectory of the system. Equilibria of time-invariant systems satisfy $x_e=f(x_e,u_e)$

\subsubsection{Linearization of a system about a trajectory}
Given the dynamics \eqref{system}
and a trajectory $(\bar{\mathbf{x}},\bar{\mathbf{u}})$, the linearization of \eqref{system} about $(\bar{\mathbf{x}},\bar{\mathbf{u}})$ is given by the linear (possibly) time-varying system 
\[
    \Delta x_{t+1} = A_t\Delta x_t + B_t \Delta u_t \quad t\in\N_0
\]
with $A_t$ and $B_t$ the Jacobians of $f_t$, with respect to state and input respectively, evaluated at $(\bar{\mathbf{x}},\bar{\mathbf{u}})$
\[
    A_t = \at{\displaystyle\frac{\partial}{\partial x}f(\bar{x}_t,\bar{u}_t)}{(\bar{\mathbf{x}},\bar{\mathbf{u}})} \quad B_t = \at{\displaystyle\frac{\partial}{\partial u}f(\bar{x}_t,\bar{u}_t)}{(\bar{\mathbf{x}},\bar{\mathbf{u}})}
\]
where $x_t \approx \bar{x}_t + \Delta x_t$ and $u_t \approx \bar{u}_t + \Delta u_t$.\\ 
($\bar{x_t}$ and $\bar{u_t}$ are the nominal trajectory values of $x_{t+1} = f_t(x_t,u_t)$ and $\Delta x_t$, $\Delta u_t$ are the trajectory of the linearized system.)


\subsection{Optimization}
\subsubsection{Main ingredients}
\begin{itemize}
    \item Decision variable: $z\in\R^d$ 
    \item Cost function: $\ell(z):\R^d\to\R$ cost associated to decision $z$
    \item Constraints (constraint sets): for some given functions $h_i:\R^d\to\R,  \text{ and } g_j:\R^d\to\R,$ the decision vector $z\in\R^d$ needs to satisfy 
        \begin{gather*}
            h_i(z)=0 \quad i=1,\dots,m \\
            g_j(z)\leq0 \quad j=1,\dots,r
        \end{gather*}
        equivalently we can say that we require $z\in Z$ with 
        \[
            Z=\{z\in\R^d|h(z)=0, g(z)\leq 0\},
        \]
        where we compactly denoted $h(z)=\col(h_1(z),\dots,h_m(z))$ and $g(z) = \col(g_1(z),\dots,g_r(z))$
\end{itemize}

\subsubsection{Minimization}
We can write our optimization problem as 
\begin{align*}
    \min_{z\in\R^d} &\ell(z)\\
    \text{subj. to } &h_i(z) = 0 \quad i=1,\dots,m\\
    &g_j(z)\leq 0 \quad j=1,\dots,r
\end{align*}
where $h_i:\R^d\to\R$ and $g_j:\R^d\to\R$\\
We can write it more compactly as 
\begin{align*}
    \min_{z\in\R^d} &\ell(z)\\
    \text{subj. to } &h(z) = 0 \\
    &g(z)\leq 0 
\end{align*}
where $h:\R^d\to\R^m$ and $g:\R^d\to\R^r$
\subsection{Discrete-time optimal control}
\subsubsection{Main Ingredients}
\begin{itemize}
    \item Dynamics: a discrete-time system in state space form 
        \[
            x_{t+1} = f_t(x_t,u_t) \quad t=0,1,\dots,T-1
        \]
    \item the dynamics introduce $T$ equality constraints 
        \[
            \begin{array}{l l l}
                x_1 = f(x_0,u_0) & \quad \text{i.e.} \quad& x_1-f_t(x_0,u_0)=0 \\
                x_2 = f(x_1,u_1) & \quad \text{i.e.} \quad & x_2-f_t(x_1,u_1)=0 \\
                \vdots \\
                x_T = f(x_{T-1},u_{T-1}) & \quad \text{i.e.} \quad & x_T-f_t(x_{T-1},u_{T-1})=0 \\
            \end{array}
        \] 
        This is equivalent to $nT$ scalar constraints
    \item Cost function: a cost "to be payed" for a chosen trajectory. We consider an additive structure in time 
        \[
            \ell(\mathbf{x},\mathbf{u}) = \displaystyle\sum_{t=0}^{T-1}\ell_t(x_t,u_t)+\ell_T(x_T)
        \]
        where $\ell_t:\R^n\times\R^m\to\R$ is called stage-cost, while $\ell_T:\R^n\to\R$ is the terminal cost. 
    \item End-point constraints: function of the state variable prescribed at initial and/or final point 
        \[
            r(x_0,x_T)=0
        \]
    \item Path constraints: point-wise (in time) constraints representing possible limits on states and inputs at each time $t$ 
        \[
            g_t(x_t,u_t)\leq 0, \quad t\in\{0,\dots,T-1\}
        \]
\end{itemize}


A discrete-time optimal control problem can be written as 
\begin{gather*}
    \min_{\substack{x_0,x_1,\dots,x_T\\u_0,\dots,u_{T-1}}} \displaystyle\sum_{t=0}^{T-1}\ell_t(x_t,u_t)+\ell_T(x_T)\\[10pt]
    \begin{array}{l l }
        \text{subj. to } & x_{t+1}=f_t(x_t,u_t), \quad t\in\{0,\dots,T-1\}\\[5pt]
                         & r(x_0,x_T) = 0  \\[5pt]
                         & g_t(x_t,u_t)\leq 0, \quad t\in\{0,\dots,T-1\}
    \end{array}
\end{gather*}

\subsubsection{Optimal control for trajectory generation}
We can pose a trajectory generation problem as 
\begin{gather*}
    \min_{\mathbf{x}\in\R^n,\mathbf{u}\in\R^m}\displaystyle\sum_{t=0}^{T-1}\displaystyle\frac{1}{2}\|x_t-x_t^{\des}\|^2_Q + \displaystyle\frac{1}{2}\|u_t-u_t^{\des}\|^2_R + \displaystyle\frac{1}{2}\|x_T-x_T^{\des}\|^2_{P_f}
\end{gather*}

\subsubsection{Continuous-time Optimal Control problem}
A continuous-time optimal control problem, i.e., $t\in\R$ can be written as 
\begin{gather*}
    \min_{(x(\cdot),u(\cdot))\in \mathcal{F}}\displaystyle\int_{0}^{T}\ell_\tau(x(\tau),u(\tau))d\tau+\ell_T(x(T))\\
    \begin{array}{l l}
        \text{subj. to } & \dot{x}(t) = f_t(x(t),u(t)) \quad t\in[0,T]\\
                         & r(x(0),x(T)) = 0 \\
                         & g_t(x(t),u(t))\leq 0 \quad t\in[0,T)
    \end{array}
\end{gather*}
Note that $\mathcal{F}$ is a space of functions (function space). This is an infinite dimensional optimization problem
\begin{itemize}
    \item Cost functional $\ell:\mathcal{F}\to\R$
        \[
            \ell(x(\cdot),u(\cdot)) = \displaystyle\int_{0}^{T}\ell_\tau(x(\tau),u(\tau))d\tau+\ell_T(x(T))
        \]
    \item Space of trajectories (or trajectory manifold)
        \[
            \mathcal{T} = \{(x(\cdot),u(\cdot))\in\mathcal{F}\ |\ \dot{x}(t)=f_t(x(t),u(t)),\ t\geq 0\}
        \]
\end{itemize}

\chapter{Nonlinear Optimization}

\section{Overview}

\begin{enumerate}
    \item \textbf{Motivating Examples}
        \begin{itemize}
            \item Drone trajectory generation
            \item Pendulum control
            \item Trajectory manifold concept
        \end{itemize}
    
    \item \textbf{Theoretical Foundations}
        \begin{itemize}
            \item Types of minima
                \begin{itemize}
                    \item Global/Local
                    \item Strict/Non-strict
                \end{itemize}
            \item Optimality conditions
                \begin{itemize}
                    \item First-order necessary
                    \item Second-order necessary
                    \item Second-order sufficient
                \end{itemize}
        \end{itemize}
    
    \item \textbf{Convex Optimization}
        \begin{itemize}
            \item Convex sets and functions
            \item Constraint definitions
            \item Properties
                \begin{itemize}
                    \item Local = Global minima
                    \item First-order sufficiency
                \end{itemize}
            \item Quadratic programs
        \end{itemize}
    
    \item \textbf{Solution Methods}
        \begin{itemize}
            \item Iterative descent methods
                \begin{itemize}
                    \item Gradient methods
                    \item Quadratic minimization
                \end{itemize}
            \item Newton's method
                \begin{itemize}
                    \item Zero-finding
                    \item Unconstrained optimization
                \end{itemize}
            \item Step-size selection
                \begin{itemize}
                    \item Constant
                    \item Diminishing
                    \item Armijo rule
                \end{itemize}
        \end{itemize}
    
    \item \textbf{Constrained Optimization}
        \begin{itemize}
            \item Convex set optimization
                \begin{itemize}
                    \item Projection methods
                    \item Feasible directions
                \end{itemize}
            \item Equality/Inequality constraints
                \begin{itemize}
                    \item Active constraints
                    \item KKT conditions
                    \item Complementary slackness
                \end{itemize}
            \item Advanced methods
                \begin{itemize}
                    \item Sequential Quadratic Programming
                    \item Barrier functions
                \end{itemize}
        \end{itemize}
    
    \item \textbf{Implementation}
        \begin{itemize}
            \item Python implementation
            \item Convergence visualization
            \item Performance analysis
        \end{itemize}
\end{enumerate}

\textbf{Nonlinear programming (NLP) is the process of solving an optimization problem where the objective function or some of the constraints are Non-Linear.}

\section{Unconstrained Optimization}
Consider the unconstrained optimization problem 
\[
    \min_{z\in\R^n}\ell(z)
\]
with $\ell:\R^d\to\R$ a cost function to be minimized and $z$ a decision vector 

We say that $z^*$ is a  
\begin{itemize}
    \item global minimum if $\ell(z^*)\leq\ell(z)$ for all $z\in\R^n$ 
    \item strict global minimum if $\ell(z^*)<\ell(z)$ for all $z\neq z^*$
    \item local minimum if there exists $\epsilon>0$ such that $\ell(z^*)\leq\ell(z)$ for all $z\in B(z^*,\epsilon) = \{z\in\R^d| \|z-z^*\|<\epsilon\}$
    \item strict local minimum if there exists $\epsilon>0$ such that $\ell(z^*)<\ell(z)$ for all $z\in B(z^*,\epsilon)$ 

\end{itemize}

\begin{figure}[ht]
    \centering
    \includegraphics[width=0.35\textwidth]{min.png}
    \caption{Illustration of different types of minima in an optimization problem}
    \label{fig:min}
\end{figure}

\subsubsection{Notation}
We denote $\ell(z^*)$ the optimal (minimum) value of a generic optimization problem, i.e. 
\[
    \ell(z^*) = \min_{z\in\R^d}\ell(z)
\]
where $z^*$ is the minimum point (optimal value for the optimization variable) i.e. 
\[
    z^* = \argmin_{z\in\R^d} \ell(z)
\]
\subsubsection{Gradient and Hessian}
Gradient of a function: for a function $r:\R^d\to\R$  the gradient is denoted as 
\[
    \nabla r(z) = \begin{bmatrix}
        \displaystyle\frac{\partial r(z)}{\partial z_1} \\ \vdots \\ \displaystyle\frac{\partial r(z)}{\partial z_d}
    \end{bmatrix} \in \R^{d\times 1}
\]
Hessian matrix of a function: for a function $r:\R^d\to\R$  the Hessian matrix is denoted as 
\[
    \nabla^2(r(z)) = \begin{bmatrix}
        \displaystyle\frac{\partial^2 r(z)}{\partial z_1^2} & \cdots & \displaystyle\frac{\partial^2 r(z)}{\partial z_1z_d} \\ \vdots & \ddots & \vdots \\ \displaystyle\frac{\partial^2 r(z)}{\partial z_dz_1} & \cdots & \displaystyle\frac{\partial^2 r(z)}{\partial z_d^2}
    \end{bmatrix} \in \R^{d\times d}
\]
The Hessian matrix is a symmetric matrix, since the assumption of continuity of the second derivatives implies that the order of differentiation does not matter.\\

Gradient of a vector-valued function: for a vector field $r:\R^d\to\R^m$, the gradient is denoted as 
\[
    \nabla r(z) = \begin{bmatrix}
        \nabla r_1(z) & \cdots & \nabla r_m(z)
    \end{bmatrix} = \begin{bmatrix}
        \displaystyle\frac{\partial r_1(z)}{\partial z_1} & \cdots & \displaystyle\frac{\partial r_m(z)}{\partial z_1} \\
        \vdots & \ddots & \vdots \\
        \displaystyle\frac{\partial r_1(z)}{\partial z_d} & \cdots & \displaystyle\frac{\partial r_m(z)}{\partial z_d} \\
    \end{bmatrix} \in \R^{d\times m}
\]
which is the transpose of the Jacobian matrix of $r$

\subsection{Conditions of optimality}

\subsubsection{First order Necessary condition (FNC) of optimality (unconstrained)}

Let $z^*$ be an unconstrained local minimum of $\ell:\R^d\to\R$ and assume that $\ell$ is continuously differentiable ($\mathcal{C}^1$) in $B(z^*,\varepsilon)$\footnote{Ball of radius $\varepsilon$ centered in $z^*$} for some $\varepsilon>0$. Then $\nabla \ell(z^*)=0$ (the gradient of $\ell$ at $z^*$ is zero).

\subsubsection{Second order Necessary condition (FNC) of optimality (unconstrained)}

If additionally $\ell$ is twice continuously differentiable ($\mathcal{C}^2$) in $B(z^*,\varepsilon)$, then $\nabla^2 \ell(z^*)\geq 0$ (The Hessian of $\ell$ is positive semidifinite).
\\
% Remark 
\begin{remark}
    Points $\bar{z}$ satisfying $\nabla \ell(\bar{z})=0$ are called stationary points. They include minima, maxima and
saddle points.
\end{remark}

\subsubsection{Second order Sufficient conditions of optimality (unconstrained)}

Let $\ell:\R^d\to\R\in\mathcal{C}^2$ (twice differentiable) in $B(z^*,\varepsilon)$ for some $\varepsilon>0$. Suppose that $z^*\in\R^d$ satisfies 
\[
    \nabla\ell(z^*) = 0\quad \text{and} \quad \nabla^2\ell(z^*)>0
\]
Then $z^*$ is a strict (unconstrained) local minimum of $\ell$

\subsubsection{Convex set} 
A set $Z\subset \R^d$ is convex if for any two points $z_A$ and $z_B$ in $Z$ and for all $\theta\in[0,1]$, then 
\[
    \theta z_A + (1-\theta)z_B \in Z
\]
\begin{figure}[ht]
    \centering
    \begin{minipage}{.5\textwidth}
        \centering
        \includegraphics[width=0.3\linewidth]{cvxset}
    \caption{Convex set}
    \end{minipage}%
    \begin{minipage}{.5\textwidth}
        \centering
        \includegraphics[width=0.3\linewidth]{ncvxset}
    \caption{Non convex set}
    \end{minipage}
\end{figure}

\subsubsection{Convex functions}
Let $Z \subset \R^d$ be a convex set. A function $\ell:Z\to\R$ is convex if for any two points $z_A$ and $z_B$ in $Z$ and for all $\theta\in[0,1]$, then 
\[
    \ell(\theta z_A + (1-\theta)z_B)\leq \theta\ell(z_A)+ (1-\theta)\ell(z_B)
\]
\begin{figure}[ht]
    \centering
    \includegraphics[width=0.3\textwidth]{cvxfcn}
    \caption{Convex function}
\end{figure}

\begin{remark}
A function $\ell$ is \emph{concave} if $-\ell$ is convex. A function $\ell$ is strictly convex if the inequality holds strictly for $z_A\neq z_B$ and $\theta\in(0,1)$.
\end{remark}

\subsubsection{Inequality constraints and convex sets}
Let $g:\R^d\to\R^p$, we can define a set $Z_{\text{ineq}}\subset \R^d$ as
\[
    Z_{\text{ineq}} = \{z\in\R^d|g(z)\leq 0\}
\]
The set $Z_{\text{ineq}}$ is convex iff $g$ is a quasi-convex function (e.g., monotone functions on the axis)

\begin{corollary}
    if $g$ is convex then $Z_{\text{ineq}}$ is convex and vice versa.
\end{corollary}

% add const.png
\begin{figure}[ht]
    \centering
    \includegraphics[width=0.7\textwidth]{const}
    \caption{Convex set identified through inequality constraints}
\end{figure}

\subsubsection{Equality constraints and convex sets}
Let $h:\R^d\to\R^p$, we can define a set $Z_{\text{eq}}\subset \R^d$ as
\[
    Z_{\text{eq}} = \{z\in\R^d|h(z)= 0\}
\]
The set $Z_{\text{eq}}$ is convex iff $h$ is an affine function. Convex sets identified through equality constraints are linear spaces (hyperplanes).

\subsubsection{Convexity and gradient monotonicity}
If $\ell$ is differentiable and convex, then for all $z_A, z_B \in Z$ it holds
\[
    \ell(z_B) \geq \ell(z_A) + \nabla\ell(z_A)^T(z_B - z_A)
\]

If $\ell$ is differentiable and convex, then its gradient $\nabla\ell: \mathbb{R}^d \to \mathbb{R}^d$ satisfies
\[
    (\nabla\ell(z_A) - \nabla\ell(z_B))^T(z_A - z_B) \geq 0
\]
for all $z_A, z_B$. That is, the gradient $\nabla\ell$ is a monotone operator.

\subsubsection{Strict convexity and gradient monotonicity}
A function $\ell$ is strictly convex if for $z_A \neq z_B$ and $\theta \in (0,1)$
\[
    \ell(\theta z_A + (1-\theta)z_B) < \theta\ell(z_A) + (1-\theta)\ell(z_B)
\]

If the strictly convex function $\ell$ is also differentiable, then its gradient satisfies
\[
    (\nabla\ell(z_A) - \nabla\ell(z_B))^T(z_A - z_B) > 0
\]
for all $z_A, z_B$. That is, the gradient $\nabla\ell$ is a strictly monotone operator.

\subsubsection{Strong convexity and gradient monotonicity}
A function $\ell$ is strongly convex with parameter $\mu > 0$ if for $z_A \neq z_B$ and $\theta \in (0,1)$
\[
    \ell(\theta z_A + (1-\theta)z_B) < \theta\ell(z_A) + (1-\theta)\ell(z_B) - \mu\theta(1-\theta)\|z_A - z_B\|^2
\]

The gradient of a differentiable strongly convex function satisfies
\[
    (\nabla\ell(z_A) - \nabla\ell(z_B))^T(z_A - z_B) \geq \mu\|z_A - z_B\|^2
\]
for all $z_A, z_B$. That is, the gradient $\nabla\ell$ is a strongly monotone operator.

\subsubsection{Convexity and Lipschitz continuity of the gradient}
Consider a differentiable convex function $\ell$ with a Lipschitz continuous gradient with parameter $L > 0$, i.e.,
\[
    \|\nabla\ell(z_A) - \nabla\ell(z_B)\| \leq L\|z_A - z_B\|
\]
for all $z_A, z_B$.

Then, the following characterization holds
\[
    (\nabla\ell(z_A) - \nabla\ell(z_B))^T(z_A - z_B) \geq \frac{1}{L}\|\nabla\ell(z_A) - \nabla\ell(z_B)\|^2
\]
for all $z_A, z_B$. That is, the gradient $\nabla\ell$ is a co-coercive operator.

\subsubsection{Strong convexity and Lipschitz continuity of the gradient}
Consider a strongly convex (with parameter $\mu > 0$) function $\ell$ with Lipschitz continuous gradient (with parameter $L > 0$)

Then the following characterization holds
\[
    (\nabla\ell(z_A) - \nabla\ell(z_B))^T(z_A - z_B) \geq \frac{\mu L}{\mu+L}\|z_A - z_B\|^2 + \frac{1}{\mu+L}\|\nabla\ell(z_A) - \nabla\ell(z_B)\|^2
\]
for all $z_A, z_B$

\subsubsection{Graphical interpretation of convexity}
If $\ell$ is differentiable and convex, then
\[
    \ell(z_B) \geq \ell(z_A) + \nabla\ell(z_A)^T(z_B - z_A)
\]
for all $z_A, z_B \in Z$

If $\ell$ is differentiable and $\mu$-strongly convex, then
\[
    \ell(z_B) \geq \ell(z_A) + \nabla\ell(z_A)^T(z_B - z_A) + \frac{\mu}{2}\|z_B - z_A\|^2
\]
for all $z_A, z_B \in Z$
\begin{figure}[ht]
    \centering
    \includegraphics[width=0.4\textwidth]{quadratic lower bound on l}
    \caption{Strong convexity imposes a quadratic lower bound on $\ell$}
\end{figure}

\subsubsection{Graphical interpretation of Lipschitz continuity of the gradient}
If $\ell$ is differentiable with Lipschitz continuous gradient, then
\[
    \ell(z_B) \leq \ell(z_A) + \nabla\ell(z_A)^T(z_B - z_A) + \frac{L}{2}\|z_B - z_A\|^2
\]
for all $z_A, z_B \in Z$
\begin{figure}[ht]
    \centering
    \includegraphics[width=0.4\textwidth]{quadratic upper bound on l}
    \caption{Lipschitz continuity of $\nabla\ell$ imposes a quadratic upper bound on $\ell$}
\end{figure}

\subsection{Minimization of convex functions}

\subsubsection{Proposition}
Let $Z\subset\R^d$ be a convex set and $\ell: Z\to\R$ a convex function. Then a local minimum of $\ell$ is also a global minimum. \\
Proof: not done in class but present in slides for funsies

\subsubsection{Necessary and sufficient condition of optimality (unconstrained)}
For the unconstrained minimization of a convex function it can be shown that the first order necessary condition of optimality is also sufficient (for a global minimum).

\subsubsection{Proposition}
Let $\ell:\R^d \to \R$ be a convex function. Then $z^*$ is a global minimum if and only if $\nabla\ell(z^*)=0$.\\
Proof: not done in class but present in slides for funsies

\section{Quadratic programming (unconstrained)}
Let us consider a special class of optimization problems, namely quadratic optimization problems or quadratic programs: 
\[
    \min_{z\in\R^d}z^TQz+b^Tz
\]
with $Q=Q^T\in\R^{d\times d}$ and $b\in\R^d$.


\subsubsection{Optimality conditions}
First-order necessary condition for optimality: if $z^*$ is a minimum then 
\[
    \nabla \ell(z^*)=0 \implies 2Qz^*+b=0
\]
Second-order necessary condition for optimality: if $z^*$ is a minimum then 
\[
    \nabla^2\ell(z^*)\geq 0 \implies 2Q\geq0
\]
A necessary condition for the existence of minima for a quadratic program is that $Q\geq 0$. Thus, quadratic programs admitting at least a minimum are convex optimization problems.


\subsubsection{Properties}
Since quadratic programs are convex programs ($Q\geq 0$ is necessary to have a local minimum), then the following holds: 
\begin{center}
     For a quadratic program necessary conditions of optimality are also sufficient and minima are global
\end{center}
If $Q>0$, then there exists a unique global minimum given by 
\[
    z^* = -\displaystyle\frac{1}{2}Q^{-1}b
\]

\section{Unconstrained Optimization Algorithms}
\subsection{Iterative descent methods}
We consider optimization algorithms relying on the iterative descent idea. We denote $z^k\in\R^d$ an estimate of a local minimum at iteration $k\in\N$. The algorithm starts at a given initial guess $z^0$ and iteratively generates vectors $z^1,z^2,\dots$ such that $\ell$ is decreased at each iteration, i.e. 
\[
    \ell(z^{k+1})<\ell(z^k) \qquad k = 1,2,\dots
\]
% add descent.png
\begin{figure}[ht]
    \centering
    \includegraphics[width=0.35\textwidth]{descent}
    \caption{Illustration of the iterative descent idea}
\end{figure}
\subsubsection{Two-step procedure}
We consider a general two-step procedure (it's a class of algorithms) that reads as follows 
\[
    z^{k+1} = z^k+\gamma^k d^k, \qquad k=1,2,\dots
\]
in which 
\begin{enumerate}
    \item each $\gamma^k>0$ is a "step-size" 
    \item $d^k\in\R^d$ is a "direction"
\end{enumerate}
Procedure : 
\begin{enumerate}
    \item choose a direction $d^k$ along which the cost decreases for $\gamma^k$ sufficiently small;
    \item select a step-size $\gamma^k$ guaranteeing a sufficient decrease. 
\end{enumerate}
In other references these are called line-search methods.

\subsection{Gradient methods}
Let $z^k$ be such that $\nabla\ell(z^k)\neq 0$. We start by considering the update rule 
\[
    z^{k+1} = z^k-\gamma^k\nabla\ell(z^k)
\]
i.e., we choose $d^k = -\nabla\ell(z^k)$.\\
From the first order Taylor expansion of $\ell$ at $z$ we have 
\begin{align*}
    \ell(z^{k+1}) & =  \ell(z^k)+\nabla\ell(z^k)^T(z^{k+1}-z^k)+o(\|z^{k+1}-z^k\|)\\
    & =  \ell(z^k)-\gamma^k\|\nabla\ell(z^k)\|^2+o(\gamma^k)
\end{align*}
Thus, for $\gamma^k>0$ sufficiently small it can be shown that $\ell(z^k+1)<\ell(z^k)$.\\
The update rule 
\[
    z^{k+1}=z^k-\gamma^k\nabla\ell(z^k)
\]
can be generalized to so called \emph{gradient methods}
\[
    z^{k+1}=z^k+\gamma^kd^k
\]
with $d^k$ such that
\[
    \nabla\ell(z^k)^Td^k<0
\]
Also, $d^k$ must be gradient related, i.e. $d^k$ must not asymptotically become perpendicular to $\nabla\ell$

\subsubsection{Selecting the descent direction}

Several gradient methods (Generalized form) can be written as 
\[
    z^{k+1} = z^k-\gamma^kD^k\nabla\ell(z^k) \quad k=1,2,\dots
\]
where $D^k\in\R^{d\times d}$ is  a symmetric positive definite matrix. It can be immediately seen that 
\[
    -\nabla\ell(z^k)^TD^k\nabla\ell(z^k)<0
\]
i.e. $d^k = -D^k\nabla\ell(z^k)$ is a descent direction.\\
The choice of $D^k$ must be made such that there exist $d_1,d_2$ positive real, such that $d_1 I \leq D^k \leq d_2 I$.\\ [0.5cm]
Some choices for $D^k$:
\begin{itemize}
    \item Steepest descent $D^k=I_d$
    \item Newton's method $D^k = (\nabla^2\ell(z^k))^{-1}$\\
        It can be used when $\nabla^2\ell(z^k)>0$. It typically converges very fast asymptotically. For $\gamma^k = 1$ pure Newton's method
    \item Discretized Newton's method $D^k=(H(z^k))^{-1}$, where $H(z^k)$ is a positive definite symmetric approximation of $\nabla^2\ell(z^k)$ obtained by using finite difference approximations of the second derivatives 
    \item Some regularized version of the Hessian
\end{itemize}

\subsection{Steepest descent (Gradient method)}
The update rule obtained for $D^k=I$ is called steepest descent $z^{k+1} = z^k-\gamma^k\nabla\ell(z^k)$.\\
The name steepest descent is due to the following property: the normalized negative gradient direction 
\[
    d^k = -\displaystyle\frac{\nabla\ell(z^k)}{\|\nabla\ell(z^k)\|}
\]
minimizes the slope $\nabla \ell(z^k)^Td^k$ among all normalized directions, i.e. it gives the steepest descent.

\subsection{Newton's method for root(zero) finding}

Consider the nonlinear root(zero) finding problem 
\[
    r(z) = 0
\]
Idea: iteratively refine the solution such that the improved guess $z^{k+1}$ represents a root(zero) of the linear approximation of $r$ about the current tentative solution $z^k$.\\ 
Consider the linear approximation of $r$ about $z^k$, we have 
\[
    r^k(z^k+\Delta z^k) = r(z^k)+\nabla r(z^k)^T\Delta z^k
\]
then, finding the zeros of the approximation, we have
\[
    \Delta z^k = -(\nabla r(z^k)^T)^{-1}r(z^k)
\]
where $\Delta z^k = z^{k+1}-z^k$.
Thus, the solution is improved as 
\[
    z^{k+1} = z^k-(\nabla r(z^k)^T)^{-1}r(z^k)
\]

\subsection{Newton's method for Unconstrained Optimization}
Consider the unconstrained optimization problem 
\[
    \min_{z\in\R^d} \ell(z)
\]
stationary points $\bar{z}$ satisfy the first order optimality condition 
\[
    \nabla \ell (\bar{z}) = 0
\]
We can look at it as a root finding problem, with $r(z)=\nabla\ell(z)$, and solve it via Newton's method. Therefore, we can compute $\Delta z^k$ as the solution of the linearization of $r(z)=\nabla\ell(z)$ at $z^k$, i.e. 
\[
    \nabla \ell(z^k) + \nabla^2\ell(z^k)\Delta z^k = 0
\]
and run the update 
\[
    z^{k+1} = z^k -\gamma^k(\nabla^2\ell(z^k))^{-1}\nabla\ell(z^k)
\]
where in the pure Newton's method $\gamma^k=1$, in other cases $\gamma^k$ has to be selected.

\subsection{Newton's method via Quadratic Optimization}
This approach it's a way to reduce a complex problem into a sequence of simpler QP problems.
Observe that 
\[
    \nabla\ell(z^k) +\nabla^2\ell(z^k)\Delta z^k = 0
\]
is the first-order necessary and sufficient condition of optimality for the quadratic program 
\begin{equation}
    \label{qp}
    \Delta z^k = \argmin_{\Delta z}\nabla\ell(z^k)^T \Delta z+\displaystyle\frac{1}{2}\Delta z^T\nabla^2\ell(z^k)\Delta z
\end{equation}
we define $q^T = \nabla\ell(z^k)^T$ and $Q = \nabla^2\ell(z^k)$, $Q$ must be positive definite to be solvable.\\
Thus, the $k$-th iteration of Newton's method can be seen as 
\[
    z^{k+1} = z^k-\Delta z^k
\]
with $\Delta z^k$ solution of the quadratic problem \eqref{qp}. Generalized version: 
\[
    z^{k+1} = z^k - \gamma^k \Delta z^k
\]

\subsection{Gradient methods via quadratic optimization}
The "Quadratic approach" can be used with all gradient methods characterized by a certain descent direction $d^k$.
Similarly to Newton's method, a descent direction $\Delta z^k=D^k\nabla\ell(z^k)$ can be seen as the direction that minimizes at each iteration a different quadratic approximation of $\ell$ about $z^k$. 
In fact, consider the quadratic approximation $\ell^k(z)$ of $ \ell$ about $z^k$ given by the second order Taylor expansion
\[
    \ell^k(z) = \ell(z^k)+\nabla\ell(z^k)^T(z-z^k)+\displaystyle\frac{1}{2}(z-z^k)^T(D^k)^{-1}(z-z^k)
\]
By setting the derivative to zero, we have 
\[
    \nabla\ell(z^k)+(D^k)^{-1}(z-z^k)=0
\]
we can calculate the minimum of $\ell^k(z)$ and set it as the next iterate $z^{k+1}$
\[
    z^{k+1} = z^k - D^k\nabla\ell(z^k)
\]
namely, the update rule of a gradient method.
\[
    \Delta z^k = -D^k\nabla\ell(z^k)
\]
% Remark
\begin{remark}
    Direction $\Delta z^k$ can be computed by solving the quadratic approximation. This can be very
    helpful in optimization problems in which a gradient is not (easily) available, but it is to solve a quadratic
    optimization problem. It is useful also in case the domain is a compact convex set.
\end{remark}

\subsection{Step-size Selection Rules}
\begin{itemize}
    \item Constant step-size: $\gamma^k=\gamma>0$
    \item Diminishing step-size: $\gamma^k\to 0$ as $k\to\infty$. It must hold that \[
            \displaystyle\sum_{k=0}^{\infty}\gamma^k = +\infty \quad \text{and} \quad \displaystyle\sum_{k=0}^{\infty}(\gamma^k)^2 < +\infty
        \]
        The above conditions avoid pathological choices of $\gamma^k$, it must be avoided that beacons too small.
    \item Armijo rule
\end{itemize}

\subsubsection{Armijo rule}
Given the descent direction $d^k$ we can consider 
\[
    g^k(\gamma) = \ell(z^k+\gamma d^k), \quad g:\R\to\R
\]
% TODO insert image slide 30
The value of $g^k(\gamma)$ for $\gamma=0$ is $\ell(z^k)$. The minimization rule chooses as the value for $\gamma$ the value that minimizes $g^k(\gamma)$. The partial minimization rule would search for a minimum in a restricted set of values for $\gamma$. Let us differentiate $g$ wrt $\gamma$:
\begin{gather*}
    g'(\gamma)=\displaystyle\frac{d}{d\gamma}g(\gamma)=\displaystyle\frac{d}{d\gamma}\ell(z^k+\gamma d^k)\\
    g'(0) = \displaystyle\frac{d}{d\gamma}\at{\ell(z^k+\gamma d^k)}{\gamma=0} = \nabla \ell(z^k)^Td^k\\
\end{gather*}
We compute a linear approximation of $g(\gamma)$:
\begin{gather*}
    g(\gamma) = g(0) + g'(0)\gamma+o(\gamma)\\
    \ell(z^k+\gamma d^k) = \ell(z^k)+\gamma \nabla\ell(z^k)^Td^k + o(\gamma)
\end{gather*}
This is the tangent to the $g(\gamma)$ curve at $\gamma=0$. We also consider the line 
\[
    \ell(z^k)+c\gamma\nabla\ell(z^k)^Td^k
\]
which is a line with a slightly less negative slope given that $c\in(0,1)$.
The Armijo rule is applied as follows: 
\begin{enumerate}
    \item Set $\bar{\gamma}^0>0,\quad\beta\in(0,1),\quad c\in(0,1)$
    \item While $\ell(z^k+\bar{\gamma}^id^k)\geq \ell(z^k)+c\bar{\gamma}^i\nabla\ell(z^k)^Td^k$:
        \[
            \bar{\gamma}^{i+1}=\beta\bar{\gamma}^i
        \]
    \item Set $\gamma^k = \bar{\gamma}^i$
\end{enumerate}
Typical values are $\beta=0.7$ and $c=0.5$

\begin{figure}[ht]
    \centering
    \begin{minipage}{.33\textwidth}
        \centering
        \includegraphics[width=0.9\linewidth]{armijo0}
    \end{minipage}%
    \begin{minipage}{.33\textwidth}
        \centering
        \includegraphics[width=0.9\linewidth]{armijo1}
    \end{minipage}%
    \begin{minipage}{.33\textwidth}
        \centering
        \includegraphics[width=0.9\linewidth]{armijo2}
    \end{minipage}%
    \caption{The iteration stops when the $g$ function evaluated at $\gamma^k$ is below the $c$-line.}
\end{figure}

The idea is to fix the $c$-line and iterate over $\beta$ for the value of $\gamma^k$ until we obtain a value such that the $g$ function evaluated at that value is below the $c$-line.

\paragraph{Proposition: convergence with Armijo step-size}

Let $\{z^k\}$ be a sequence generated by a gradient method $z^{k+1}=z^k-\gamma^kD^k\nabla\ell(z^k)$ with $d_1I\leq D^k \leq d_2I$ with $d_1,d_2>0$. 
Assume that $\gamma^k$ is chosen by the Armijo rule and $\ell(z)\in \mathcal{C}^1$. Then, every limit point $\bar{z}$ of the sequence $\{z^k\}$ is a stationary point, i.e. $\nabla\ell(\bar{z})=0$
\begin{remark}
Recall that a vector $z\in\R^d$ is a limit point of a sequence $\{z^k\}$ in $\R^d$ if there exists a subsequence of $\{z^k\}$ that converges to $z$.
\end{remark}

\paragraph{Convergence with constant or diminishing step-size}

Let $\{z^k\}$ be a sequence generated by a gradient method $z^{k+1}=z^k-\gamma^kD^k\nabla\ell(z^k)$ with $d_1I\leq D^k \leq d_2I$ whit $d_1,d_2>0$. Assume that for some $L>0$ 
\[
    \|\nabla\ell(z)-\nabla\ell(y)\|\leq L\|z-y\|, \qquad \forall z,y\in\R^d
\]
i.e. The gradient is a Lipschitz continuous function.
Assume either
\begin{enumerate}
    \item $\gamma^k=\gamma>0$ sufficiently small, or 
    \item $\gamma^k\to 0$ and $\displaystyle\sum_{t=0}^{\infty}\gamma^k=\infty$
\end{enumerate}
Then, every limit point $\bar{z}$ of the sequence $\{z^k\}$ is a stationary point, i.e. $\nabla\ell(\bar{z})=0$

\subsubsection{Remarks on gradient methods}

\begin{itemize}
    \item The propositions do not guarantee that the sequence converges and not even existence of limit points. Then either $\ell(z^k)\to-\infty$ or $\ell(z^k)$ converges to a finite value and $\nabla\ell(z^k)\to 0$. In the second case, one can show that any subsequnce $\{z^{k_p}\}$ converges to some stationary point $\bar{z}$ satisfying $\nabla\ell(\bar{z})=0$
    \item Existence of minima can be guaranteed by excluding $\ell(z^k)\to-\infty$ via suitable assumptions. Assume, e.g., $\ell$ coercive (radially unboundend), i.e. $\|z\|\to\infty\implies \ell(z)\to\infty$
    \item For general (nonconvex) problems, assuming coercivity, only convergence (of subsequences) to stationary points can be proven.
    \item for convex programs, assuming coercivity, convergence to global minima is guaranteed since necessary conditions of optimality are also sufficient.
\end{itemize}

\section{Constrained optimization over convex sets}

Consider the optimization problem 
\[
    \min_{z\in Z}\ell(z)
\]
where $Z \subset \R^d$ is nonempty, convex, and closed, and $\ell$ is continuously differentiable on $Z$. 
\begin{remark}
    The function $\ell$ can be nonconvex
\end{remark}

\subsubsection{Optimality conditions}
It' a Necessary condition, if a point $z^* \in Z$ is a local minimum of $\ell(z)$ over $Z$, then 
\[
    \nabla\ell(z^*)^T(\bar{z}-z^*)\geq 0 \qquad \forall\bar{z}\in Z
\]


\subsubsection{Projection over a convex set}
Given a point $z\in\R^d$ and a closed convex set $Z$, it can be shown that 
\[
    P_Z(z) := \argmin_{z\in Z}\|z-z\|^2
\]
exists and is unique. The point $P_Z(z)$ is called the projection of $z$ on $Z$.

\subsection{Projected gradient method}

Gradient methods can be generalized to optimization over convex sets 
\[
    z^{k+1}=P_Z(z^k-\gamma^k\nabla\ell(z^k))
\]
The algorithm is based on the idea of generating at each $t$ feasible points (i.e. belonging to $Z$) that give a descent in the cost. The analysis follows similar arguments to the one of unconstrained gradient methods.

\subsection{Feasible direction method}

Find $\tilde{z}\in\R^d$ such that 
\[
    \tilde{z} = \argmin_{z\in Z} \ell(z^k)+\nabla\ell(z^k)^T(z-z^k)+\displaystyle\frac{1}{2}(z-z^k)^T(z-z^k)
\]
Update the solution 
\[
    z^{k+1}=z^k+\gamma^k(\tilde{z}-z^k)
\]
where $(\tilde{z}-z^k)$ is a feasible direction as it is contained in the set by construction. For $\gamma^k$ sufficiently small, $z^{k+1}\in Z$

\subsection{Barrier function strategy for inequality constraints}

Consider the inequality constrained optimization problem 
\begin{align*}
    &\min_{z\in\R^d}\ell(z)\\
    \text{subj. to } &g_j(z)\leq 0 \quad j\in \{1,\dots,r\}
\end{align*}  
inequality constraints can be relaxed and embedded in the cost function by means of a barrier function $-\varepsilon \log(z)$. The resulting unconstraind problem reads as 
\[
    \min_{z\in\R^d} \ell(z) + \varepsilon \displaystyle\sum_{j=1}^{r}-\log(-g_j(z))
\]
Implementation: every few iterations shrink the barrier parameters $\varepsilon$.
Methods such as this go by the name of \emph{interior point methods}.
% Add barrier.png
\begin{figure}[ht]
    \centering
    \includegraphics[width=0.35\textwidth]{barrier}
    \caption{Illustration of the barrier function strategy}
\end{figure}

\section{Constrained optimization: optimality conditions}

\begin{align*}
    \min_{z\in Z}\  &\ell(z)\\
    \text{subj. to } & g_j(z)\leq 0 \quad j\in\{1,\dots,r\}\\
    & h_i(z)=0 \quad i\in\{1,\dots,m\}
\end{align*}

\begin{definition}[Set of active inequality constraints]
    For a point $z$, the set of active inequality constraints at $z$ is $A(z) = \{j\in\{1,\dots,r\}|g_j(z)=0\}$
\end{definition}
\begin{definition}[Regular point]
    A point $z$ is regular if the vectors $\nabla h_i(z), i\in \{1,\dots,m\}$ and $\nabla g_j(z), j\in A(z)$, are linearly independent
\end{definition}

\subsubsection{Lagrangian function}

In order to state the first-order necessary conditions of optimality for (equality and inequality) constrained problems it is useful to introduce the Lagrangian function 
\[
    \mathcal{L}(z,\mu,\lambda)=\ell(z)+\displaystyle\sum_{j=1}^{r}\mu_ig_j(z) + \displaystyle\sum_{i=1}^{m}\lambda_i h_i(z)
\]
with $z \in \R^d, \mu\in\R^r,\lambda\in\R^m$.

\begin{theorem}[Karush-Kuhn-Tucker Necessary conditions]
    Let $z^*$ be a regular local minimum of 
    \begin{align*}
        \min_{z\in\R^d}\  &\ell(z) \\
        \text{subj. to }\  & g_j(z)\leq 0 \quad j\in\{1,\dots,r\}\\
        & h_i(z)=0 \quad i\in\{1,\dots,m\}
    \end{align*}
    where $\ell,g_j$ and $h_i$ are $\mathcal{C}^1$. \\
    Then $\exists$ unique $\mu_j^*$ and $\lambda_i^*$, called \emph{Lagrange multipliers}, s.t.
    \[
        \begin{array}{ r c l}
            \nabla_1\mathcal{L}(z^*,\mu^*,\lambda^*) & = & 0 \\             \mu_j^* & \geq & 0 \\
            \mu^*_jg_j(z^*) & = & 0 \qquad j\in \{1,\dots,r\} 
        \end{array}
    \]
    Moreover, if $\ell, g_j$ and $h_i$ are $\mathcal{C}^2$ it holds
    \[
        y^T\nabla_{11}^2\mathcal{L}(z^*,\mu^*,\lambda^*)y \geq 0
    \]
    for all $y\in\R^d$ such that
    \[
    \nabla h_i(z)^Ty = 0, \quad i\in\{1,\dots,m\}, \qquad \nabla g_j(z)^Ty = 0, \quad j\in A(z) \quad \text{(i.e. } j\in\{1,\dots,r\} \text{ s.t. } g_j(z)=0\}
\]
\end{theorem}
\begin{remark}
    $\nabla_1\mathcal{L}(z^*,\mu^*,\lambda^*) = 0$ can be written as 
    $\nabla\ell(z^*)+\displaystyle\sum_{j=1}^{r}\mu_j^*\nabla g_j(z^*)+\displaystyle\sum_{i=1}^{m}\lambda_i^*\nabla h_i(z^*)=0$
\end{remark}
\begin{remark}
    The condition $\mu^*g_j^*(z^*)=0,j\in\{1,\dots,r\}$, is called \emph{complementary slackness}
\end{remark}
\begin{notation}
    Points satisfying the KKT necessary conditions of optimality are referred to as \emph{KKT points}. They are the counterpart of stationary points in constrained optimization.
\end{notation}
\begin{notation}
    $\nabla_1$ denotes the gradient wrt the first variable of the function
\end{notation}
\begin{notation}
    $\nabla_{11}$ denotes the hessian of a function wrt the first variable
\end{notation}

\begin{theorem}[Second order Sufficient conditions]
If $z^*$ satisfied the KKT first order necessary conditions and also $\exists$ $\mu^*,\lambda^*$ s.t.$\nabla_1^2\mathcal{L}(z^*,\mu^*,\lambda^*) > 0$, $\forall y\not=0$, $z^*$ is a local minimum.
\end{theorem}

\subsection{Quadratic programming (constrained): optimality conditions}

Let us consider quadratic optimization problems with linear equality constraints 
\begin{align*}
    \min_{z\in\R^d}\ & z^TQz+q^Tz \\
    \text{subj. to }\ & Az=b
\end{align*}
with $Q=Q^T\in\R^{d\times d}, q\in\R^d, A\in\R^{m\times d}$ and $b\in\R^m$.
The Lagrangian function is:
\[
    \mathcal{L}(z,\lambda) = z^TQz + q^Tz +\displaystyle\sum_{i=1}^{m}\lambda_i(A_iz+b_i) =  z^TQz + q^Tz + \lambda^T(Az-b)
\]
with $\lambda\in\R^m$. And the gradient computes as 
\[
    \nabla_1 \mathcal{L}(z^*,\lambda^*) = 2Qz^* + q + \displaystyle\sum_{i=1}^{m}\lambda_i^*A_i^T =  2Qz^* + q + A^T\lambda^*
\]
The equality constraints must also be enforced: 
\[
    Az^*-b = 0
\]
We can note that 
\[
    \nabla_2\mathcal{L}(z^*,\lambda^*) = Az^*-b
\]
Therefore, first order conditions of optimality may be written as 
\[
    \begin{bmatrix}
        \nabla_1\mathcal{L}(z^*,\lambda^*)\\ \nabla_2\mathcal{L}(z^*,\lambda^*)
    \end{bmatrix} = 0
\]
This is always the case when only equality constraints are present. Second order necessary conditions for optimality impose that, if $z^*$ is a minimum then 
\[
    y^T\nabla^2_{11}\mathcal{L}(z^*,\lambda^*)y = y^TQy \geq 0
\]
for all $y\in\R^d$ such that 
\[
    \nabla h_i(z)^T y = 0 \qquad i\in\{1,\dots,m\} \quad \implies \quad A^Ty = 0
\]
namely, for all $y\in\R^d$ in the null-space of $A^T$

\begin{theorem}[Second Order Sufficient Conditions (Quadratic Programming Constrained)]
If $z^*$ (satisfying $h(z^*)=0$) is such that there exists $\lambda^*$ satisfying $2Qz^* + q + A^T\lambda^* = 0$ and $y^T Q y > 0$ for all $y \neq 0$, $y \in \ker(A^T)$, then $z^*$ is a global minimum.
\end{theorem}

\section{Constrained optimization: optimization algorithms}

\subsection{Newton's method for equality constrained problems}

KKT points can be found by solving a root finding problem in variables $z,\lambda$ wrt $r(z,\lambda)=\nabla \mathcal{L}(z,\lambda)$. Newton's method for this root finding problem reads as 
\[
    \begin{bmatrix}
        z^{k+1} \\ \lambda^{k+1}
        \end{bmatrix} = \begin{bmatrix}
        z^k \\ \lambda^k
        \end{bmatrix} + \begin{bmatrix}
        \Delta z^k \\ \Delta \lambda^k
    \end{bmatrix}
\]
with 
\[  \nabla^2 \mathcal{L}(z^k,\lambda^k)
    \begin{bmatrix}
        \Delta z^k \\ \Delta \lambda^k
    \end{bmatrix} = -\nabla \mathcal{L}(z^k,\lambda^k)
\]
where 
\begin{gather*}
    \nabla^2\mathcal{L}(z^k,\lambda^k)= \begin{bmatrix}
        \nabla_{11} \mathcal{L}(z^*,\lambda^*) & \nabla_{12} \mathcal{L}(z^*,\lambda^*)\\
        \nabla_{21} \mathcal{L}(z^*,\lambda^*) & \nabla_{22} \mathcal{L}(z^*,\lambda^*)
        \end{bmatrix} = \begin{bmatrix}
        B^k & \nabla h(z^k) \\
        \nabla h(z^k)^T & 0
        \end{bmatrix} \\ \nabla \mathcal{L}(z^k,\lambda^k) = \begin{bmatrix}
        \nabla \ell(z^k)+\nabla h(z^k)\lambda^k \\ h(z^k)
    \end{bmatrix}
    %\nabla_{11}\mathcal{L}(z,\lambda) = \nabla^2 \ell(z) + \displaystyle\sum_{i=1}^{m}\lambda_i\nabla^2h_i(z)
\end{gather*}
whit $B^k = \nabla_{11}^2 \mathcal{L}(z^k,\lambda^k)$.\\
We can write 
\[
    \nabla^2\mathcal{L}(z^k,\lambda^k)\begin{bmatrix}
        \Delta z^k \\ \Delta \lambda^k
    \end{bmatrix} = -\nabla \mathcal{L}(z^k,\lambda^k)
\]
namely 
\[
    \begin{bmatrix}
        B^k & \nabla h(z^k) \\
        \nabla h(z^k)^T & 0
        \end{bmatrix}  \begin{bmatrix}
        \Delta z^k \\ \Delta \lambda^k
        \end{bmatrix} = -\begin{bmatrix}
        \nabla \ell(z^k)+\nabla h(z^k)\lambda^k \\ h(z^k)
    \end{bmatrix}
\]
thus, $\Delta z^k, \Delta\lambda^k$ can be obtained as solution of a linear system of equations in the variables $\Delta z, \Delta\lambda$.\\
The linear system of equations can be rewritten as 
\begin{gather*}
    B^k\Delta z^k+\nabla h(z^k)\Delta \lambda^k = -\nabla\ell(z^k) - \nabla h(z^k)\lambda^k \\
    \nabla h(z^k)^T \Delta z^k = -h(z^k)
\end{gather*}
and equivalently as 
\begin{gather*}
    \nabla\ell(z^k) +B^k\Delta z^k+\nabla h(z^k)\lambda^{k+1} = 0 \\
    h(z^k)+\nabla h(z^k)^T\Delta z^k = 0
\end{gather*}
We can observe that the above equations are the necessary and sufficient optimality conditions for the Quadratic Program (QP)
\begin{align*}
    \min_{\Delta z}\  &\nabla\ell(z^k)^T\Delta z + \displaystyle\frac{1}{2}\Delta z^T B^k\Delta z \\
    \text{subj. to }\  & h(z^k) + \nabla h(z^k)^T \Delta z = 0
\end{align*}
Therefore, in the Newton's update, we can obtain $(\Delta z^k, \lambda^{k+1})$ by solving this QP.

\subsection{Sequential Quadratic Programming (SQP)}

Start from a tentative solution $z^0$. For $k=0,1,\dots$ (up to convergence)
\begin{enumerate}
    \item Compute $\nabla\ell(z^k),B^k,\nabla h(z^k)$ 
    \item Obtain ($\Delta z^k, \Delta \lambda^k_{QP}$) from 
        \begin{align}\label{sqp}
                \Delta z^k = \argmin_{\Delta z}\  & \nabla \ell(z^k)^T\Delta z +\displaystyle\frac{1}{2}\Delta z^T B^k \Delta z \\
                \text{subj. to }\  & h(z^k) + \nabla h(z^k)^T \Delta z = 0
        \end{align}
        with $\Delta\lambda^k_{QP}$ the Lagrange multiplier associated to the optimal solution of \eqref{sqp}
    \item Choose $\gamma^k$ using Armijo's rule on \emph{merit function} $M_1(z^k+\gamma\Delta z^k)$
    \item Update 
        \begin{gather*}
            z^{k+1} = z^k+ \gamma^k\Delta z^k \\
            \lambda^{k+1} = \Delta \lambda^*_{QP}
        \end{gather*}
\end{enumerate}


\chapter{Optimality conditions for optimal control}
\section{Recall on optimal control problems}
\subsection{Discrete-time Control System and Notation}

We consider nonlinear, discrete-time systems described by
\begin{equation}
    x_{t+1} = f_t(x_t, u_t) \qquad t \in \N_0
\end{equation}
where $x_t \in \R^n$ and $u_t \in \R^m$ are the state and the input of the system at time $t$.

\subsubsection*{Notation}
We use $\mathbf{x} \in \R^{nT}$ and $\mathbf{u} \in \R^{mT}$ to denote, respectively, the stack of the states $x_t$ for all $t \in 1,\ldots,T$ and the inputs $u_t$ for all $t \in 0,\ldots,T-1$, that is
\begin{align*}
    \mathbf{x} &:= \col(x_1,\ldots,x_T) \\
    \mathbf{u} &:= \col(u_0,\ldots,u_{T-1}).
\end{align*}
where $\col(x_1,\ldots,x_T)$ is the (column) vector with components $x_1,\ldots,x_T$.

\subsection{System Trajectories}
We recall the definition of trajectory of a (possibly) nonlinear, discrete-time control system.

\begin{definition}
A pair $(\mathbf{x}, \mathbf{u}) \in \R^{nT} \times \R^{mT}$ is called a \emph{trajectory} of system (1) if $\bar{x}_{t+1} = f_t(\bar{x}_t, \bar{u}_t)$ for all $t \in 0,\ldots,T-1$. In particular, $\mathbf{x}$ is the state trajectory, while $\mathbf{u}$ is the input trajectory.
\end{definition}

\begin{remark}
It will be sometimes useful to refer to a generic pair $(\boldsymbol{\alpha}, \boldsymbol{\mu}) \in \R^{nT} \times \R^{mT}$ with $\boldsymbol{\alpha} := \col(\alpha_1,\ldots,\alpha_T)$ and $\boldsymbol{\mu} := \col(\mu_0,\ldots,\mu_{T-1})$ as a state-input curve. Notice that a curve $(\boldsymbol{\alpha}, \boldsymbol{\mu})$ is not necessarily a trajectory, i.e., it does not necessarily satisfy the dynamics (1).
\end{remark}

\section{Dynamics as equality constraints}
We consider nonlinear, discrete-time systems described by 
\begin{equation} \label{system2}
    x_{t+1} = f_t(x_t,u_t) \qquad t\in\N_0
\end{equation}
Let us rewrite the nonlinear dynamics of a dt system as an implicit equality constraint $h:\R^{nT}\times \R^{mT}\to\R^{nT}$ 
\[
    h\traj:=\begin{bmatrix}
        f_0(x_0,u_0)-x_1 \\ \vdots \\ f_{T-1}(x_{T-1},u_{T-1})-x_T
    \end{bmatrix}
\]
so that a curve $\traj$ is a trajectory of the system if it satisfies the (possibly nonlinear) equality constraint 
\[
    h\traj = 0
\]

\subsection{System trajectories and trajectory manifold}

We can now define the trajectory manifold $\mathcal{T}\subset \R^{nT}\times \R^{mT}$ of \eqref{system2}
\begin{align*}
    \mathcal{T} := &\left\{\left(\mathbf(x),\mathbf(u)\right)\in\R^{nT}\times\R^{mT}|h\left(\mathbf(x),\mathbf(u)\right)=0\right\}\\=&\left\{\left(\mathbf(x),\mathbf(u)\right)\in\R^{nT}\times\R^{mT}|x_{t+1}=f_t\left(x_t,u_t\right),t=0,\dots,T-1\right\}
\end{align*}
% Add manifold.png
\begin{figure}[ht]
    \centering
    \includegraphics[width=0.7\textwidth]{manifold}
    \caption{Trajectory manifold $\mathcal{T}$}
\end{figure}

Let $\traj\in\mathcal{T}$ be a trajectory of the system, i.e. a point on the trajectory manifold $\mathcal{T}$. The tangent space to $\mathcal{T}$ at a given trajectory (point) $\traj$, denoted as $T_{(\bar{\mathbf{x}},\bar{\mathbf{u}})}\mathcal{T}$, is the set of trajectories satisfying the linearization of $x_{t+1} = f_t(x_t,u_t)$ about the trajectory $(\bar{\mathbf{x}},\bar{\mathbf{u}})$.\\
That is, $T_{(\bar{\mathbf{x}},\bar{\mathbf{u}})}\mathcal{T}=\{(\mathbf{\Delta x, \Delta u})\in\R^{nT}\times\R^{mT}|\nabla_1h(\mathbf{x},\mathbf{u})^T\mathbf{\Delta x} + \nabla_2h(\mathbf{x},\mathbf{u})^T\mathbf{\Delta u} = 0\}$ is the set of trajectories $(\mathbf{\Delta x, \Delta u})$ of
\[
    \Delta x_{t+1} = A_t\Delta x_t + B_t \Delta u_t
\]
with 
\begin{gather*}
    A_t = \nabla_1f_t(\bar{x}_t,\bar{u}_t)^T\\
    B_t = \nabla_2f_t(\bar{x}_t,\bar{u}_t)^T
\end{gather*}

\section{Unconstrained optimal control problem (D-T)}
We look for a solution of the discrete-time optimal control problemm 
\begin{align*}
        \min_{\mathbf{x}\in\R^n,\mathbf{u}\in\R^m} & \displaystyle\sum_{t=0}^{T-1}\ell_t(x_t,u_t)+\ell_T(x_T)\\
        \text{subj. to } & x_{t+1} = f_t(x_t,u_t), \quad t\in\{0,\dots,T-1\} 
\end{align*}
with given initial condition $x_0 = x^0\in \R^n$, where $\ell_t:\R^n\times\R^m\to\R$ and $\ell_T:\R^n\to\R$ are the stage and terminal costs, respectively, and $f_t:\R^n\times\R^m\to\R^n$ are the dynamics of the system.
\begin{remark}
    An optimal control problem constrainded only by the dynamics is called \emph{unconstrained} optimal control problem.
\end{remark}

From now on, we will assume that functions $\ell_t(\cdot,\cdot),\ell_T(\cdot),f_t(\cdot,\cdot)$ are twice continuosly differentiable, i.e. they are class $\mathcal{C}^2$.
Consider the discrete-time system \eqref{system2}. We can introduce the compact notation 
\[
    \mathbf{x} := \begin{bmatrix}
        x_1 \\ \vdots \\ x_T
    \end{bmatrix} \qquad \mathbf{u} := \begin{bmatrix}
        u_0 \\ \vdots \\ u_{T-1}
    \end{bmatrix}
\]
which allows us to write the cost function compactly as 
\[
    \ell({\mathbf{x},\mathbf{u}}) := \sum_{t=0}^{T-1}\ell_t(x_t,u_t)+\ell_T(x_T)
\]
and the equality constraint represented by the dynamics 
\[
    h(\mathbf{x},\mathbf{u}) := \begin{bmatrix}
        f_0(x_0,u_0)-x_1 \\ \vdots \\ f_{T-1}(x_{T-1}u_{T-1})-x_T
    \end{bmatrix}
\]
In light of this compact notation, we can rewrite the optimal control law problem as 
\begin{align*}
    \min_{\mathbf{x}\in\R^{nT},\mathbf{u}\in\R^{mT}} &\ell(\mathbf{x},\mathbf{u})\\
    \text{subj. to } & h(\mathbf{x},\mathbf{u})=0
\end{align*}
where $\ell:\R^{nT}\times\R^{mT}\to\R$ and $h:\R^{nT}\times\R^{mT}\to\R^{nT}$. This is a constrained nonlinear optimization problem with decision variable $(\mathbf{x},\mathbf{u})$.


\section{KKT conditions for unconstrained optimal control}

The Lagrangian function has the form 
\begin{align*}
    \mathcal{L}(\mathbf{x,u,\lambda}) &= \ell(\mathbf{x,u})+\boldsymbol{\lambda}^Th(\mathbf{x,u}) \\
    &= \displaystyle\sum_{t=0}^{T-1}\ell_t(x_t,u_t)+\ell_T(x_T) + \displaystyle\sum_{t=0}^{T-1}\lambda^T_{t+1}(f_t(x_t,u_t)-x_{t+1}) \\
    &= \displaystyle\sum_{t=0}^{T-1}\left(\ell_t(x_t,u_t)+\lambda^T_{t+1}(f_t(x_t,u_t)-x_{t+1})\right)+\ell_T(x_T) \\
    &=\displaystyle\sum_{t=0}^{T}\mathcal{L}_t(x_t,u_t,\boldsymbol{\lambda})
\end{align*}
where $\boldsymbol{\lambda}\in\R^{nT}$ and 
\begin{align*}
    \mathcal{L}_0(x_0,u_0,\boldsymbol{\lambda}) &= \ell_0(x_0,u_0)+\lambda_1^T f_0(x_0,u_0)\\
    \mathcal{L}_t(x_t,u_t,\boldsymbol{\lambda}) &= \ell_t(x_t,u_t)+\lambda_1^T f_t(x_t,u_t)-\lambda_tx_t\\
    \mathcal{L}_T(x_T,\boldsymbol{\lambda}) &= \ell_T(x_T) - \lambda_T^Tx_T
\end{align*}
Let $(\mathbf{x}^*,\mathbf{u}^*)$ be a regular point for the dynamics constraints and an optimal (state-input) trajectory. Then there exists $\boldsymbol{\lambda}^*$ such that $\nabla \mathcal{L}(\mathbf{x}^*,\mathbf{u}^*,\boldsymbol{\lambda}^*)=0$ 

Let us explicitly write condition $\nabla_{(1,2)} \mathcal{L}(\mathbf{x}^*,\mathbf{u}^*,\boldsymbol{\lambda}^*) = 0 $ 
\[
    \nabla_{(1,2)} \mathcal{L}(\mathbf{x}^*,\mathbf{u}^*,\boldsymbol{\lambda}^*) = \begin{bmatrix}
        \nabla_1\mathcal{L}(\mathbf{x}^*,\mathbf{u}^*,\boldsymbol{\lambda}^*)\\
        \nabla_2\mathcal{L}(\mathbf{x}^*,\mathbf{u}^*,\boldsymbol{\lambda}^*)
    \end{bmatrix} = 0
\]
Let us note that 
\[
    \nabla_1\mathcal{L}(\mathbf{x}^*,\mathbf{u}^*,\boldsymbol{\lambda}^*) = \at{\begin{bmatrix}
            \displaystyle\frac{\partial \mathcal{L}(\mathbf{x},\mathbf{u},\boldsymbol{\lambda})}{\partial (x_1)_1}\\
            \vdots \\
            \displaystyle\frac{\partial \mathcal{L}(\mathbf{x},\mathbf{u},\boldsymbol{\lambda})}{\partial (x_1)_n}\\
            \vdots \\
            \displaystyle\frac{\partial \mathcal{L}(\mathbf{x},\mathbf{u},\boldsymbol{\lambda})}{\partial (x_T)_1}\\
            \vdots \\
            \displaystyle\frac{\partial \mathcal{L}(\mathbf{x},\mathbf{u},\boldsymbol{\lambda})}{\partial (x_T)_n}
    \end{bmatrix}}{\mathbf{x}=\mathbf{x}^*}
\]
Since $\mathcal{L}(\mathbf{x},\mathbf{u},\boldsymbol{\lambda})= \displaystyle\sum_{t=0}^{T}\mathcal{L}(x_t,u_t,\boldsymbol{\lambda})$, we can exploit this sparsity and write 
\begin{align*}
    \nabla_2\mathcal{L}_0(x_0,u_0,\boldsymbol{\lambda}) = 0 \qquad & \nabla_2\ell_0(x_0,u_0) + \nabla_2f_0(x_0,u_0)\lambda_1\\
    \begin{bmatrix}
        \nabla_1\mathcal{L}_t(x_t,u_t,\boldsymbol{\lambda})\\
        \nabla_2\mathcal{L}_t(x_t,u_t,\boldsymbol{\lambda})
    \end{bmatrix} = 0 \qquad &  
    \begin{bmatrix}
        \nabla_1\ell_t(x_t,u_t) + \nabla_1f_t(x_t,u_t)\lambda_{t+1}-\lambda_t \\
        \nabla_2\ell_t(x_t,u_t) + \nabla_2f_t(x_t,u_t)\lambda_{t+1}
    \end{bmatrix} = 0 \quad t=1,\dots,T-1 \\
    \nabla_1\mathcal{L}_t(x_t,\boldsymbol{\lambda}) = 0\qquad & \nabla\ell_T(x_T)-\lambda_T = 0
\end{align*}
Let us introduce $A_{t}^{*}$ and $B_{t}^{*}$ that are the state and input matrix of the linearization of the dynamics around the optimal trajectory $(\mathbf{x}^*,\mathbf{u}^*)$:
\begin{gather*} 
    \nabla_1\ell_t(x_t^*,u_t^*) = a_t^*\in \R^n\\
    \nabla_1f_t(x_t^*,u_t^*) = A_{t}^{*T} \quad \text{where } A_t^* = \left.\frac{\partial f_t(x_t,u_t)}{\partial x_t}\right|_{(x_t^*,u_t^*)} \\
    \nabla_2\ell_t(x_t^*,u_t^*) = b_t^*\in \R^n\\
    \nabla_2f_t(x_t^*,u_t^*) = B_{t}^{*T} \quad \text{where } B_t^* = \left.\frac{\partial f_t(x_t,u_t)}{\partial u_t}\right|_{(x_t^*,u_t^*)}
\end{gather*}
So we can rewrite the KKT conditions for unconstrained optimal control as:
\begin{flalign*}
    &a_t^* + A_{t}^{*T}\lambda_{t+1}^* - \lambda_t^* = 0  \qquad t=T-1,\dots,1\\
    &b_t + B_{t}^{*T}\lambda_{t+1}^* = 0 \qquad t=0,\dots,T-1 \\
    &\nabla\ell_T(x_T^*) + \lambda_T^* = 0
\end{flalign*}

These equations form the basis for solving an optimal control problem using the KKT conditions. The process involves defining an iterative algorithm that starts with an initial guess of the input trajectory and then iterates through the following procedure (for a generic step \( k \)) until convergence is achieved.

\subsection{Indirect methods for optimal control}
Given an initial guess for the input trajectory $\mathbf{u}^k = \col(u_0^k,\ldots,u_{T-1}^k)$, the algorithm proceeds as follows:
\begin{enumerate} 
    \item \textbf{Run "forward"} 
        \[
            x_{t+1}^k = f_t(x_t^k,u_t^k), \quad x_0^k = x_{\text{init}}, \quad t=0,\ldots,T-1
        \]
    \item \textbf{Run "backward"}
        \[
            \lambda_t^k = A_{t}^{kT}\lambda_{t+1}^k + a_t^k
        \]
        starting from $\lambda_T^k = \nabla \ell_T (x_T^k)$, to obtain $\lambda_t^k = col(\lambda_1^k,\ldots,\lambda_T^k)$
    \item \textbf{Given $\lambda_t^k$ solve:}
        \[
            b_t + B_{t}^{*T}\lambda_{t+1}^* =\nabla_2\ell(x_t^0,u_t) + \nabla_2f(x_t^0,u_t)\lambda_{t+1}^0 = 0 \quad t=0,\dots,T-1
        \]
        to get $u^{k+1}=\col(u_0^{k+1},\ldots,u_{T-1}^{k+1})$
\end{enumerate}


\section{KKT conditions for constrained optimal control}
We look for a solution of the discrete-time optimal control problem
\begin{align*}
        \min_{x_0\in\R^n,\mathbf{x}\in\R^{nT},\mathbf{u}\in\R^{mT}} & \displaystyle\sum_{t=0}^{T-1}\ell_t(x_t,u_t)+\ell_T(x_T) \\
        \text{subj. to } & x_{t+1} = f_t(x_t,u_t), \quad t=0,\dots,T-1\\
                        & r(x_0,x_T) = 0 \\
                        & g_t(x_t,u_t)\leq 0 ,\quad t=0,\dots,T-1
\end{align*}
where 
\begin{itemize}
    \item $\ell_t:\R^n\times\R^m\to\R$ is the stage cost,
    \item $\ell_T:\R^n\to\R$ is the terminal cost,
    \item $r: \R^n\times\R^n\to\R^{p_0}$ identifies a \emph{boundary constraint} on initial and final states,
    \item $g_t:\R^n\times\R^m\to\R^p$ for each $t$ identifies \emph{point-wise constraints} on state and input at some time $t$
\end{itemize}
The Lagrangian function has the form 
\begin{align*}
    \mathcal{L}(\mathbf{x},\mathbf{u},\boldsymbol{\lambda},\boldsymbol{\mu}) =& \ell(\mathbf{x},\mathbf{u})+\boldsymbol{\lambda}_d^T h(\mathbf{x},\mathbf{u})+ \lambda_b^T r(x_0,x_T) + \boldsymbol{\mu}^Tg(\mathbf{x},\mathbf{u}) \\ 
    =&\displaystyle\sum_{t=0}^{T-1}\ell_t(x_t,u_t)+\ell_T(x_T) + \displaystyle\sum_{t=0}^{T-1}\lambda_{d,t+1}(f_t(x_t,u_t)-x_{t+1})+\lambda_b^Tr(x_0,x_T) + \displaystyle\sum_{t=0}^{T-1}\mu_t^Tg_t(x_t,u_t)\\
    =&\displaystyle\sum_{t=0}^{T}\mathcal{L}_t(x_t,u_t,\boldsymbol{\lambda},\boldsymbol{\mu})
\end{align*}
where
\begin{align*}
    \mathcal{L}_0(x_0,u_0,\boldsymbol{\lambda},\boldsymbol{\mu}) &= \ell_0(x_0,u_0)+\lambda_{d,1}^T f_0(x_0,u_0) + \lambda_{b,0}r_0(x_0)\\
    \mathcal{L}_t(x_t,u_t,\boldsymbol{\lambda}) &= \ell_t(x_t,u_t)+\lambda_1^T f_t(x_t,u_t)-\lambda_tx_t+\mu_t^Tg_t(x_t,u_t) \qquad t=1,\dots,T-1\\
    \mathcal{L}_T(x_T,\boldsymbol{\lambda}) &= \ell_T(x_T) - \lambda_T^Tx_T + \lambda_{b,T}^Tr_T(x_T)
\end{align*}
and we assumed $r(x_0,x_T)=\col(r_0(x_0),r_T(x_T))$ and $\lambda_b = \col(\lambda_{b,0},\lambda_{b,T})$.

Let $(\mathbf{x}^*,\mathbf{u}^*)$ be a regular point for the dynamics constraints and an optimal (state-input) trajectory. 
Then there exists $\boldsymbol{\lambda}^*$ and $\boldsymbol{\mu}^*>0$ such that $\nabla \mathcal{L}(\mathbf{x}^*,\mathbf{u}^*,\boldsymbol{\lambda}^*)=0$ + complementary slackness conditions.

Let us explicitly write condition $\nabla_{(1,2)} \mathcal{L}(\mathbf{x}^*,\mathbf{u}^*,\boldsymbol{\lambda}^*,\boldsymbol{\mu}^*) = 0 $ 
\[
    \nabla_{(1,2)} \mathcal{L}(\mathbf{x}^*,\mathbf{u}^*,\boldsymbol{\lambda}^*,\boldsymbol{\mu}^*) = \begin{bmatrix}
        \nabla_1\mathcal{L}(\mathbf{x}^*,\mathbf{u}^*,\boldsymbol{\lambda}^*,\boldsymbol{\mu}^*)\\
        \nabla_2\mathcal{L}(\mathbf{x}^*,\mathbf{u}^*,\boldsymbol{\lambda}^*,\boldsymbol{\mu}^*)
    \end{bmatrix} = 0
\]
Since $\mathcal{L}(\mathbf{x},\mathbf{u},\boldsymbol{\lambda}^*,\boldsymbol{\mu}^*)= \displaystyle\sum_{t=0}^{T}\mathcal{L}(x_t,u_t,\boldsymbol{\lambda})$, we can exploit this sparsity and write 
\begin{align*}
    \begin{bmatrix}
        \nabla_1\mathcal{L}_0(x_0,u_0,\boldsymbol{\lambda},\boldsymbol{\mu}) \\
        \nabla_2\mathcal{L}_0(x_0,u_0,\boldsymbol{\lambda},\boldsymbol{\mu}) 
    \end{bmatrix} = 0 \qquad &
    \begin{bmatrix}
        \nabla_1\ell(x_0,u_0)\nabla_1f_0(x_0,u_0)\lambda_1+\nabla r_0(x_0)\lambda_{b,0}+\nabla_1g_t(x_0,\mu_0)\\
        \nabla_2\ell_0(x_0,u_0)\nabla_2f_0(x_0,u_0)\lambda_1 + \nabla_2g_t(x_0,u_0)\mu_0\\
    \end{bmatrix} = 0\\
    \begin{bmatrix}
        \nabla_1\mathcal{L}_t(x_t,u_t,\boldsymbol{\lambda})\\
        \nabla_2\mathcal{L}_t(x_t,u_t,\boldsymbol{\lambda})
    \end{bmatrix} = 0 \qquad &  
    \begin{bmatrix}
        \nabla_1\ell_t(x_t,u_t) + \nabla_1f_t(x_t,u_t)\lambda_{t+1}-\lambda_t + \nabla_1g_t(x_t,u_t)\mu_t\\
        \nabla_2\ell_t(x_t,u_t) + \nabla_2f_t(x_t,u_t)\lambda_{t+1} + \nabla_2g_t(x_t,u_t)\mu_t
    \end{bmatrix} = 0 \quad t=1,\dots,T-1 \\
    \nabla_1\mathcal{L}_t(x_t,\boldsymbol{\lambda}) = 0\qquad & \nabla\ell_T(x_T)-\lambda_T + \nabla r_T(x_T)\lambda_{b,T} = 0
\end{align*}
for $(x_t,u_t,\lambda_t,\mu_t) = (x_t^*,u_t^*,\lambda_t^*,\mu_t^*)$


\chapter{Linear Quadratic (LQ) Optimal Control}

Consider a linear quadratic optimal control problem as: 
\begin{align*}
        \min_{\substack{x_1,\dots,x_T \\ u_0,\dots,u_{T-1}}} & \displaystyle\sum_{t=0}^{T-1}\displaystyle\frac{1}{2}[x_t^TQ_tx_t+u_t^TR_tu_t] + \displaystyle\frac{1}{2}x_T^TQ_Tx_T\\
        \text{subj. to } & x_{t+1} = A_tx_t + B_tu_t \quad t=0,\dots,T-1\\
                        &x_0 = x_{\text{init}}
\end{align*}
We assume $Q_t=Q_t^T\geq 0,\ $ for $ t=0,\dots,T-1$, $Q_T = Q_T^T \geq 0$, and $R_t=R_t^T > 0 \ $ for $t=0,\dots,T-1$

\section{First order optimality condition}

Considering the optimal control problem
\begin{align*}
        \min_{\substack{x_1,\dots,x_T \\ u_0,\dots,u_{T-1}}} & \displaystyle\sum_{t=0}^{T-1}\displaystyle \ell_t(x_t,u_t) + \ell_T(x_T)\\
        \text{subj. to } & x_{t+1} = f(x_t,u_t) \quad t=0,\dots,T-1\\
                        &x_0 = x_{\text{init}}
\end{align*}

\textbf{Recall} the first order necessary and sufficient conditions for $(\mathbf{x}^*,\mathbf{u}^*)$ to be an optimal trajectory $\exists \boldsymbol{\lambda}^*$ such that:
\begin{align*}
    x^*_{t+1} &= f_t(x^*_t, u^*_t) & t &= 0,\ldots,T-1 \\
    \lambda^*_t &= \nabla_1 f_t(x^*_t, u^*_t)\lambda^*_{t+1} + \nabla_1 \ell_t(x^*_t, u^*_t) & t &= T-1,\ldots,1 \\
    0 &= \nabla_2 f_t(x^*_t, u^*_t)\lambda^*_{t+1} + \nabla_2 \ell_t(x^*_t, u^*_t) & t &= 0,\ldots,T-1
\end{align*}
$\text{with } x^*_0 = x_{\text{init}} \text{ and } \lambda^*_T = \nabla\ell_T(x^*_T).$\\
In the case of the LQ problem becomes:
\begin{gather*}
    \nabla_1f_t(x_t,u_t) = A_t^T\\
    \nabla_1\ell(x_t,u_t) = \nabla_1(\displaystyle\frac{1}{2}x_t^TQ_tx_t+\displaystyle\frac{1}{2}u_t^TR_tu_t)=Q_tx_t\\
    \nabla_2f_t(x_t,u_t) = B_t^T\\
    \nabla_2\ell_t(x_t,u_t) = R_tu_t\\
\end{gather*}
Therefore :
\begin{align*}
    x^*_{t+1} &= A_t x^*_t + B_t u^*_t \quad t=0,\dots,T-1\\
    \lambda_t^* &= A_t^T \lambda_{t+1}^* + Q_t x_t^* \quad t=T-1,\dots,0\\
    0 &= B_t^T \lambda_{t+1}^* + R_t u_t^* \quad t=0,\dots,T-1
\end{align*}
$\text{with } x^*_0 = x_{\text{init}} \text{ and } \lambda^*_T = Q_Tx^*_T.$

\begin{remark}
    Note that $\lambda_t^*$ follows a "backward" update rule for the components of $\lambda$, with $\lambda_T = Q_Tx_T$.
\end{remark}

\begin{remark}
$A_t$ and $B_t$ are the Jacobians of $f_t(x_t,u_t)$ with respect to $x_t$ and $u_t$ respectively:
\[
    \nabla f_t(x_t,u_t) = \begin{bmatrix}
        \nabla_1f_t(x_t,u_t) & \nabla_2f_t(x_t,u_t)
    \end{bmatrix} = \begin{bmatrix}
        A_t & B_t
    \end{bmatrix}^T = \begin{bmatrix}
        \frac{\partial f_t(x_t,u_t)}{\partial x_t} & \frac{\partial f_t(x_t,u_t)}{\partial u_t}
    \end{bmatrix}^T
\]
\end{remark}

\begin{remark}
Second order optimality conditions:
\[
    y^T\nabla^2_{(1,2)(1,2)}\mathcal{L}(\mathbf{x}^*,\mathbf{u}^*,\boldsymbol{\lambda}^*,)y \geq 0
\]
\end{remark}
\begin{remark}
    For vectors $y$ satisfying the "linear approximation of the constraint". The hessian turns out as 
    \[
        \begin{bmatrix}
            Q_1 & & &  0\\
                & \ddots & & & \\ 
            0 & & & Q_n
        \end{bmatrix}
    \]    
\end{remark}

\subsubsection{One Possible Way in Finding the Optimal Solution (Open Loop)}

Having that $R_t > 0$ is always invertible, we can always write $u_t^* = -R_t^{-1}B_t^T\lambda_{t+1}^*$ for $t=0,...,T-1$ and then write the (equivalent to the previous necessary and sufficient conditions) linear system:

\[
    \begin{bmatrix}
        x_{t+1}^* \\ \lambda_t^*
    \end{bmatrix} = \begin{bmatrix}
        A_t & -B_tR_t^{-1}B_t^T \\ Q_t & R_t
    \end{bmatrix} \begin{bmatrix}
        x_t^* \\ \lambda_{t+1}^*
    \end{bmatrix} \quad t\in[0,T)
\]
with $x_0^* = x_{\text{init}}$ and $\lambda_T^* = Q_Tx_T^*$

This linear discrete-time system, which is in a so-called Hamiltonian form, has the peculiarity to involve two dynamics, one forward in time (for $x_t^*$) and one backward in time (for $\lambda_{t+1}^*$), respectively for $t$ from 0 to T-1 (for $x_t^*$) and for $t$ from T-1 back to 0 (for $\lambda_{t+1}^*$), with a fixed initial condition for $x_t^*$ and a fixed initial (in a sense) condition for $\lambda_{t+1}^*$ that depends on the final value for $x_t$ (two initial conditions with a dynamic in between).

Solving a system like that is possible (it's the so-called "two points boundary value problem") BUT in this case leads us to an $u_t^* = -R_t^{-1}B_t^T\lambda_{t+1}^*$ open-loop solution (suboptimal).

\subsubsection{Linear Dependence Between \texorpdfstring{$\lambda^*$}{λ*} and \texorpdfstring{$x^*$}{x*}}

The fact that it's definable a linear system that involves only the optimal state trajectory $x^*$ and the optimal set of Lagrange multipliers $\lambda^*$ indicates that between them there MUST be a linear dependence: $\lambda_t^* = P_tx_t^*$

Having for sure that such a relation exists and holds $\forall t=1...T$ (notice anyway that $\lambda$ has components for $t=1...T$ and $x$ for $t=0...T-1$ BUT also $x_T$ is defined), the way to compute $P_t$ $\forall t=1...T-1$ can be found by enveloping the computation $A_t^TP_{t+1}x_{t+1}^* + Q_tx_t^*$ (substituting $x_{t+1}^*$ and within inside it also $u_t^*$) and imposing it to be the same as $P_tx_t^*$ (notice in fact that $\lambda_{t+1}^* = P_{t+1}x_{t+1}^* \Rightarrow A_t^TP_{t+1}x_{t+1}^* + Q_tx_t^* = A_t^T\lambda_{t+1}^* + Q_tx_t^* = \lambda_t^* = P_tx_t^*$): this will bring to a relation $P_t = P_t(P_{t+1})$ and so $P_t$ $\forall t=1...T-1$ can be computed in an iterative way backward in time from $P_T = Q_T$ (recalling $\lambda_T^* = Q_Tx_T^*$). Notice: $P_t \geq 0$.

\section{LQ problem: Optimal solution}

Starting from $B_t^T\lambda_{t+1}^* + R_tu_t^* = 0$, ($R_t>0$ it is invertible) we can write:
\[
    u_t^* = -R_t^{-1}B_t^T\lambda_{t+1}^*
\]

Introducing a matrix $P_t = P_t^T \geq 0$, it can be proven that
\[
    \lambda_t^* = P_tx_t^*
\]
where $P_t \geq 0$ is defined $\forall t=1...T$.

Assuming that it holds for some $t \leq T-1$, then we have
\[
    u_t^* = -R_t^{-1}B_t^TP_{t+1}x_{t+1}^*
\]

Now, considering the constraint represented by the dynamics:
\[
    u_t^* = -R_t^{-1}B_t^TP_{t+1}(A_tx_t^* + B_tu_t^*)
\]

Solving for $u_t^*$ it follows:
\[
    u_t^* = -(R_t + B_t^TP_{t+1}B_t)^{-1}B_t^TP_{t+1}A_tx_t^* \quad t = 0,\ldots,T-1
\]

\begin{remark}
We have found that the optimal input is not just an open loop function of time BUT it's a feedback-defined quantity, function of the optimal state.
\end{remark}

We thus obtain:
\begin{align*}
    x_{t+1}^* &= A_tx_t^* + B_tu_t^* \\
    &= A_tx_t^* - B_t(R_t + B_t^TP_{t+1}B_t)^{-1}B_t^TP_{t+1}A_tx_t^*
\end{align*}

Multiplying both sides with $A_t^TP_{t+1}$ we obtain:
\[
    A_t^T\underbrace{P_{t+1}x_{t+1}^*}_{\lambda_{t+1}^*} = A_t^TP_{t+1}A_tx_t^* - A_t^TP_{t+1}B_t(R_t + B_t^TP_{t+1}B_t)^{-1}B_t^TP_{t+1}A_tx_t^*
\]

Adding $Q_tx_t^*$ to both sides and collecting $x_t^*$, we obtain:
\[
    \underbrace{A_t^T\lambda_{t+1}^* + Q_tx_t^*}_{\lambda_t^*} = \underbrace{\left[A_t^TP_{t+1}A_t - A_t^TP_{t+1}B_t(R_t + B_t^TP_{t+1}B_t)^{-1}B_t^TP_{t+1}A_t + Q_t\right]}_{P_t}x_t^*
\]
\[
    \lambda_t^* = P_tx_t^*
\]
where we have used the first-order necessary condition $\lambda_t^* = A_t^T\lambda_{t+1}^* + Q_tx_t^*$.

Starting from the terminal condition $P_T = Q_T$, matrix $P_t$ can be computed by:
\[
    P_t = A_t^TP_{t+1}A_t - A_t^TP_{t+1}B_t(R_t + B_t^TP_{t+1}B_t)^{-1}B_t^TP_{t+1}A_t + Q_t
\]

Defining the gain matrix $K_t^*$ as:
\[
    K_t^* = -(R_t + B_t^TP_{t+1}B_t)^{-1}B_t^TP_{t+1}A_t
\]

the optimal control law for the linear quadratic problem is:
\begin{align*}
    x_{t+1}^* &= A_tx_t^* + B_tu_t^* & t &= 0,\ldots,T-1 & x_0 &= x_{\text{init}} \\
    u_t^* &= K_t^*x_t^* & t &= 0,\ldots,T-1
\end{align*}

where $P_t$ is obtained by backward integration:
\begin{align*}
    P_T &= Q_T \\
    P_t &= A_t^TP_{t+1}A_t - A_t^TP_{t+1}B_t(R_t + B_t^TP_{t+1}B_t)^{-1}B_t^TP_{t+1}A_t + Q_t \quad t = T-1,\ldots,0
\end{align*}
which is known as the Difference Riccati equation.

\begin{remark}
The optimal control law $u_t^*$ is a state feedback controller. Notice that the obtained solution (feedback control law) has a similar structure to Kalman filtering; indeed, it can be shown that optimal control problem and optimal estimation problem are one the dual of the other.
\end{remark}

\begin{remark}
Therefore, by propagating the Riccati equation back in time, $P_t$ can be calculated. The Riccati equation is called difference Riccati equation
    \begin{itemize}
        \setlength{\itemsep}{0pt}
        \item gains $K_t^*$ can be precomputed offline and then used for different $x_0$ 
        \item It can be shown that if $T\to\infty$ the gains $K_t^*$ converge and asymptotically stabilize the system
    \end{itemize}
\end{remark}

\subsubsection{Other formulations of the Riccati equation}

The usual Riccati recursion reads:
\[
    P_t = Q_t + A_t^TP_{t+1}A_t - A_t^TP_{t+1}B_t(R_t + B_t^TP_{t+1}B_t)^{-1}B_t^TP_{t+1}A_t
\]

By exploiting the matrix inversion lemma [$(A+BC)^{-1} = A^{-1}-A^{-1}B(I+CA^{-1}B)^{-1}CA^{-1}$], we can write:
\begin{align*}
    P_t &= Q_t + A_t^TP_{t+1}A_t - A_t^TP_{t+1}B_t(R_t + B_t^TP_{t+1}B_t)^{-1}B_t^TP_{t+1}A_t\\
    &= Q_t + A_t^TP_{t+1}(I-B_t(R_t + B_t^TP_{t+1}B_t)^{-1}B_t^TP_{t+1})A_t\\
    &= Q_t + A_t^TP_{t+1}(I-B_t((I+B_t^TP_{t+1}B_tR_t^{-1})R_t)^{-1}B_t^TP_{t+1})A_t\\
    &= Q_t + A_t^TP_{t+1}(I-B_tR_t^{-1}(I+B_t^TP_{t+1}B_tR_t^{-1})^{-1}B_t^TP_{t+1})A_t\\
    &= Q_t + A_t^T(I+P_{t+1}BR_t^{-1}B_t^T)^{-1}P_{t+1}A_t
\end{align*}

\begin{remark}
This alternative formulation of the Riccati equation is numerically more stable compared to the standard form.
\end{remark}


\section{Infinite horizion LQ optimal control}
Let's now study, for time evolving up to infinity, the "individuated" optimal feedback control law and whether it stabilizes and steers the state to zero.\\
Consider the infinite-horizon optimal control problem
\begin{align*}
        \min_{\substack{x_1,x_2,\dots \\ u_0,u_1,\dots}} & \displaystyle\sum_{t=0}^{\infty}\displaystyle\frac{1}{2}\left[x_t^TQx_t+u_t^TRu_t\right] \\
        \text{subj. to } & x_{t+1} = Ax_t + Bu_t \quad t=0,1,\dots\\
                        &x_0 = x_{\text{init}}
\end{align*}
where 
\begin{itemize}
    \item $x\in\R^n$ and $u\in\R^m$
    \item $A\in\R^{n\times n}$
    \item $B\in\R^{n\times m}$
    \item $Q\in\R^{n\times n}$ and $Q=Q^T\geq 0$
    \item $R\in\R^{m\times m}$ and $R=R^T> 0$
\end{itemize}

\begin{remark}
The pair $(A,B)$ is controllable and the pair $(A,C)$, with $Q=C^TC$, is observable.
\end{remark}

\begin{remark}
The controllability assumption guarantees a finite cost. Indeed, if the system is controllable, there exists a sequence driving the state to zero in finite time, so that the cost is finite.
\end{remark}

\begin{remark}
Once the horizon is infinite, we have the possibility of the cost diverging to infinity, and we need to introduce some suppositions in order that a possibility for that cost function to be finite surely exists.
\end{remark}

\begin{remark}
This control law has been proven to be asymptotically stabilizing - a remarkable result as we have managed to stabilize a time-variant linear system!
\end{remark}

\subsubsection{More about Dante's Inferno}

We assume the pair $(A,B)$ is controllable and the pair $(A,C)$ with $Q=C^TC$ is observable.\\
Let us write 
\[
    y_t=Cx_t
\]
which leads to 
\[
    \displaystyle\frac{1}{2}x_t^TQx_t = \displaystyle\frac{1}{2}x_t^TC^TCx_t = \displaystyle\frac{1}{2}y_t^Ty_t
\]
The controllability assumption guardantees that an optimal controller exists: if $(A,B)$ controllable, then $\exists \bar{u}_0,\dots,\bar{u}_{T-1}$ for $T$ sufficiently large ($T=n$) such that $\forall x_0\in \R^n\implies x_T=0$. Consider the input
\[
    \bar{u}_0,\dots,\bar{u}_{T-1},0,\dots,0,\dots
\]
Let us compute the cost associated to this input 
\[
    \displaystyle\sum_{t=0}^{\infty}\displaystyle\frac{1}{2}[x_t^TQx_t+u_t^TRu_t] =\displaystyle\sum_{t=0}^{T-1}\displaystyle\frac{1}{2}\bar{x}_t^TQ\bar{x}_t + \displaystyle\frac{1}{2}\bar{u}_t^TR\bar{u}_t
\]
We can note that the cost is a finite quantity. Because the cost is finite, There must exist a solution which minimizes the cost.
\proposition

Let the pair $(A,B)$ be controllable and the pair $(A,C)$ with $Q=C^TC$ be observable. Then the following holds:
\begin{itemize}
    \item there exists a unique positive definite $P_\infty$ equilibrium solution of the Difference Riccati Equation. That is, $P_\infty$ is a solution of \[
            P_\infty = Q+A^TP_\infty A-A^TP\infty B(R+B^TP_\infty B)^{-1}B^TP_\infty A
        \]
        which is called \emph{Algebraic Riccati Equation }
    \item the optimal control is a feedback of the state given by:
        \begin{flalign*}
            & K^* = -(R+B^TP_\infty B)^{-1}(B^TP_\infty A)\\
            & u_t^* = K^*x^*_t\\
            & x_{t+1}^* = Ax_t^*+Bu_t^* \quad t=1,2,\dots \quad x_0^* = x_{\text{init}}
        \end{flalign*}
        and it asymptotically stabilizes the system (It's called Linear Quadratic Regulator).
\end{itemize}
\remark The observability of $(A,C)$ guardantees that if the stage cost goes to zero, then the state trajectory goes to zero.

\subsubsection{Linear Quadratic Regulator $u_t = K_t^*x_t$ for LTV Systems}

It can be demonstrated that considering the time-varying system
\[
    x_{t+1} = A_tx_t + B_tu_t
\]
under the conditions introduced in the previous slides, if the time horizon is sent to infinity $(T\to+\infty)$ then the optimal feedback $K_t^*$ (as introduced before), for $t\geq 0$, is exponentially stabilizing if $x_{t+1} = (A_t + B_tK_t^*)x_t$ is exponentially stable (alias $A_t + B_tK_t^*$ is Hurwitz).

\begin{remark}
Note that for linear systems "exponentially stable" simply means "asymptotically stable".
\end{remark}

\begin{remark}
So the said controller is stabilizing also in the LTV case (not only LTI).
\end{remark}

\subsubsection{The Role of the Controllability Assumption - Demonstration}

Let's consider a linear time-invariant system: $x_{t+1} = Ax_t + Bu_t$

Having $(A,B)$ controllable means that $\forall x_0 \exists \bar{u}$ : $\bar{x}$ associated to $\bar{u}$ has $\bar{x}_T = 0$.
Then, if $\bar{x}_T = 0$ then $\bar{u}_t = 0 \: \forall t \geq T \Rightarrow \bar{x}_t = 0 \: \forall t \geq T$

Under these hypothesis:
\[
    \ell(\bar{x},\bar{u}) = \sum_{t=0}^{+\infty}\frac{1}{2}(\bar{x}_t^TQ\bar{x}_t + \bar{u}_t^TR\bar{u}_t) = \sum_{t=0}^{T-1}\frac{1}{2}(\bar{x}_t^TQ\bar{x}_t + \bar{u}_t^TR\bar{u}_t)
\]
which is both a finite cost and also lower bounded by zero, alias, an optimal solution indeed surely exists!

\subsubsection{The Role of the Observability Assumption - Demonstration}

Let's consider a linear time-invariant system: $x_{t+1} = Ax_t + Bu_t$

Considering $y_t = Cx_t$ with $C$ defined such that $Q = C^TC$, then we can write:
\[
    \ell(\bar{x},\bar{u}) = \sum_{t=0}^{+\infty}\frac{1}{2}(\bar{x}_t^TQ\bar{x}_t + \bar{u}_t^TR\bar{u}_t) = \sum_{t=0}^{+\infty}\frac{1}{2}(\bar{x}_t^TC^TC\bar{x}_t + \bar{u}_t^TR\bar{u}_t)
\]
meaning that in general the cost depends on the output $y$ and it's in general possible having them bounded but with the state trajectory diverging (recall that the output depends in general only on the observable state, not the unobservable one, that is totally uncorrelated with it). The observability of $(A,C)$ guarantees that it is not possible this situation, is not possible for the cost to be bounded if the state trajectory diverges (alias $\bar{x}$ diverging $\Rightarrow$ cost diverging, unfeasible).

\section{Affine LQR}

\subsection{Problem Formulation}
Consider a LQR problem with affine cost and affine dynamics:
\begin{align*}
    \min_{\substack{x_1,\dots,x_T\\u_0,\dots,u_{T-1}}} & \displaystyle\sum_{t=0}^{T-1} \begin{bmatrix}
        q_t \\ r_t
    \end{bmatrix}^T \begin{bmatrix}
        x_t \\ u_t
    \end{bmatrix} + \frac{1}{2}\begin{bmatrix}
        x_t \\ u_t
    \end{bmatrix}^T \begin{bmatrix}
        Q_t & S_t^T \\ S_t & R_t
    \end{bmatrix} \begin{bmatrix}
        x_t \\ u_t
    \end{bmatrix} + q_T^Tx_T + \frac{1}{2}x_T^TQ_Tx_T\\
    \text{subj. to } & x_{t+1} = A_tx_t + B_tu_t + c_t \quad t = 0,\dots,T-1
\end{align*}
where:
\begin{itemize}
    \item $Q_t\in\R^{n_x\times n_x}$ and $Q_t = Q_t^T \geq 0$ for all $t = 0,\dots,T$
    \item $R_t\in\R^{n_u\times n_u}$ and $R_t = R_t^T > 0$ for all $t = 0,\dots,T-1$
    \item $S_t\in\R^{n_u\times n_x}$ such that the problem is convex\footnote{Namely, $Q_t - S_t^TR_t^{-1}S_t$ positive semi-definite.}
\end{itemize}

\subsection{Solution via State Augmentation}
It can be conveniently solved by augmenting the state as:
\[
    \tilde{x}_t := \begin{bmatrix}
        1 \\ x_t
    \end{bmatrix}
\]

We can rewrite the cost and system matrices as:
\begin{align*}
    \tilde{Q}_t &:= \begin{bmatrix}
        0 & q_t^T \\ q_t & Q_t
    \end{bmatrix} & \tilde{S}_t &:= \begin{bmatrix}
        r_t & S_t
    \end{bmatrix} & \tilde{R}_t &:= R_t \\
    \tilde{A}_t &:= \begin{bmatrix}
        1 & 0 \\ c_t & A_t
    \end{bmatrix} & \tilde{B}_t &:= \begin{bmatrix}
        0 \\ B_t
    \end{bmatrix}
\end{align*}

Then, we solve the associated LQR problem:
\begin{align*}
    \min_{\substack{\tilde{x}_1,\dots,\tilde{x}_T\\u_0,\dots,u_{T-1}}} & \displaystyle\sum_{t=0}^{T-1}\frac{1}{2}\begin{bmatrix}
        \tilde{x}_t \\ u_t
    \end{bmatrix}^T \begin{bmatrix}
        \tilde{Q}_t & \tilde{S}_t^T \\ \tilde{S}_t & \tilde{R}_t
    \end{bmatrix} \begin{bmatrix}
        \tilde{x}_t \\ u_t
    \end{bmatrix} + \frac{1}{2}\tilde{x}_T^T\tilde{Q}_T\tilde{x}_T\\
    \text{s.t. } & \tilde{x}_{t+1} = \tilde{A}_t\tilde{x}_t + \tilde{B}_tu_t \quad t = 0,\dots,T-1\\
    & x_0 = x^0
\end{align*}

\subsection{Optimal Solution}
The optimal solution of the problem reads:
\begin{align*}
    u_t^* &= K_t^*x_t^* + \sigma_t^* & t &= 0,\dots,T-1\\
    x_{t+1}^* &= A_tx_t^* + B_tu_t^*
\end{align*}
where:
\begin{align*}
    K_t^* &= -(R_t + B_t^TP_{t+1}B_t)^{-1}(S_t + B_t^TP_{t+1}A_t)\\
    \sigma_t^* &= -(R_t + B_t^TP_{t+1}B_t)^{-1}(r_t + B_t^Tp_{t+1} + B_t^TP_{t+1}c_t)\\
    p_t &= q_t + A_t^Tp_{t+1} + A_t^TP_{t+1}c_t - {K_t^*}^T(R_t + B_t^TP_{t+1}B_t)\sigma_t^*\\
    P_t &= Q_t + A_t^TP_{t+1}A_t - {K_t^*}^T(R_t + B_t^TP_{t+1}B_t)K_t^*
\end{align*}
with $p_T = q_T$ and $P_T = Q_T$.



\chapter{Optimality Conditions for Unconstrained Optimal Control via Shooting}
\section{Recall: Discrete-time Control System }
We consider nonlinear, discrete-time systems described by
\[
    x_{t+1} = f_t(x_t, u_t) \qquad t \in \mathbb{N}_0
\]
where $x_t \in \mathbb{R}^n$ and $u_t \in \mathbb{R}^m$ are the state and the input of the system at time $t$.

\subsubsection{Notation}
We use $\mathbf{x} \in \mathbb{R}^{nT}$ and $\mathbf{u} \in \mathbb{R}^{mT}$ to denote, respectively, the stack of the states $x_t$ for all $t \in 1,\ldots,T$ and the inputs $u_t$ for all $t \in 0,\ldots,T-1$, that is
\begin{align*}
    \mathbf{x} &:= \col(x_1,\ldots,x_T) \\
    \mathbf{u} &:= \col(u_0,\ldots,u_{T-1})
\end{align*}
where $\col(x_1,\ldots,x_T)$ is the (column) vector with components $x_1,\ldots,x_T$.

\section{Recall: System Trajectories}
We start by introducing the definition of trajectory of a (possibly) nonlinear, discrete-time control system.

\begin{definition}
A pair $(\bar{\mathbf{x}}, \bar{\mathbf{u}}) \in \mathbb{R}^{nT} \times \mathbb{R}^{mT}$ is called a \emph{trajectory} of the system if its components satisfy the constraint represented by the dynamics for all $t \in 0,\ldots,T-1$. In particular, $\bar{\mathbf{x}}$ is the state trajectory, while $\bar{\mathbf{u}}$ is the input trajectory.
\end{definition}

\begin{remark}
It will be sometimes useful to refer to a generic pair $(\boldsymbol{\alpha}, \boldsymbol{\mu}) \in \mathbb{R}^{nT} \times \mathbb{R}^{mT}$ with $\boldsymbol{\alpha} := \col(\alpha_1,\ldots,\alpha_T)$ and $\boldsymbol{\mu} := \col(\mu_0,\ldots,\mu_{T-1})$ as a state-input curve. Notice that a curve $(\boldsymbol{\alpha}, \boldsymbol{\mu})$ is not necessarily a trajectory, i.e., it does not necessarily satisfy the dynamics.
\end{remark}

Let us rewrite the nonlinear dynamics as an implicit equality constraint $h : \mathbb{R}^{nT} \times \mathbb{R}^{mT} \to \mathbb{R}^{nT}$:
\[
    h(\mathbf{x}, \mathbf{u}) := \begin{bmatrix}
        f_t(x_0, u_0) - x_1 \\
        \vdots \\
        f_t(x_{T-1}, u_{T-1}) - x_T
    \end{bmatrix}
\]
so that a curve $(\bar{\mathbf{x}}, \bar{\mathbf{u}})$ is a trajectory if it satisfies the (possibly nonlinear) equality constraint
\[
    h(\bar{\mathbf{x}}, \bar{\mathbf{u}}) = 0.
\]

\begin{remark}
Recall that a nonlinear equality constraint is a nonconvex constraint.
\end{remark}

\subsection{Recall: Trajectory Manifold}
We can now define the trajectory manifold $\mathcal{T} \subset \mathbb{R}^{nT} \times \mathbb{R}^{mT}$ of the system.

\begin{definition}
The trajectory manifold $\mathcal{T} \subset \mathbb{R}^{nT} \times \mathbb{R}^{mT}$ is defined as
\[
    \mathcal{T} := \{(\mathbf{x}, \mathbf{u}) \in \mathbb{R}^{nT} \times \mathbb{R}^{mT} \mid h(\mathbf{x}, \mathbf{u}) = 0\}.
\]
\end{definition}

\begin{figure}[ht]
    \centering
    \includegraphics[width=0.5\textwidth]{manifold}
    \caption{Trajectory manifold $\mathcal{T}$}
\end{figure}

\begin{remark}
It can be shown that the tangent space to the trajectory manifold at a given trajectory (point), denoted as $T_{(\bar{\mathbf{x}},\bar{\mathbf{u}})}\mathcal{T}$, is represented by the set of trajectories satisfying the linearization of the nonlinear dynamics $f_t(\cdot,\cdot)$ about the trajectory $(\bar{\mathbf{x}}, \bar{\mathbf{u}})$.
\end{remark}

\section{Recall: Unconstrained Optimal Control Problem (D-T)}

We look for a solution of the discrete-time optimal control problem
\begin{align}
    \min_{\mathbf{x}\in\mathbb{R}^n,\mathbf{u}\in\mathbb{R}^m} & \sum_{t=0}^{T-1}\ell_t(x_t,u_t)+\ell_T(x_T) \\
    \text{subj.to } & x_{t+1} = f_t(x_t,u_t), \quad t \in 0,\ldots,T-1
\end{align}
with given initial condition $x_0 = x^0 \in \mathbb{R}^n$, where $\ell_t : \mathbb{R}^n \times \mathbb{R}^m \to \mathbb{R}$ is the so-called stage cost while $\ell_T : \mathbb{R}^n \to \mathbb{R}$ is the terminal cost.

\begin{assumption}
Functions $\ell_t(\cdot,\cdot)$, $\ell_T(\cdot)$, $f_t(\cdot,\cdot)$ are twice continuously differentiable, i.e., they are class $\mathcal{C}^2$.
\end{assumption}

\subsection{Recall: Compact Notation}
Consider the discrete-time system $x_{t+1} = f_t(x_t,u_t)$. We can introduce the compact notation
\[
    \mathbf{x} := \begin{bmatrix}
        x_1 \\ \vdots \\ x_T
    \end{bmatrix} \qquad
    \mathbf{u} := \begin{bmatrix}
        u_0 \\ \vdots \\ u_{T-1}
    \end{bmatrix}
\]
which lets us write compactly the cost function as
\[
    \ell(\mathbf{x},\mathbf{u}) := \sum_{t=0}^{T-1}\ell_t(x_t,u_t)+\ell_T(x_T)
\]
and the equality constraint represented by the dynamics
\[
    h(\mathbf{x},\mathbf{u}) := \begin{bmatrix}
        f_t(x_0,u_0)-x_1 \\
        \vdots \\
        f_t(x_{T-1},u_{T-1})-x_T
    \end{bmatrix}
\]

\section{Recall: Equality Constrained Optimization}
In light of this compact notation, we can rewrite the optimal control problem as
\begin{align*}
    \min_{\mathbf{x}\in\mathbb{R}^{nT},\mathbf{u}\in\mathbb{R}^{mT}} & \ell(\mathbf{x},\mathbf{u})\\
    \text{subj.to } & h(\mathbf{x},\mathbf{u})=0
\end{align*}
where $\ell : \mathbb{R}^{nT} \times \mathbb{R}^{mT} \to \mathbb{R}$ and $h : \mathbb{R}^{nT} \times \mathbb{R}^{mT} \to \mathbb{R}^{nT}$.
This is a constrained nonlinear optimization problem with decision variable $(\mathbf{x},\mathbf{u})$.

\section{Recall: Optimality Conditions}
Let $z^*$ be a regular local minimum of the equality-constrained minimization problem:
\begin{align*}
    \min_{z\in\mathbb{R}^n} & \ell(z)\\
    \text{subj.to } & h(z)=0
\end{align*}
where $\ell : \mathbb{R}^n \to \mathbb{R}$ and $h : \mathbb{R}^n \to \mathbb{R}^n$.
Then
\[
    \nabla\ell(z^*) + \nabla h(z^*)\lambda^* = 0, \quad h(z^*) = 0
\]
which can be written as
\[
    \nabla\mathcal{L}(z^*,\lambda^*) = \begin{bmatrix}
        \nabla\ell(z^*) + \nabla h(z^*)\lambda^* \\
        h(z^*)
    \end{bmatrix} = 0
\]

\begin{remark}
If you have doubts go back to Part ``Equality and Inequality Constrained Optimization''.
\end{remark}


\section{Shooting Method in Optimal Control}

\subsection{State Expression via Input Sequence}
Consider a discrete-time system with dynamics
\[
    x_{t+1} = f_t(x_t, u_t), \quad x_0 = x^0
\]
where $x_0 = x^0$ is a given initial condition and $\mathbf{u} = \col(u_0,\ldots,u_T)$ is the input sequence.

The key idea of the shooting method is to express each state $x_t$ at time $t = 1,\ldots,T$ as a function of only the input sequence $\mathbf{u}$. This leads to:

\begin{align*}
    x_1 &= f_0(x_0, u_0) := \phi_1(\mathbf{u})\\
    x_2 &= f_1(f_0(x_0, u_0), u_1) := \phi_2(\mathbf{u})\\
    &\vdots\\
    x_t &= f_{t-1}(f_{t-2}(\ldots, u_{t-2}), u_{t-1}) := \phi_t(\mathbf{u})\\
    &\vdots\\
    x_T &= f_{T-1}(f_{T-2}(\ldots, u_{T-2}), u_{T-1}) := \phi_T(\mathbf{u})
\end{align*}

\subsection{Compact Notation}
For all $t$, we can define a map $\phi_t: \mathbb{R}^m \to \mathbb{R}^n$ such that:
\[
    x_t := \phi_t(\mathbf{u})
\]

This allows us to write in compact notation:
\[
    \phi(\mathbf{u}) = \col(\phi_1(\mathbf{u}),\ldots,\phi_T(\mathbf{u}))
\]
leading to:
\[
    \mathbf{x} = \phi(\mathbf{u})
\]

\begin{remark}
Given an input sequence $\mathbf{u}$, the equality constraint $\phi_{t+1}(\mathbf{u}) = f_t(\phi_t(\mathbf{u}), u_t)$ is satisfied by construction, as the system dynamics are embedded within the definition of $\phi$. This is equivalent to the equality constraint for the optimal control problem.
\end{remark}

\section{Reduced Optimal Control Problem}\label{shoot}

\subsection{Problem Reformulation}
We can define a reduced cost function:
\begin{align*}
    J(\mathbf{u}) &:= \sum_{t=0}^{T-1}\ell_t(\phi_t(\mathbf{u}), u_t) + \ell_T(\phi_T(\mathbf{u}))\\
    &= \ell(\phi(\mathbf{u}), \mathbf{u})
\end{align*}

This allows us to recast the optimal control problem into its reduced (or condensed) form:
\[
    \min_{\mathbf{u}\in\mathbb{R}^{mT}} J(\mathbf{u})
\]
which is an unconstrained optimization problem with cost function $J: \mathbb{R}^{mT} \to \mathbb{R}$ where the optimization variable is only the input sequence $\mathbf{u} \in \mathbb{R}^{mT}$.

\begin{remark}
This is an unconstrained optimization problem in $\mathbf{u}$ with a $\mathcal{C}^2$ cost function. The cost function $J(\mathbf{u})$ inherits from the original optimal control problem both its smoothness properties and its nonconvexity.
\end{remark}

\subsection{Extension to Constrained Cases}
The reduced formulation can be extended to handle various types of constraints:

\subsubsection{Input Path Constraints}
If we consider input path constraints of the form:
\begin{align*}
    g_0(u_0) &\leq 0 \\
    &\vdots \\
    g_{T-1}(u_{T-1}) &\leq 0
\end{align*}
The problem becomes:
\begin{align*}
    \min_{\mathbf{u}\in\mathbb{R}^{mT}} \quad & J(\mathbf{u}) \\
    \text{subject to} \quad & g_t(u_t) \leq 0, \quad t = 0,\ldots,T-1
\end{align*}

\subsubsection{State-Input Path Constraints}
For constraints involving both states and inputs:
\begin{align*}
    g_0(x_0,u_0) &\leq 0\\
    &\vdots \\
    g_{T-1}(x_{T-1},u_{T-1}) &\leq 0
\end{align*}
These can be rewritten as functions of $x_0$ and $\mathbf{u}$ only by substituting $x_t = \phi_t(\mathbf{u})$. However, this requires $\phi(\cdot)$ to be explicitly known.

\begin{remark}
The shooting method refers to this procedure of expressing $\mathbf{x}$ as a function of $\mathbf{u}$ and substituting it into the optimal control problem, effectively reducing the number of optimization variables.
\end{remark}

\section{Algorithms for optimal control problem solution}

\subsection{Recall: Optimality Conditions for Unconstrained Optimization}

\subsubsection{First Order Necessary Condition (FNC)}
Let $\mathbf{u}^*$ be an unconstrained local minimum of $J: \mathbb{R}^{mT} \to \mathbb{R}$ and assume that $J$ is continuously differentiable ($\mathcal{C}^1$) in $B(\mathbf{u}^*, \epsilon)$ for some $\epsilon > 0$. Then:
\[
    \nabla J(\mathbf{u}^*) = 0
\]

\begin{remark}
To find this unconstrained local minimum, we can implement a gradient method, i.e., iteratively refine the solution according to:
\[
    \mathbf{u}^{k+1} = \mathbf{u}^k - \gamma\nabla J(\mathbf{u}^k)
\]
where $\gamma > 0$ is the step size.
\end{remark}

\subsubsection{Important Notation Distinctions}
For clarity, we distinguish between:
\begin{itemize}
    \item $\mathbf{u}^k$ denotes the input sequence $\mathbf{u} \in \mathbb{R}^{mT}$ at iteration $k$ of the optimization
    \item $u_t$ denotes the input $u \in \mathbb{R}^m$ at time $t$ in the dynamics
    \item $u_t^k$ denotes the input $u \in \mathbb{R}^m$ at time $t$ at iteration $k$ of the optimization algorithm
\end{itemize}

\subsection{Expression of \texorpdfstring{$\nabla J(\mathbf{u})$}{Expression of the gradient of J(u)}}
We can formally write the expression of $\nabla J(\mathbf{u}) = \nabla \ell(\phi(\mathbf{u}), \mathbf{u})$ by using the chain rule of differentiation.

Consider
\[
    J(\mathbf{u}) = \ell(\phi(\mathbf{u}), \mathbf{u})
\]

For the scalar case $\mathbf{u} \in \mathbb{R}$, we have:
\[
    \frac{d}{d\mathbf{u}}J(\bar{\mathbf{u}}) = \left.\frac{\partial \ell(\mathbf{x}, \mathbf{u})}{\partial \mathbf{x}}\right|_{\substack{\mathbf{x}=\phi(\bar{\mathbf{u}})\\ \mathbf{u}=\bar{\mathbf{u}}}} \cdot \left.\frac{\partial \phi(\mathbf{u})}{\partial \mathbf{u}}\right|_{\mathbf{u}=\bar{\mathbf{u}}} + \left.\frac{\partial \ell(\mathbf{x}, \mathbf{u})}{\partial \mathbf{u}}\right|_{\substack{\mathbf{x}=\phi(\bar{\mathbf{u}})\\ \mathbf{u}=\bar{\mathbf{u}}}} \cdot \underbrace{\frac{\partial \mathbf{u}}{\partial \mathbf{u}}}_{=1}
\]

In the general case where $\mathbf{u} \in \mathbb{R}^{mT}$, this extends to:
\[
    \nabla J(\mathbf{u}) = \nabla\phi(\mathbf{u})\nabla_1\ell(\phi(\mathbf{u}), \mathbf{u}) + \nabla_2\ell(\phi(\mathbf{u}), \mathbf{u})
\]

\begin{remark}
However, notice that the calculation of $\nabla J(\mathbf{u})$ requires $\nabla\phi(\mathbf{u})$ which may be difficult to compute.
\[
    \nabla\phi(\mathbf{u}) = \nabla \begin{bmatrix}
        \Phi_{1,1}(\mathbf{u})\\
        \Phi_{1,2}(\mathbf{u})\\
        \vdots \\
        \Phi_{t,1}(\mathbf{u})\\
        \Phi_{t,2}(\mathbf{u})\\
        \vdots
    \end{bmatrix}
= \begin{bmatrix}
        \displaystyle\frac{\partial \Phi_{1,1}}{\partial u_0} & \displaystyle\frac{\partial \Phi_{1,2}}{\partial u_0} & \cdots & \displaystyle\frac{\partial \Phi_{T,n}}{\partial u_0}\\
        \vdots & \vdots & \ddots & \vdots \\
        \displaystyle\frac{\partial \Phi_{1,1}}{\partial u_{T-1}} & \displaystyle\frac{\partial \Phi_{1,2}}{\partial u_{T-1}} &\cdots & \displaystyle\frac{\partial \Phi_{T,n}}{\partial u_{T-1}}
    \end{bmatrix}
\]
where $\Phi_{t,j}:\R^{mT}\to \R$, therefore the above matrix is a matrix of scalars.
\end{remark}


Let us introduce an auxiliary function $\mathcal{L}(\mathbf{x}, \mathbf{u}, \boldsymbol{\lambda}) : \mathbb{R}^{nT} \times \mathbb{R}^{mT} \times \mathbb{R}^{nT} \to \mathbb{R}$ given by

\[
\mathcal{L}(\mathbf{x}, \mathbf{u}, \boldsymbol{\lambda}) = \ell(\mathbf{x}, \mathbf{u}) + h(\mathbf{x}, \mathbf{u})^T\boldsymbol{\lambda}
\]

where $\boldsymbol{\lambda} \in \mathbb{R}^{nT}$ is a "costate vector".

To compute $\nabla J(\mathbf{u})$, let us evaluate $\mathcal{L}(\mathbf{x}, \mathbf{u}, \boldsymbol{\lambda})$ for $\mathbf{x} = \phi(\mathbf{u})$. Since $h(\phi(\mathbf{u}), \mathbf{u}) = 0$, it holds that

\[
\mathcal{L}(\phi(\mathbf{u}), \mathbf{u}, \boldsymbol{\lambda}) = \ell(\phi(\mathbf{u}), \mathbf{u}) + h(\phi(\mathbf{u}), \mathbf{u})^T\boldsymbol{\lambda} = J(\mathbf{u})
\]

for all $\boldsymbol{\lambda} \in \mathbb{R}^{nT}$. Therefore,

\[
\nabla_{\mathbf{u}}\mathcal{L}(\phi(\mathbf{u}), \mathbf{u}, \boldsymbol{\lambda}) = \nabla J(\mathbf{u}) \quad \forall \boldsymbol{\lambda}
\]

\begin{remark}
The function $\mathcal{L}(\mathbf{x}, \mathbf{u}, \boldsymbol{\lambda})$ is a Lagrangian, but here it will be used in a different way as an auxiliary function to compute $\nabla J(\mathbf{u})$.
\end{remark}

\begin{remark}
Recall that $\nabla_{\mathbf{u}}\mathcal{L}(\phi(\mathbf{u}), \mathbf{u}, \boldsymbol{\lambda})$ indicates the (total) gradient of the function $\mathcal{L}(\phi(\mathbf{u}), \mathbf{u}, \boldsymbol{\lambda})$ with respect to $\mathbf{u}$ (with $\boldsymbol{\lambda}$ independent of $\mathbf{u}$).
\end{remark}

Therefore, we can compute $\nabla J(\mathbf{u})$ as $\nabla \mathcal{L}(\phi(\mathbf{u}), \mathbf{u}, \boldsymbol{\lambda})$ and use $\boldsymbol{\lambda}$ as a degree of freedom to be properly used to ease the computation.

We can write:
\begin{align*}
\nabla J(\mathbf{u}) &= \nabla\phi(\mathbf{u})(\nabla_1\ell(\phi(\mathbf{u}), \mathbf{u}) + \nabla_1h(\phi(\mathbf{u}), \mathbf{u})\boldsymbol{\lambda}) \\
&+ \nabla_2\ell(\phi(\mathbf{u}), \mathbf{u}) + \nabla_2h(\phi(\mathbf{u}), \mathbf{u})\boldsymbol{\lambda}
\end{align*}

which holds for every $\boldsymbol{\lambda}$. Therefore, for a given $\mathbf{u}$, we can cleverly select $\boldsymbol{\lambda} = \boldsymbol{\lambda}(\mathbf{u})$ such that:

\[
\nabla_1\ell(\phi(\mathbf{u}), \mathbf{u}) + \nabla_1h(\phi(\mathbf{u}), \mathbf{u})\boldsymbol{\lambda}(\mathbf{u}) = 0
\]

which leads to:

\[
\nabla J(\mathbf{u}) = \nabla_2\ell(\phi(\mathbf{u}), \mathbf{u}) + \nabla_2h(\phi(\mathbf{u}), \mathbf{u})\boldsymbol{\lambda}(\mathbf{u})
\]

By recalling that
\[
\mathcal{L}(\mathbf{x}, \mathbf{u}, \boldsymbol{\lambda}) = \ell(\mathbf{x}, \mathbf{u}) + \boldsymbol{\lambda}^Th(\mathbf{x}, \mathbf{u})
\]

we have
\[
\nabla J(\mathbf{u}) = \nabla\phi(\mathbf{u})\nabla_1\mathcal{L}(\phi(\mathbf{u}), \mathbf{u}, \boldsymbol{\lambda}) + \nabla_2\mathcal{L}(\phi(\mathbf{u}), \mathbf{u}, \boldsymbol{\lambda}),
\]

so that choosing $\boldsymbol{\lambda} = \boldsymbol{\lambda}(\mathbf{u})$ such that

\[
\nabla_1\mathcal{L}(\phi(\mathbf{u}), \mathbf{u}, \boldsymbol{\lambda}(\mathbf{u})) = 0
\]

it holds

\[
\nabla J(\mathbf{u}) = \nabla_2\mathcal{L}(\phi(\mathbf{u}), \mathbf{u}, \boldsymbol{\lambda}(\mathbf{u})).
\]


\section{First order necessary condition for optimality}
Let $\mathbf{u}^*$ be a local minimum with $\mathbf{x}^* = \phi(\mathbf{u}^*)$ 
Then
\[
    \nabla J(\mathbf{u^*}) = 0
\]
that is, if there exists a $\boldsymbol{\lambda}^*$ such that 
\[
    \nabla_1\mathcal{L}(\phi(\mathbf{u}^*),\mathbf{u}^*,\boldsymbol{\lambda}^*)=0
\]
it holds 
\[
    \nabla_2\mathcal{L}(\phi(\mathbf{u}^*),\mathbf{u}^*,\boldsymbol{\lambda}^*)=0
\]


\subsection{Explicit computation of \texorpdfstring{$\nabla J(\mathbf{u})$}{Explicit computation of the gradient of J(u)}}

Now we can compute explicitly $\nabla J(\mathbf{u})$. We have:
\[
\mathcal{L}(\mathbf{x}, \mathbf{u}, \boldsymbol{\lambda}) = \underbrace{\sum_{t=0}^{T-1} \ell_t(x_t, u_t) + \ell_T(x_T)}_{\ell(\mathbf{x},\mathbf{u})} + \underbrace{\sum_{t=0}^{T-1} (f_t(x_t, u_t) - x_{t+1})^T\lambda_{t+1}}_{h(\mathbf{x},\mathbf{u})^T \boldsymbol{\lambda}}
\]

By calculating $\nabla_1\mathcal{L}(\phi(\mathbf{u}), \mathbf{u}, \boldsymbol{\lambda}(\mathbf{u})) = 0$ for each $x_t$, we obtain:
\begin{align*}
\lambda_T &= \nabla\ell_T(x_T) \\
\lambda_t &= \nabla_1\ell_t(x_t, u_t) + \nabla_1f_t(x_t, u_t)\lambda_{t+1} \qquad t = T-1,\ldots,1
\end{align*}

then
\[
(\nabla J(\mathbf{u}))_t = \nabla_2\ell_t(x_t, u_t) + \nabla_2f_t(x_t, u_t)\lambda_{t+1} \qquad t = T-1,\ldots,0
\]

\subsubsection{Linear System Formulation}

Consider the equations:
\begin{align*}
\lambda_t &= \nabla_1\ell_t(x_t, u_t) + \nabla_1f_t(x_t, u_t)\lambda_{t+1}, \quad t = T-1,\ldots,1, \quad \text{with } \lambda_T = \nabla\ell_T(x_T),\\
(\nabla J(\mathbf{u}))_t &= \nabla_2\ell_t(x_t, u_t) + \nabla_2f_t(x_t, u_t)\lambda_{t+1} \quad t = T-1,\ldots,0
\end{align*}

Notice that we can write:
\begin{align*}
A_t^T &= \nabla_1f_t(x_t, u_t),\\
B_t^T &= \nabla_2f_t(x_t, u_t)
\end{align*}

where $A_t \in \mathbb{R}^{n\times n}$ and $B_t \in \mathbb{R}^{m\times n}$ are the state and input matrices of the linearization of the system $x_{t+1} = f(x_t, u_t)$ about the trajectory $(\mathbf{x}, \mathbf{u})$.

Thus, we can rewrite:
\begin{align*}
\lambda_t &= A_t^T\lambda_{t+1} + a_t \qquad t = T-1,\ldots,1, \quad \text{with } \lambda_T = \nabla\ell_T(x_T)\\
(\nabla J(\mathbf{u}))_t &= B_t^T\lambda_{t+1} + b_t \qquad t = T-1,\ldots,0
\end{align*}

where we have simply renamed $a_t = \nabla_1\ell_t(x_t, u_t)$ and $b_t = \nabla_2\ell_t(x_t, u_t)$ (with $a_t \in \mathbb{R}^n$ and $b_t \in \mathbb{R}^m$).

\section{Optimality conditions for unconstrained optimal control}

Optimality conditions for optimal control can be written as follows. Let $(\mathbf{x}^*, \mathbf{u}^*)$ be an optimal (minimum) state-input trajectory. Then there exists $\boldsymbol{\lambda}^*$ such that:

\begin{align*}
x_{t+1}^* &= f_t(x_t^*, u_t^*) & t &= 0,\ldots,T-1 \\
\lambda_t^* &= \nabla_1\ell_t(x_t^*, u_t^*) + \nabla_1f_t(x_t^*, u_t^*)\lambda_{t+1}^* & t &= T-1,\ldots,1 \\
0 &= \nabla_2\ell_t(x_t^*, u_t^*) + \nabla_2f_t(x_t^*, u_t^*)\lambda_{t+1}^* & t &= 0,\ldots,T-1
\end{align*}

with $x_0^* = x_{\text{init}}$ and $\lambda_T^* = \nabla\ell_T(x_T^*)$.

To find stationary trajectories, we need to solve this system of equations in $(\mathbf{x}, \mathbf{u}, \boldsymbol{\lambda})$.

\begin{remark}
It is worth noting that the system contains equations that move forward in time (the dynamics) and others that move backward in time (the costate equations). Additionally, there are algebraic equations $((\nabla J(\mathbf{u}))_t = 0)$ that may not have (in general) an explicit solution.
\end{remark}



\subsection{Hamiltonian function and optimality conditions}

It is customary to write these conditions in terms of the Hamiltonian function:
\begin{equation}
H_t(x_t, u_t, \lambda_{t+1}) = \ell_t(x_t, u_t) + \lambda_{t+1}^T f_t(x_t, u_t)
\end{equation}

The first-order necessary conditions of optimality can be written as:
\begin{equation}
\nabla_2 H_t(x_t^*, u_t^*, \lambda_{t+1}^*) = 0 \qquad t = 0,\ldots,T-1,
\end{equation}
and
\begin{equation}
\lambda_t^* = \nabla_1 H_t(x_t^*, u_t^*, \lambda_{t+1}^*) \qquad t = 1,\ldots,T-1,
\end{equation}
with terminal condition
\begin{equation}
\lambda_T^* = \nabla\ell_T(x_T^*).
\end{equation}

\subsubsection{Principle of Optimality with the Hamiltonian}

\begin{proposition}
Let $(\mathbf{x}^*, \mathbf{u}^*)$ be a local (minimum) optimal trajectory. Then
\[
\nabla_2 H_t(x_t^*, u_t^*, \lambda_{t+1}^*) = 0 \qquad t = 0,\ldots,T-1
\]
where the costate vector $\boldsymbol{\lambda}^*$ is obtained from the adjoint equation
\[
\lambda_t^* = \nabla_1 H_t(x_t^*, u_t^*, \lambda_{t+1}^*) \qquad t = 1,\ldots,T-1
\]
with terminal condition
\[
\lambda_T^* = \nabla\ell_T(x_T^*).
\]
\end{proposition}

\begin{remark}
Notice that we do not need to assume any "regularity" condition on the local (minimum) optimal (state-input) trajectory.
\end{remark}

\subsubsection{Remarks on Hamiltonian Function and Optimality Conditions}

It is worth noting that the constraint due to the dynamics, $x_{t+1}^* = f(x_t^*, u_t^*)$, can be written as
\[
x_{t+1}^* = \nabla_3 H_t(x_t^*, u_t^*, \lambda_{t+1}^*).
\]

Thus denoting
\[
\nabla H_t(x_t^*, u_t^*, \lambda_{t+1}^*) = \begin{bmatrix}
\nabla_1 H_t(x_t^*, u_t^*, \lambda_{t+1}^*) \\
\nabla_2 H_t(x_t^*, u_t^*, \lambda_{t+1}^*) \\
\nabla_3 H_t(x_t^*, u_t^*, \lambda_{t+1}^*)
\end{bmatrix}
\]

the first-order necessary conditions of optimality can be compactly written as
\[
\begin{bmatrix}
x_{t+1}^* \\ 0 \\ \lambda_t^*
\end{bmatrix} = \nabla H_t(x_t^*, u_t^*, \lambda_{t+1}^*) \qquad t = 0,\ldots,T-1
\]
with the additional condition $\lambda_T^* = \nabla\ell_T(x_T^*)$.

\section{Optimality with the Hamiltonian in Case of LQP}

\begin{remark}
Very important: notice that rewriting the generic Optimal Control Problem (OCP) in its reduced form and then searching for optimality brings us ("from a different way", in a sense) to the same results we've obtained previously just imposing the KKT conditions to the generic form of the problem; this is coherent and indeed pretty obvious, because we are always talking about the same problem expressed (and then solved) in different BUT totally equivalent ways!
\end{remark}

Indeed, let's consider a LQP a little bit generalized (with the classic assumptions), alias a generic case in which $q_t = 0 \, \forall t$, $r_t = 0 \, \forall t$, $q_T = 0$, $Q_T = 0$ (time-varying case).

In this case, the Hamiltonian function is:
\[
H_t(x_t, u_t, \lambda_{t+1}) = \frac{1}{2}\begin{bmatrix}
x_t \\ u_t
\end{bmatrix}^T \begin{bmatrix}
Q_t & S_t^T \\ S_t & R_t
\end{bmatrix}\begin{bmatrix}
x_t \\ u_t
\end{bmatrix} + (A_tx_t + B_tu_t)\lambda_{t+1}
\]

We have:

\begin{itemize}
    \item $\lambda_t^* = \nabla_1 H_t(x_t, u_t, \lambda_{t+1})\big|_{\substack{x_t=x_t^*\\ u_t=u_t^*\\ \lambda_t=\lambda_t^*}} = Q_tx_t^* + S_t^Tu_t^* + A_t^T\lambda_{t+1}^* \quad \forall t=T-1...1, \lambda_T^* = Q_T$

    \item $0 = \nabla_2 H_t(x_t, u_t, \lambda_{t+1})\big|_{\substack{x_t=x_t^*\\ u_t=u_t^*\\ \lambda_t=\lambda_t^*}} \Leftrightarrow 0 = S_tx_t^* + R_tu_t^* + B_t^T\lambda_{t+1}^* \Rightarrow u_t^* = -R_t^{-1}(S_tx_t^* + B_t^T\lambda_{t+1}^*) \quad \forall t=0...T-1, x_0 \text{ given}$

    \item Exploiting $x_{t+1}^* = \nabla_3 H_t(x_t, u_t, \lambda_{t+1})\big|_{\substack{x_t=x_t^*\\ u_t=u_t^*\\ \lambda_t=\lambda_t^*}}$ we can find $P_t^*$ such that $\lambda_t^* = P_tx_t^*$
\end{itemize}

Then the problem can be easily solved (as already shown previously) finding $u_t^*$ as a function of $x_t$: $u_t^* = K_tx_t^*$ where $K_t = -(R_t + B_t^TP_tB_t)^{-1}(S_t + B_t^TP_tA_t)$.

\section{Shooting Approach: LQ Optimal Control}

\subsubsection{Part I: Cost Function and System Representation}
We can compactly write the cost function as
\[
\ell(\mathbf{x}, \mathbf{u}) = \sum_{t=0}^{T-1} \frac{1}{2}\left[x_t^T Q_tx_t + u_t^T R_tu_t\right] + \frac{1}{2}x_T^TQ_Tx_T = \frac{1}{2}\mathbf{x}^T \mathcal{Q}\mathbf{x} + \frac{1}{2}\mathbf{u}^T \mathcal{R}\mathbf{u} + \frac{1}{2}x_0^TQ_0x_0
\]
where $\mathcal{Q}$ and $\mathcal{R}$ are block diagonal matrices collecting all the $Q_t$, $R_t$ respectively:
\[
\mathcal{Q} = \begin{bmatrix}
Q_1 & 0 & \cdots & 0 \\
0 & Q_2 & \cdots & 0 \\
\vdots & \vdots & \ddots & \vdots \\
0 & 0 & \cdots & Q_T
\end{bmatrix}, \quad
\mathcal{R} = \begin{bmatrix}
R_1 & 0 & \cdots & 0 \\
0 & R_2 & \cdots & 0 \\
\vdots & \vdots & \ddots & \vdots \\
0 & 0 & \cdots & R_T
\end{bmatrix}
\]

It can be shown that the map $\mathbf{x} = \phi(\mathbf{u})$ can be written as
\[
\phi(\mathbf{u}) = X_0 + \Phi\mathbf{u}
\]
where $X_0 \in \mathbb{R}^{nT}$ is associated to the free evolution, while $\Phi \in \mathbb{R}^{nT \times mT}$ is a state-transition matrix.

\begin{remark}
It's always possible to have a form like that ALSO for NL time-variant systems, where the $\phi(\mathbf{u})$ function is an affine one given by the state free evolution and the state forced evolution!
\end{remark}

\subsubsection{LTI System Trajectories: \texorpdfstring{$\mathbf{x} = \phi(\mathbf{u})$}{x = phi(u)}}
For the time invariant case, we can introduce the map $\phi_t(\mathbf{u}) = x_t$ defined as
\[
\phi_t(\mathbf{u}) = \underbrace{A^tx_0}_{\text{free evolution}} + \underbrace{[A^{t-1}B \quad A^{t-2}B \quad \cdots \quad B \quad 0 \quad \cdots \quad 0]}_{\text{forced evolution}}\mathbf{u}
\]

Therefore we have
\[
\mathbf{x} = \phi(\mathbf{u}) = \underbrace{\begin{bmatrix}
A \\ A^2 \\ \vdots \\ A^T
\end{bmatrix}x_0}_{X_0} + \underbrace{\begin{bmatrix}
B & 0 & \cdots & 0 \\
AB & B & \cdots & 0 \\
\vdots & \vdots & \ddots & \vdots \\
A^{T-1}B & A^{T-2}B & \cdots & B
\end{bmatrix}}_{\Phi}\mathbf{u}
\]
which can be written compactly as $\mathbf{x} = X_0 + \Phi\mathbf{u}$

\subsubsection{Part II: Cost Function Reduction}
Therefore, the cost function can be written as
\begin{align*}
\ell(\phi(\mathbf{u}), \mathbf{u}) &= \frac{1}{2}(X_0 + \Phi\mathbf{u})^T \mathcal{Q}(X_0 + \Phi\mathbf{u}) + \frac{1}{2}\mathbf{u}^T \mathcal{R}\mathbf{u} + \frac{1}{2}x_0^T Q_0x_0 \\
&= \frac{1}{2}X_0^T \mathcal{Q}X_0 + \frac{1}{2}\mathbf{u}^T \Phi^T \mathcal{Q}\Phi\mathbf{u} + X_0^T \mathcal{Q}\Phi\mathbf{u} + \frac{1}{2}\mathbf{u}^T \mathcal{R}\mathbf{u} + \frac{1}{2}x_0^T Q_0x_0 \\
&= \frac{1}{2}\mathbf{u}^T (\Phi^T \mathcal{Q}\Phi + \mathcal{R})\mathbf{u} + X_0^T \mathcal{Q}\Phi\mathbf{u} + \underbrace{\frac{1}{2}X_0^T \mathcal{Q}X_0 + \frac{1}{2}x_0^T Q_0x_0}_{\text{constant}}
\end{align*}

Since the constant term does not affect the minimizer, the optimal control problem can be written in its reduced form as
\[
\min_{\mathbf{u}} \frac{1}{2}\mathbf{u}^T (\Phi^T \mathcal{Q}\Phi + \mathcal{R})\mathbf{u} + X_0^T \mathcal{Q}\Phi\mathbf{u}
\]

\begin{remark}
Notice that:
\begin{itemize}
    \item $Q_t = Q_t^T \geq 0 \,\forall t\in[0,T] \Rightarrow \mathcal{Q} \geq 0$
    \item $R_t = R_t^T > 0 \,\forall t\in[0,T] \Rightarrow \mathcal{R} > 0$
\end{itemize}
$\Rightarrow$ the problem has an UNIQUE global solution $\mathbf{u}^* = -\frac{1}{2}(\Phi^T\mathcal{Q}\Phi+\mathcal{R})^{-1}\Phi^T\mathcal{Q}X_0$
\end{remark}

\subsection{Refresh: Optimality Conditions for QP}
Let us consider a quadratic optimization problem
\[
\min_{z\in\mathbb{R}^n} z^TQ z + b^T z
\]
with $Q = Q^T \in \mathbb{R}^{n\times n}$ and $b \in \mathbb{R}^n$.

\begin{remark}[Important]
For a quadratic program necessary conditions of optimality are also sufficient and minima are global.
\end{remark}

If $z^*$ is a minimum then:
\begin{itemize}
    \item First-order necessary condition for optimality: $\nabla\ell(z^*) = 0 \implies 2Q z^* + b = 0$
    \item Second-order necessary condition for optimality: $\nabla^2\ell(z^*) \geq 0 \implies 2Q \geq 0$
\end{itemize}

If $Q_t > 0$, then there exists a unique global minimum given by $z^* = -\frac{1}{2}Q^{-1}b$.

\section{LQ Problem Optimality Conditions}

Let us define the reduced cost
\[
J(\mathbf{u}) := \frac{1}{2}\mathbf{u}^T(\Phi^T\mathcal{Q}\Phi + \mathcal{R})\mathbf{u} + X_0^T\mathcal{Q}\Phi\mathbf{u}
\]

Since $(\Phi^T\mathcal{Q}\Phi + \mathcal{R}) > 0$, then $J(\mathbf{u})$ is a convex, positive-definite quadratic function.

Therefore, the reduced problem can be written as
\[
\min_{\mathbf{u}} J(\mathbf{u})
\]

Being $J(\mathbf{u})$ convex and positive definite, first order necessary conditions are also sufficient.

Then, $\mathbf{u}^*$ is the (unique) optimal solution if and only if
\[
\nabla J(\mathbf{u}^*) = 0
\]



%%%%%%%%%%%%%%%%%%%%%%%%%%%%%%%%%%%%%%%%%%%%%%%%%%%%%%%%%%%%%%%%%%%%%%%%%%%%%%%%%%%%%%%%%%%%%%%%%%%%%%%%%%%%%%%%%%%%%%%%%%%%%%%%%%%%%%%%%%%%%%%%%%%%%%%%%%%%%%%%%%%%%%%%%%%%%%
\chapter{Gradient Method for Optimal Control}

\section{Unconstrained Optimal Control Problem (Discrete-Time)}
We look for a solution of the discrete-time optimal control problem:
\begin{align*}
    \min_{\mathbf{x}\in\R^{nT},\mathbf{u}\in\R^{mT}} & \sum_{t=0}^{T-1}\ell_t(x_t,u_t)+\ell_T(x_T)\\
    \text{subj. to } & x_{t+1} = f_t(x_t,u_t), \quad t\in\{0,\dots,T-1\}
\end{align*}
with given initial condition $x_0 = x_{\text{init}} \in \R^n$, where $\ell_t:\R^n\times\R^m\to\R$ is the so-called stage cost while $\ell_T:\R^n\to\R$ is the terminal cost.

\begin{assumption}
Functions $\ell_t(\cdot,\cdot)$, $\ell_T(\cdot)$, $f_t(\cdot,\cdot)$ are twice continuously differentiable, i.e., they are class $\mathcal{C}^2$.
\end{assumption}

\section{Recall : Shooting Method}
\subsection{Key Idea}
Express the state $x_t$ at each $t\in\{0,\dots,T-1\}$ as a function of the input sequence $\mathbf{u}$ only.

For all $t$ we can introduce a map $\phi_t:\R^{mT}\to\R^n$ such that:
\[
    x_t := \phi_t(\mathbf{u})
\]

Using compact notation:
\[
    \phi(\mathbf{u}) = \col(\phi_1(\mathbf{u}),\dots,\phi_T(\mathbf{u}))
\]
so that
\[
    \mathbf{x} = \phi(\mathbf{u})
\]

\begin{remark}
Observe that, given an input sequence $\mathbf{u}$, the equality constraint $\phi_{t+1}(\mathbf{u}) = f_t(\phi_t(\mathbf{u}),u_t)$ is satisfied by construction.
\end{remark}

\section{Recall : Reduced Optimal Control Problem}
We can define
\begin{align*}
    J(\mathbf{u}) &:= \sum_{t=0}^{T-1} \ell_t(\phi_t(\mathbf{u}), u_t) + \ell_T(\phi_T(\mathbf{u}))\\
    &= \ell(\phi(\mathbf{u}), \mathbf{u})
\end{align*}

Thus, the optimal control problem can be recast into the so-called reduced version given by
\[
    \min_{\mathbf{u}\in\R^{mT}} J(\mathbf{u})
\]
which is an unconstrained problem with cost function $J:\R^{mT}\to\R$ where the optimization variable is only the input sequence $\mathbf{u}\in\R^{mT}$.

\begin{remark}
This is an unconstrained optimization problem in $\mathbf{u}$ with a $\mathcal{C}^2$ cost function. Notice that the cost function $J(\mathbf{u})$ inherits from the original optimal control problem its smoothness properties, but also its nonconvexity.
\end{remark}

\section{Recall : Calculation of \texorpdfstring{$\nabla J(\mathbf{u})$}{Gradient of J(u)}}
By recalling that
\[
    \mathcal{L}(\mathbf{x}, \mathbf{u}, \boldsymbol{\lambda}) = \ell(\mathbf{x}, \mathbf{u}) + \boldsymbol{\lambda}^T h(\mathbf{x}, \mathbf{u})
\]
we have
\[
    \nabla J(\mathbf{u}) = \nabla\phi(\mathbf{u})\nabla_1\mathcal{L}(\phi(\mathbf{u}), \mathbf{u}, \boldsymbol{\lambda}) + \nabla_2\mathcal{L}(\phi(\mathbf{u}), \mathbf{u}, \boldsymbol{\lambda})
\]

Thus, choosing $\boldsymbol{\lambda} = \boldsymbol{\lambda}(\mathbf{u})$ such that
\[
    \nabla_1\mathcal{L}(\phi(\mathbf{u}), \mathbf{u}, \boldsymbol{\lambda}(\mathbf{u})) = \nabla_1\ell(\phi(\mathbf{u}), \mathbf{u}) + \nabla_1h(\phi(\mathbf{u}), \mathbf{u})\boldsymbol{\lambda}(\mathbf{u}) = 0
\]
it holds
\[
    \nabla J(\mathbf{u}) = \nabla_2\mathcal{L}(\phi(\mathbf{u}), \mathbf{u}, \boldsymbol{\lambda}(\mathbf{u})) = \nabla_2\ell(\phi(\mathbf{u}), \mathbf{u}) + \nabla_2h(\phi(\mathbf{u}), \mathbf{u})\boldsymbol{\lambda}(\mathbf{u})
\]

\section{Recall : Gradient Method}
Given an unconstrained problem:
\begin{align*}
    \min_z & \quad \ell(z)\\
    \text{s.t.} & \quad z\in\R^d
\end{align*}

where $\ell:\R^d\to\R$, the gradient method applied to this problem reads:
\[
    z^{k+1} = z^k - \gamma^k\nabla\ell(z^k)
\]
where $\gamma^k > 0$ chosen according to Armijo, constant or diminishing rules.

\section{Gradient Method for Optimal Control}

\subsection{Basic Formulation}
The gradient descent method applied to the reduced optimization problem reads as:
\[
    \mathbf{u}^{k+1} = \mathbf{u}^k - \gamma^k\nabla J(\mathbf{u}^k)
\]

Choosing $\boldsymbol{\lambda}^k$ such that:
\[
    \nabla_1\mathcal{L}(\phi(\mathbf{u}^k), \mathbf{u}^k, \boldsymbol{\lambda}^k) = 0
\]
the gradient descent iteration reduces to:
\[
    \mathbf{u}^{k+1} = \mathbf{u}^k - \gamma^k\nabla_2\mathcal{L}(\phi(\mathbf{u}^k), \mathbf{u}^k, \boldsymbol{\lambda}^k)
\]
or explicitly:
\[
    \mathbf{u}^{k+1} = \mathbf{u}^k - \gamma^k(\nabla_2\ell(\phi(\mathbf{u}^k), \mathbf{u}^k) + \nabla_2h(\phi(\mathbf{u}^k), \mathbf{u}^k)\boldsymbol{\lambda}^k)
\]

\subsection{Forward and Backward Simulations}
Given a trajectory $(x^k, u^k)$ at the $k$-th iteration, we compute:
\begin{enumerate}
    \item \textbf{Backward computation:} Solve $\nabla_1\mathcal{L}(\phi(\mathbf{u}^k), \mathbf{u}^k, \boldsymbol{\lambda}^k)=0$ to find $\boldsymbol{\lambda}^k$, where $\lambda_T^k = \nabla\ell(x_T)$ is fixed and well-known for all $k$
    \item \textbf{Forward computation:} Compute $x^{k+1} = f_t(x^{k+1}, u^{k+1})$ with $x_0^k = x_0^{\text{const}}$ fixed and well-known for all $k$
\end{enumerate}

This process yields a new trajectory $(x^{k+1}, u^{k+1})$ with lower cost compared to $(x^k, u^k)$.

\begin{remark}
The forward simulation constitutes an open-loop integration. Running the system dynamics, i.e., computing $x^{k+1}=f_t(x^{k+1}, u^{k+1})$ (given $u^{k+1}$), means performing an open-loop integration. This may lead to numerical instabilities as small errors can be amplified, particularly if the system itself is unstable. This consideration must be taken into account when selecting the step size.
\end{remark}

\subsection{Comparison of SQP and Gradient Method Approaches}
\begin{itemize}
    \item \textbf{SQP Approach:} While it converges to an input-state feasible trajectory, it does not provide feasible trajectories during iterations
    \item \textbf{Gradient Method:} Provides feasible trajectories at each iteration (though generally suboptimal) that converge to an optimal solution
\end{itemize}

Moving along the trajectory manifold $\mathcal{T}$, the gradient method moves along its tangent space and then projects back onto $\mathcal{T}$ through the open-loop system dynamics.

\section{Recall : Explicit Computation of \texorpdfstring{$\nabla J(\mathbf{u})$}{Gradient}}

Now we can compute explicitly $\nabla J(\mathbf{u})$. We have:
\begin{equation}
    \mathcal{L}(\mathbf{x}, \mathbf{u}, \boldsymbol{\lambda}) = \underbrace{\sum_{t=0}^{T-1}\ell_t(x_t, u_t) + \ell_T(x_T)}_{\ell(\mathbf{x},\mathbf{u})} + \underbrace{\sum_{t=0}^{T-1}(f_t(x_t, u_t) - x_{t+1})^T\lambda_{t+1}}_{h(\mathbf{x},\mathbf{u})^T\boldsymbol{\lambda}}
\end{equation}

Thus, given a trajectory $(\mathbf{x}, \mathbf{u})$, i.e., such that
\[
    x_{t+1} = f_t(x_t, u_t), \quad t = 0,1,\ldots,T-1, \quad x_0 \text{ given}
\]

the gradient at each $t$ is computed as:
\begin{align*}
    \lambda_t &= \nabla_1\ell_t(x_t, u_t) + \nabla_1f(x_t, u_t)\lambda_{t+1} & t &= T-1,\ldots,1, & \lambda_T &= \nabla\ell_T(x_T)\\
    (\nabla J(\mathbf{u}))_t &= \nabla_2\ell_t(x_t, u_t) + \nabla_2f(x_t, u_t)\lambda_{t+1} & t &= T-1,\ldots,0
\end{align*}

This can be rewritten more compactly by renaming:
\begin{align*}
    \lambda_t &= A_t^T\lambda_{t+1} + a_t & t &= T-1,\ldots,1, & \text{with }\lambda_T &= \nabla\ell_T(x_T)\\
    (\nabla J(\mathbf{u}))_t &= B_t^T\lambda_{t+1} + b_t & t &= T-1,\ldots,0
\end{align*}

where we have simply renamed:
\begin{align*}
    A_t^T &= \nabla_1f(x_t, u_t), & B_t^T &= \nabla_2f(x_t, u_t)\\
    a_t &= \nabla_1\ell_t(x_t, u_t), & b_t &= \nabla_2\ell_t(x_t, u_t)
\end{align*}

\newpage

\section{Gradient Method Algorithm}
\begin{algorithm}
\caption{Gradient Method for Optimal Control}

\textbf{Initialization:} Consider an initial guess trajectory $(x^0, u^0)$

\vspace{0.5cm}

\textbf{For} $k = 0,1,\ldots$

\vspace{0.5cm}

\textbf{Step 1:} Compute descent direction
\begin{enumerate}
    \item Set matrices and vectors:
        \begin{align*}
            A_t^{k\top} &= \nabla_1f_t(x_t^k, u_t^k), & B_t^{k\top} &= \nabla_2f_t(x_t^k, u_t^k)\\
            a_t^k &= \nabla_1\ell_t(x_t^k, u_t^k), & b_t^k &= \nabla_2\ell_t(x_t^k, u_t^k)
        \end{align*}
        
    \item Solve backwards the co-state equation:
        \begin{align*}
            \lambda_t^k &= A_t^{k\top}\lambda_{t+1}^k + a_t^k, & t &= T-1,\ldots,1\\
            \lambda_T^k &= \nabla\ell_T(x_T^k)
        \end{align*}
        
    \item Compute descent direction:
        \[
            \Delta u_t^k = -B_t^{k\top}\lambda_{t+1}^k - b_t^k, \quad t = T-1,\ldots,0
        \]
\end{enumerate}

\vspace{0.5cm}

\textbf{Step 2:} Compute new input sequence
\[
    u_t^{k+1} = u_t^k + \gamma^k\Delta u_t^k
\]
where $\gamma^k$ is chosen according to a step-size rule (e.g., Armijo)

\vspace{0.5cm}

\textbf{Step 3:} Compute new state trajectory\\
Forward integrate (open-loop) for $t = 0,\ldots,T-1$, with initial condition $x_0^{k+1} = x_{\text{init}}$:
\[
    x_{t+1}^{k+1} = f(x_t^{k+1}, u_t^{k+1})
\]
\end{algorithm}

\section{Example: Pendulum Control}
Search in the Vince's Note



\chapter{Continuous-time Optimal Control}

\section{Continuous-time Optimal Control Problem}
A continuous-time, i.e., $t \in \mathbb{R}$, optimal control problem can be written as
\begin{gather*}
    \min_{(x(\cdot),u(\cdot))\in \mathcal{F}_x \times \mathcal{F}_u} \int_{0}^{T}\ell_\tau(x(\tau),u(\tau))d\tau+\ell_T(x(T))\\
    \begin{array}{l l }
        \text{subj. to } & \dot{x}(t) = f_t(x(t),u(t)) \quad \forall t \in [0,T]\\
                         & u(t) \in \mathcal{U}\\
                         & x(0) = x^0
    \end{array}
\end{gather*}
where
\begin{itemize}
    \item $x(t) \in \mathbb{R}^n$ state, $u(t) \in \mathbb{R}^m$ input with $x(\cdot):[0,T]\to\mathbb{R}^n$, $u(\cdot):[0,T]\to\mathbb{R}^m$
    \item $\mathcal{U} \subseteq \mathbb{R}^m$ admissible control space
    \item $f_t: \mathbb{R}^n \times \mathbb{R}^m \to \mathbb{R}^n$ continuous-time dynamics
    \item $\ell_t: \mathbb{R}^n \times \mathbb{R}^m \to \mathbb{R}$ stage-cost
    \item $\ell_T: \mathbb{R}^n \to \mathbb{R}$ terminal cost
\end{itemize}

\begin{remark}
$\mathcal{F}_x$, $\mathcal{F}_u$ are function spaces.
\end{remark}

\section{Hamiltonian Function}
Let us introduce the Hamiltonian function $H: \mathbb{R}^n \times \mathbb{R}^m \times \mathbb{R}^n \times \mathbb{R} \to \mathbb{R}$
\[
    H(x(t),u(t),\lambda(t),t) = \ell_t(x(t),u(t)) + \lambda(t)^T f_t(x(t),u(t))
\]
where $\lambda:[0,T] \to \mathbb{R}^n$.

We denote the partial derivatives of $H(\cdot)$ with respect to its arguments evaluated at a certain $t$ as
\begin{align*}
    H_x(x(t),u(t),\lambda(t),t) &:= \frac{\partial}{\partial x}H(x(t),u(t),\lambda(t),t)\\
    H_u(x(t),u(t),\lambda(t),t) &:= \frac{\partial}{\partial u}H(x(t),u(t),\lambda(t),t)\\
    H_\lambda(x(t),u(t),\lambda(t),t) &:= \frac{\partial}{\partial \lambda}H(x(t),u(t),\lambda(t),t)
\end{align*}

\section{Trajectory Manifold}
The trajectory manifold $\mathcal{T}$ is defined as
\[
    \mathcal{T} = \{(x(\cdot),u(\cdot)) \in \mathcal{F}_x \times \mathcal{F}_u | \dot{x}(t) = f_t(x(t),u(t)), t \in [0,T]\}
\]

This represents an infinite dimensional optimization problem in which the decision variable is a function.

\begin{remark}
For points $(x(\cdot),u(\cdot))$ in the trajectory manifold, we have a functional that maps from the space of functions to $\mathbb{R}$, making this an infinite-dimensional optimization problem.
\end{remark}
\section{Additional Necessary Optimality Conditions: Variational Approach}

\subsection{Cost Function Reformulation}
Let $(x^*(\cdot),u^*(\cdot))$ be an optimal trajectory for the optimal control problem with $\mathcal{U} = \mathbb{R}^m$. Since the optimal trajectory fulfills the dynamics and the initial state constraint, we can rewrite the cost as
\[
    J(x^*(\cdot),u^*(\cdot)) = \int_0^T[\ell_\tau(x^*(\tau),u^*(\tau)) + \lambda(\tau)^T(f_t(x^*(\tau),u^*(\tau)) - \dot{x}^*(\tau))]d\tau + \ell_T(x^*(T))
\]
where the multiplier function $\lambda:[0,T]\to\mathbb{R}^n$ is called costate function.

Now, since
\[
    -\lambda(t)^T\dot{x}^*(\tau) = -\frac{d}{dt}\{\lambda(t)^Tx^*(t)\} + \dot{\lambda}(t)^Tx^*(t)
\]
we can rewrite the cost as:
\begin{align*}
    J(x^*(\cdot),u^*(\cdot)) &= \int_0^T[\ell_\tau(x^*(\tau),u^*(\tau)) + \lambda(\tau)^Tf_t(x^*(\tau),u^*(\tau)) + \dot{\lambda}(\tau)^Tx^*(\tau)]d\tau\\
    &\quad -[\lambda(\tau)^Tx^*(\tau)]_0^T + \ell_T(x^*(T))\\
    &= \int_0^T[H(x^*(\tau),u^*(\tau),\lambda(\tau),\tau) + \dot{\lambda}(\tau)^Tx^*(\tau)]d\tau\\
    &\quad -\lambda(T)^Tx^*(T) + \lambda(0)^Tx^*(0) + \ell_T(x^*(T))
\end{align*}

\subsection{Variational Analysis}
Consider an admissible perturbation $(\delta x(\cdot),\delta u(\cdot))$ to the optimal trajectory, with $\delta x(0)=0$ since $x^*(0) + \delta x(0) = x_{\text{init}}$ and
\[
    \delta \dot{x}(t) = \nabla_1f_t(x^*(t),u^*(t))^T\delta x(t) + \nabla_2f_t(x^*(t),u^*(t))^T\delta u(t)
\]

The corresponding cost first variation can be computed as:
\begin{align*}
    \delta J_{(x^*,u^*)}(\delta x(\cdot),\delta u(\cdot)) &= \int_0^T(H_x(x^*(\tau),u^*(\tau),\lambda(\tau),\tau) + \dot{\lambda}(\tau)^T)\delta x(\tau)d\tau\\
    &\quad + \int_0^T H_u(x^*(\tau),u^*(\tau),\lambda(\tau),\tau)\delta u(\tau)d\tau\\
    &\quad + (\nabla\ell_T(x^*(T)) - \lambda(T))^T\delta x(T) + \lambda(0)^T\delta x(0)
\end{align*}

Now, since the state perturbation $\delta x$ depends on the input perturbation $\delta u$, we can express the cost variation as a function of $\delta u$ only by choosing the multiplier $\lambda^*$ as the solution to the following initial value problem:
\begin{align*}
    \dot{\lambda}^*(t) &= -H_x(x^*(t),u^*(t),\lambda^*(t),t)^T, \quad t \in [0,T]\\
    \lambda^*(T) &= \nabla\ell_T(x^*(T))
\end{align*}

For linear systems, this becomes:
\[
    \dot{\lambda}^*(t) = -A^T(t)\lambda^*(t) - q(t)
\]
where
\begin{align*}
    q(t) &= \frac{\partial}{\partial x}\ell_t(x^*(t),u^*(t))\\
    A(t) &= \frac{\partial}{\partial x}f_t(x^*(t),u^*(t))
\end{align*}

Hence, we end up with
\[
    \delta J_{(x^*,u^*,\lambda^*)}(\delta u(\cdot)) = \int_0^T H_u(x^*(\tau),u^*(\tau),\lambda^*(\tau),\tau)\delta u(\tau)d\tau
\]

As in the finite dimensional case, we have that $\delta J_{(x^*,u^*,\lambda^*)}(\delta u(\cdot))$ must be zero for all the possible perturbations $\delta u(\cdot)$, which implies:
\[
    H_u(x^*(t),u^*(t),\lambda^*(t),t) = 0, \quad \forall t \in [0,T]
\]

If this were not the case, we could choose $\delta u(\tau) = -\varepsilon H_u(x^*(\tau),u^*(\tau),\lambda^*(\tau),\tau)^T$ producing a cost variation
\[
    \delta J_{(x^*,u^*,\lambda^*)}(\delta u(\cdot)) = -\varepsilon \int_0^T \|H_u(x^*(\tau),u^*(\tau),\lambda^*(\tau),\tau)\|_2^2 d\tau < 0 \tag{1}
\]
and for sufficiently small $\varepsilon > 0$ this perturbation would decrease the cost, meaning that $(x^*,u^*)$ would not be a (local) minimum.

Hence, under proper differentiability assumptions on the dynamics, the stage cost and the terminal cost, if $(x^*,u^*)$ is an optimal trajectory then there exists a differentiable costate function $\lambda^*$ such that:
\begin{align*}
    \dot{x}^*(t) &= H_\lambda(x^*(\tau),u^*(\tau),\lambda^*(\tau),\tau)^T, \quad x^*(0) = x_{\text{init}}\\
    \dot{\lambda}^*(t) &= -H_x(x^*(\tau),u^*(\tau),\lambda^*(\tau),\tau)^T, \quad \lambda^*(T) = \nabla\ell_T(x^*(t))\\
    0 &= H_u(x^*(\tau),u^*(\tau),\lambda^*(\tau),\tau)
\end{align*}
for all $t \in [0,T]$.

\begin{remark}
These equations are a generalization of the Hamilton equations in mechanics. In fact, consider an unforced mechanical system and let:
\begin{itemize}
    \item $q$ be a set of generalized coordinates
    \item $U(q)$ the potential energy
    \item $T(q,\dot{q})$ the kinetic energy
\end{itemize}
Then let us define the stage cost as the Lagrangian function:
\[
    \ell_t(x,u) = \mathcal{L}(x,u) := T(x,u) - U(x)
\]
where $x = q$ and $u = \dot{q}$ (hence $\dot{x} = f(x,u) = u$).

Then the Hamiltonian function turns out to be $H(q,\dot{q},\lambda) = \mathcal{L}(q,\dot{q}) + \lambda^T\dot{q}$ and the last two optimality equations result in
\begin{align*}
    \dot{\lambda}^*(t) &= -\nabla_1H(q^*(t),\dot{q}^*(t),\lambda^*(t)) = -\nabla_1\mathcal{L}(q^*(t),\dot{q}^*(t))\\
    \nabla_2H(q^*(t),\dot{q}^*(t),\lambda^*(t)) &= \nabla_2\mathcal{L}(q^*(t),\dot{q}^*(t)) + \lambda^*(t) = 0
\end{align*}

By solving for the costate $\lambda$ (which in mechanics is called momentum) in the second and replacing it in the first we obtain:
\[
    \frac{d}{dt}\{\nabla_2\mathcal{L}(q^*(t),\dot{q}^*(t))\} - \nabla_1\mathcal{L}(q^*(t),\dot{q}^*(t)) = 0
\]
which are the Euler-Lagrange equations.
\end{remark}

\section{Pontryagin Minimum Principle}
Let $(x^*(t),u^*(t)), t\in[0,T]$ be an optimal trajectory where $u(t)\in\mathcal{U}$, for all $t$, then 
\begin{enumerate}
    \item There exists a $\lambda^*(t)$ such that, for all $t\in[0,T]$:
        \[
            \dot{\lambda}^*(t) = -H_x(x^*(t),u^*(t),\lambda^*(t),t)^T
        \]
        where $\lambda^*(T)=\nabla\ell_T(x_T^*)$
    \item The Hamiltonian function assumes its minimum in $\mathcal{U}$ when evaluated along $u^*(t)$, i.e. 
        \[
            H(x^*(t),u^*(t),\lambda^*(t),t) \leq H(x^*(t),u(t),\lambda^*(t),t) \qquad \forall u(t)\in \mathcal{U},\ \forall t\in[0,T]
        \]
    \item The optimal trajectory $(x^*(t),u^*(t)), t\in[0,T]$ satisfies, $\forall t \in[0,T]$ 
        \[
            \dot{x}(t) = H_\lambda(x^*(t),u^*(t),\lambda(t)^*,t)^T
        \]
        where $x^*(0) = x_{\text{init}}$
\end{enumerate}
\begin{remark}
    The principle was in fact stated for a maximization problem so that the original name is
Pontryagin Maximum Principle.
\end{remark}


\section{Continuous-time LQ Optimal Control Problem}

Consider a continuous-time linear quadratic optimal control problem:
\begin{align*}
    \min_{(x(\cdot),u(\cdot))\in\mathcal{F}_x\times\mathcal{F}_u} &\int_0^T \left(\frac{1}{2}x(\tau)^TQ(\tau)x(\tau) + \frac{1}{2}u(\tau)^TR(\tau)u(\tau)\right)d\tau + \frac{1}{2}x(T)^TQ_Tx(T)\\
    \text{subj. to } & \dot{x}(t) = A(t)x(t) + B(t)u(t) \quad t \in [0,T]\\
    & x(0) = x^0
\end{align*}

where:
\begin{itemize}
    \item $x(t) \in \mathbb{R}^n$ and $u(T) \in \mathbb{R}^m$
    \item $A(t) \in \mathbb{R}^{n\times n}$
    \item $B(t) \in \mathbb{R}^{n\times m}$
    \item $Q(t) \in \mathbb{R}^{n\times n}$ and $Q(t) = Q(t)^T \geq 0$ for all $t \in [0,T]$
    \item $Q_T \in \mathbb{R}^{n\times n}$ and $Q_T = Q_T^T \geq 0$
    \item $R(t) \in \mathbb{R}^{m\times m}$ and $R(t) = R(t)^T > 0$ for all $t \in [0,T]$
\end{itemize}

\begin{remark}
In classical approach, while $A(t)$ and $B(t)$ may change slowly (gain scheduling), we can design $\lambda(t)$ to be continuous and achieve stability.
\end{remark}

\section{Necessary Conditions for Optimality in CT LQ Problems}

\subsection{Hamiltonian Function}
For the LQ optimal control problem, we define the Hamiltonian function as:
\[
    H(x(t),u(t),\lambda(t),t) = \frac{1}{2}x(t)^TQ(t)x(t) + \frac{1}{2}u(t)^TR(t)u(t) + \lambda(t)^T(A(t)x(t) + B(t)u(t))
\]

\subsection{Pontryagin's Maximum Principle}
For $(x^*(t),u^*(t))$ to be optimal, the following conditions must be satisfied for all $t\in[0,T]$:

\begin{enumerate}
    \item \textbf{Costate equation} (from PMP i):
        \[
            \dot{\lambda}^*(t) = -A(t)^T\lambda^*(t) - Q(t)x^*(t)
        \]
        with terminal condition $\lambda^*(T) = Q_Tx^*(T)$

    \item \textbf{Stationarity condition} (from PMP ii):
        \[
            R(t)u^*(t) + B(t)^T\lambda^*(t) = 0
        \]

    \item \textbf{State equation} (from PMP iii):
        \[
            \dot{x}^*(t) = A(t)x^*(t) + B(t)u^*(t)
        \]
        with initial condition $x^*(0) = x^0$
\end{enumerate}

\subsection{Matrix Form of Optimality Conditions}
Rewriting the necessary conditions for optimality, we have:
\begin{align*}
    \dot{x}^*(t) &= A(t)x^*(t) - B(t)R^{-1}(t)B(t)^T\lambda^*(t)\\
    \dot{\lambda}^*(t) &= -A(t)^T\lambda^*(t) - Q(t)x^*(t)
\end{align*}

This system can be written in matrix form as:
\[
    \begin{bmatrix}
        \dot{x}^*(t) \\ \dot{\lambda}^*(t)
    \end{bmatrix} = 
    \begin{bmatrix}
        A(t) & -B(t)R(t)^{-1}B(t)^T \\
        -Q(t) & -A(t)^T
    \end{bmatrix}
    \begin{bmatrix}
        x^*(t) \\ \lambda^*(t)
    \end{bmatrix}
\]
with boundary conditions $\lambda^*(T) = Q_Tx^*(T)$ and $x^*(0) = x^0$. This constitutes a two-point boundary value problem.

\subsection{Feedback Solution via Riccati Equation}
Let us assume that $\lambda^*(t) = P(t)x^*(t)$. Under this assumption, the optimal feedback input takes the form:
\[
    u^*(t) = K(t)x^*(t)
\]
where
\[
    K(t) = -R(t)^{-1}B(t)^TP(t)
\]

\subsection{Verification of the Assumption}
To verify this assumption, let us expand the costate equation:

\subsubsection{Left-Hand Side}
\begin{align*}
    \dot{\lambda}^*(t) &= \dot{P}(t)x^*(t) + P(t)\dot{x}^*(t)\\
    &= \dot{P}(t)x^*(t) + P(t)A(t)x^*(t) + P(t)B(t)K(t)x^*(t)\\
    &= \dot{P}(t)x^*(t) + P(t)A(t)x^*(t) - P(t)B(t)R(t)^{-1}B(t)^TP(t)x^*(t)
\end{align*}

\subsubsection{Right-Hand Side}
\[
    Q(t)x^*(t) + A(t)^T\lambda^*(t) = Q(t)x^*(t) + A(t)^TP(t)x^*(t)
\]

\subsubsection{Riccati Differential Equation}
Equating both sides and requiring the equality to hold for all $x^*(t)$, we obtain:
\[
    \dot{P}(t) = -P(t)A(t) - A(t)^TP(t) + P(t)B(t)R(t)^{-1}B(t)^TP(t) - Q(t)
\]
with terminal condition $P(T) = Q_T$ to be consistent with $\lambda^*(T) = Q_Tx^*(T)$.

\begin{remark}
The use of this linear mapping between the costate and state variables (seen back in liberation and Hamiltonian mechanics) is a key step for solving the optimal control problem, reducing it to finding the solution of the Riccati differential equation.
\end{remark}

\section{Infinite Horizon Linear Quadratic Optimal Control Problem}

Consider the infinite horizon optimal control problem:
\begin{align*}
    \min_{x(\cdot),u(\cdot)} & \int_0^{\infty} \left(\frac{1}{2}x(\tau)^TQx(\tau) + \frac{1}{2}u(\tau)^TRu(\tau)\right)d\tau\\
    \text{subj. to } & \dot{x}(t) = Ax(t) + Bu(t), \quad t \in [0,\infty)\\
    & x(0) = x_0 \text{ given}
\end{align*}

where:
\begin{itemize}
    \item $A,B,Q,R$ are constant matrices
    \item $Q = Q^T \geq 0$
    \item $R = R^T > 0$
    \item $x(\cdot):[0,\infty) \to \mathbb{R}^n$
    \item $u(\cdot):[0,\infty) \to \mathbb{R}^m$
\end{itemize}

Under controllability of $(A,B)$ and observability of $(A,C)$ , the optimal control law takes the form:
\[
    u^*(t) = K^*x^*(t) \quad \text{with } K^* = -R^{-1}B^TP_\infty
\]

where $P_\infty$ is the solution of the algebraic Riccati equation:
\[
    P_\infty A + A^TP_\infty - P_\infty BR^{-1}B^TP_\infty + Q = 0
\]

The resulting closed-loop system:
\[
    \dot{x}(t) = (A + BK^*)x(t)
\]
is asymptotically stable, i.e., $A + BK^*$ is Hurwitz.

\begin{remark}
The Hurwitz property of $A + BK^*$ follows from the controllability assumption and guarantees that we can choose freely the poles for the feedback system.
\end{remark}


\chapter{Newton's method for Optimal Control}
We look for a solution to the discrete-time optimal control problem 
\begin{align*}
    \min{\substack{\mathbf{x}\in\R^{nT}\\\mathbf{u}\in\R^{mT}}} & \displaystyle\sum_{t=0}^{T-1} \ell_t(x_t,u_t)+\ell_T(x_T)
\end{align*}
with given initial condition $x_0=x_{\text{init}}\in\R^n$, where $\ell_t:\R^n\times\R^m\to\R$ is the so-called stage cost while $\ell_T:\R^n\to\R$ is the terminal cost. In this chapter, we will use the following assumption:\\
\emph{Assumption}: Functions $\ell_t(\cdot,\cdot),\ell_T(\cdot),f_t(\cdot,\cdot)$ are $\mathcal{C}^2$

\section{Newton's method for unconstrained optimization}
Given an unconstrained problem: 
\begin{align*}
    \min_{z} & \ell(z) (=J(u))\\
    \text{subj. to } & z\in\R^d
\end{align*}
where $\ell:\R^d\to\R$, Newton's method applied to this problem reads:
\[
    z^{k+1} = z^k-\gamma^k\nabla^2\ell(z^k)^{-1}\nabla \ell(z^k) = z^k-\gamma^k\Delta z^k
\]
where $\gamma^k>0$ chosen according to Arimjo, constant or diminishing rules. The descent direction $\Delta z^k$ can be calculated as the solution of the QP 
\[
    \Delta z^k = \argmin_{\Delta z}\nabla\ell(z^k)^T\nabla z + \displaystyle\frac{1}{2} \Delta z^T\nabla^2\ell(z^k)\Delta z
\]
Newton's method applied to the reduced optimization problem (see section \ref{shoot}) reads as: 
\[
    \mathbf{u}^{k+1} = \mathbf{u}^k - \gamma^k \Delta\mathbf{u}^k
\]
where 
\[
    \Delta\mathbf{u}^k = \argmin_{\Delta u} \nabla J(\mathbf{u}^k)^T \Delta\mathbf{u} + \displaystyle\frac{1}{2}\Delta \mathbf{u}^T\nabla^2J(\mathbf{u}^k)\Delta\mathbf{u}
\]

\section{Calculation of \texorpdfstring{$\nabla^2J(u)$}{Calculation of the Hessian of J(u)}}

In order to calculate $\nabla^2J(\mathbf{u})$ we move from:
\[
    \nabla J(\mathbf{u}) = \nabla \phi(\mathbf{u}) \nabla_1\mathcal{L}(\phi(\mathbf{u}),\mathbf{u},\boldsymbol{\lambda}) + \nabla_2\mathcal{L}(\phi(\mathbf{u}),\mathbf{u},\boldsymbol{\lambda})
\]
differentiating again this equation we have:
\begin{align*}
    \nabla^2J(\mathbf{u}) = &\nabla^2\phi(\mathbf{u})\nabla_1\mathcal{L}(\phi(\mathbf{u}),\mathbf{u},\boldsymbol{\lambda}) + \nabla\phi(\mathbf{u})\nabla_{11}^2\mathcal{L}(\phi(\mathbf{u}),\mathbf{u},\boldsymbol{\lambda})\nabla \phi(\mathbf{u})^T + \nabla\phi(\mathbf{u})\nabla_{12}^2\mathcal{L}(\phi(\mathbf{u}),\mathbf{u},\boldsymbol{\lambda})\\ &+ \nabla_{22}^2\mathcal{L}(\phi(\mathbf{u}),\mathbf{u},\boldsymbol{\lambda}) + \nabla_{21}^2\mathcal{L}(\phi(\mathbf{u}),\mathbf{u},\boldsymbol{\lambda})\nabla\phi(\mathbf{u})^T\\
     =& \nabla\phi(\mathbf{u})\nabla_{11}^2\mathcal{L}(\phi(\mathbf{u}),\mathbf{u},\boldsymbol{\lambda})\nabla \phi(\mathbf{u})^T + \nabla\phi(\mathbf{u})\nabla_{12}^2\mathcal{L}(\phi(\mathbf{u}),\mathbf{u},\boldsymbol{\lambda})\\ &+ \nabla_{22}^2\mathcal{L}(\phi(\mathbf{u}),\mathbf{u},\boldsymbol{\lambda}) + \nabla_{21}^2\mathcal{L}(\phi(\mathbf{u}),\mathbf{u},\boldsymbol{\lambda})\nabla\phi(\mathbf{u})^T
\end{align*}
since $\boldsymbol{\lambda}=\boldsymbol{\lambda}(\mathbf{u})$ such that $\nabla_1\mathcal{L}(\phi(\mathbf{u}),\mathbf{u},\boldsymbol{\lambda}) =0$

In this case, we also need to take into account $\nabla \phi(\mathbf{u})$. 

\section{Algorithm}
Consider a discrete-time optimal control problem:
\begin{align*}
    \min_{\substack{x_1,\dots,x_T\\u_0,\dots,u_{T-1}}} & \displaystyle\sum_{t=0}^{T-1}\ell_t(x_t,u_t)+\ell_T(x_T)\\
    \text{subj.\ to } & x_{t+1}=f_t(x_t,u_t), \quad t=0,\dots,T-1\\
    & x_0 = x_{\text{init}}
\end{align*}

The Newton's method algorithm for this problem consists of the following steps:

\subsection{Initialization}
Consider an initial guess trajectory $(x^0,u^0)$

\subsection{Main Algorithm}
For each iteration $k=0,1,\dots$:

\subsubsection{Step 1: Descent direction calculation}
\begin{enumerate}
    \item Evaluate $\nabla_1f_t(x_t^k,u_t^k),\nabla_2f_t(x_t^k,u_t^k),\nabla_1\ell_t(x_t^k,u_t^k),\nabla_2\ell_t(x_t^k,u_t^k),\nabla\ell_T(x_T^k)$
    \item Solve backwards the co-state equation, with $\lambda_T^k = \nabla\ell_T(x_T^k)$:
        \[
            \lambda_t^k = \nabla_1f_t(x_t^k,u_t^k)\lambda_{t+1}^k + \nabla_1\ell(x_t^k,u_t^k) \quad t = T-1,\dots,1
        \]
    \item Compute for all $t= 0,\dots,T-1$:
        \begin{gather*}
            Q_t^k := \nabla_{11}^2\ell_t(x_t^k,u_t^k) + \nabla^2_{11}f_t(x_t^k,u_t^k)\cdot \lambda_{t+1}^k \\
            R_t^k := \nabla_{22}^2\ell_t(x_t^k,u_t^k) + \nabla^2_{22}f_t(x_t^k,u_t^k)\cdot \lambda_{t+1}^k\\
            S_t^k := \nabla_{12}^2\ell_t(x_t^k,u_t^k) + \nabla^2_{12}f_t(x_t^k,u_t^k)\cdot \lambda_{t+1}^k
        \end{gather*}
        and $Q_T^k := \nabla^2\ell_T(x_T^k)$
    \item Compute the descent direction by solving:
        \begin{align*}
            \min_{\mathbf{\Delta x, \Delta u}} &\displaystyle\sum_{t=0}^{T-1}\begin{bmatrix}
                \nabla_1\ell_t(x_t^k,u_t^k)\\
                \nabla_2\ell_t(x_t^k,u_t^k)
            \end{bmatrix}^T \begin{bmatrix}
                \Delta x_t \\ \Delta u_t
            \end{bmatrix} + \displaystyle\frac{1}{2}\begin{bmatrix}
                \Delta x_t \\ \Delta u_t
            \end{bmatrix}^T \begin{bmatrix}
                Q_t^k & {S_t^k}^T \\ S_t^k & R_t^k
            \end{bmatrix}\begin{bmatrix}
                \Delta x_t \\ \Delta u_t
            \end{bmatrix}\\
            &+ \nabla\ell_T(x_T^k)^T\Delta x_T + \displaystyle\frac{1}{2}\Delta x_T^TQ_T^k\Delta x_T\\
            \text{subj.\ to } & \Delta x_{t+1} = \nabla_1f_t(x_t^k,u_t^k)^T\Delta x_t + \nabla_2f_t(x_t^k,u_t^k)^T\Delta u_t \quad t=0,\dots,T-1\\
            & \Delta x_0 = 0
        \end{align*}
\end{enumerate}

\subsubsection{Step 2: Input sequence update}
Compute new input sequence (implement step-size selection rule, e.g., Armijo):
\[
    u_t^{k+1} = u_t^k + \gamma^k\Delta u_t^k
\]

\subsubsection{Step 3: State trajectory update}
Forward integrate (open-loop), for all $t=0,\dots,T-1$, with $x_0^{k+1}=x^0$:
\[
    x_{t+1}^{k+1} = f(x_t^{k+1},u_t^{k+1})
\]

\subsection{Example: Pendulum (DT)}
Consider the discrete-time pendulum system:
\[
    \begin{cases}
        x_{t+1,1} = \delta x_{t,1}\\
        x_{t+1,2} = \delta(-g/l\sin x_{t,1}-b/lm x_{t,2}+1/m u_t)
    \end{cases}
\]

This leads to the following gradients:
\begin{gather*}
    \nabla_1f_{t,1}(x_t,u_t) = \begin{bmatrix}
        1 \\ \delta
    \end{bmatrix}, \quad
    \nabla_1f_{t,2}(x_t,u_t) = \begin{bmatrix}
        -\delta g/l\cos x_{t,1} \\ 1-\delta b/lm
    \end{bmatrix}\\
    \nabla_{11}^2f_{t,1}(x_t,u_t) = \begin{bmatrix}
        0 & 0 \\ 0 & 0
    \end{bmatrix}, \quad
    \nabla_{11}^2f_{t,2}(x_t,u_t) = \begin{bmatrix}
        \delta g/l\sin x_{t,1} & 0 \\ 0 & 0
    \end{bmatrix}
\end{gather*}

\section{Newton's Method for Optimal Control - Practical Implementation}

\subsection{Regularization}
In some cases (especially during the first iterations), the matrices $Q_t^k$, $R_t^k$, and $S_t^k$ may not be positive definite. A viable solution is to implement a "first-order" update by solving the affine LQR problem using the following matrices instead:

\begin{align*}
    \tilde{Q}_t^k &:= \nabla_{11}^2\ell_t(x_t^k, u_t^k), & \tilde{R}_t^k &:= \nabla_{22}^2\ell_t(x_t^k, u_t^k),\\
    \tilde{S}_t^k &:= \nabla_{12}^2\ell_t(x_t^k, u_t^k), & \tilde{Q}_T^k &:= \nabla^2\ell_T(x_T^k).
\end{align*}

Specifically, at the first iterations (before switching to the full algorithm version), solve:
\begin{align*}
    \min_{\Delta\mathbf{x},\Delta\mathbf{u}} &\sum_{t=0}^{T-1} \begin{bmatrix}
        \nabla_1\ell_t(x_t^k, u_t^k) \\
        \nabla_2\ell_t(x_t^k, u_t^k)
    \end{bmatrix}^T \begin{bmatrix}
        \Delta x_t \\ \Delta u_t
    \end{bmatrix} + \frac{1}{2}\begin{bmatrix}
        \Delta x_t \\ \Delta u_t
    \end{bmatrix}^T \begin{bmatrix}
        \tilde{Q}_t^k & \tilde{S}_t^{k\top} \\
        \tilde{S}_t^k & \tilde{R}_t^k
    \end{bmatrix}\begin{bmatrix}
        \Delta x_t \\ \Delta u_t
    \end{bmatrix} \\
    &\quad + \nabla\ell_T(x_T^k)^\top \Delta x_T + \frac{1}{2}\Delta x_T^\top\tilde{Q}_T^k\Delta x_T\\
    \text{subject to } & \Delta x_{t+1} = \nabla_1f_t(x_t^k, u_t^k)^\top \Delta x_t + \nabla_2f_t(x_t^k, u_t^k)^\top \Delta u_t,\\
    & \Delta x_0 = 0
\end{align*}

\begin{remark}
This algorithm loses the quadratic convergence property of Newton's method but remains usable and avoids issues with non-positive definite matrices.
\end{remark}

Alternative strategy: use $\tilde{R}_t^k = R_t^k + E_t$, where $E_t$ is a positive definite diagonal matrix.

\subsection{Affine LQR}
Consider an LQR problem with affine cost and affine dynamics:
\begin{align*}
    \min_{\substack{\Delta x_1,\ldots,\Delta x_T\\\Delta u_0,\ldots,\Delta u_{T-1}}} &\sum_{t=0}^{T-1} \begin{bmatrix}
        q_t \\ r_t
    \end{bmatrix}^\top \begin{bmatrix}
        \Delta x_t \\ \Delta u_t
    \end{bmatrix} + \frac{1}{2}\begin{bmatrix}
        \Delta x_t \\ \Delta u_t
    \end{bmatrix}^\top \begin{bmatrix}
        Q_t & S_t^\top \\ S_t & R_t
    \end{bmatrix}\begin{bmatrix}
        \Delta x_t \\ \Delta u_t
    \end{bmatrix} + q_T^\top x_T + \frac{1}{2}\Delta x_T^\top Q_T\Delta x_T\\
    \text{subject to } &\Delta x_{t+1} = A_t\Delta x_t + B_t\Delta u_t + c_t, \quad t = 0,\ldots,T-1
\end{align*}
where:
\begin{itemize}
    \item $Q_t \in \mathbb{R}^{n_x \times n_x}$ and $Q_t = Q_t^\top \geq 0$ for all $t = 0,\ldots,T$
    \item $R_t \in \mathbb{R}^{n_u \times n_u}$ and $R_t = R_t^\top > 0$ for all $t = 0,\ldots,T-1$
    \item $S_t \in \mathbb{R}^{n_u \times n_x}$ such that $Q_t - S_t^\top R_t^{-1}S_t$ is positive semidefinite
\end{itemize}

This problem can be solved by augmenting the state as:
\[
    \Delta \tilde{x}_t := \begin{bmatrix} 1 \\ \Delta x_t \end{bmatrix}
\]

The system matrices can then be rewritten as:
\begin{gather*}
    \tilde{Q}_t := \begin{bmatrix} 0 & q_t^\top \\ q_t & Q_t \end{bmatrix}, \quad
    \tilde{S}_t := \begin{bmatrix} r_t & S_t \end{bmatrix}, \quad
    \tilde{R}_t := R_t\\
    \tilde{A}_t := \begin{bmatrix} 1 & 0 \\ c_t & A_t \end{bmatrix}, \quad
    \tilde{B}_t := \begin{bmatrix} 0 \\ B_t \end{bmatrix}
\end{gather*}

\begin{remark}
The Riccati equation in the augmented state framework can be obtained by imposing $\tilde{P}$ to have the form $\begin{bmatrix} 0 & p_t^\top \\ p_t & P_t \end{bmatrix}$ where $P_t$ is given by the Riccati equation in the original state framework, and $p_t$ is defined to make the $\tilde{P}_t$ expression match a Riccati equation structure.
\end{remark}

\subsection{Optimal Solution of the Affine LQR Problem}

The optimal solution exhibits a feedback structure that can be advantageous for addressing open-loop simulation issues in Newton's method for optimal control. The solution takes the following form for $t = 0,\ldots,T-1$:

\begin{align*}
    \Delta u_t^* &= K_t^*\Delta x_t^* + \sigma_t^* \\
    \Delta x_{t+1}^* &= A_t\Delta x_t^* + B_t\Delta u_t^*
\end{align*}

The feedback gain matrix $K_t^*$ and feedforward term $\sigma_t^*$ are given by:
\begin{align*}
    K_t^* &= -(R_t + B_t^\top P_{t+1}B_t)^{-1}(S_t + B_t^\top P_{t+1}A_t) \\
    \sigma_t^* &= -(R_t + B_t^\top P_{t+1}B_t)^{-1}(r_t + B_t^\top p_{t+1} + B_t^\top P_{t+1}c_t)
\end{align*}

The vectors $p_t$ and matrices $P_t$ are computed recursively backward in time according to:
\begin{align*}
    p_t &= q_t + A_t^\top p_{t+1} + A_t^\top P_{t+1}c_t - {K_t^*}^\top(R_t + B_t^\top P_{t+1}B_t)\sigma_t^* \\
    P_t &= Q_t + A_t^\top P_{t+1}A_t - {K_t^*}^\top(R_t + B_t^\top P_{t+1}B_t)K_t^*
\end{align*}

with terminal conditions:
\begin{align*}
    p_T &= q_T \\
    P_T &= Q_T
\end{align*}

\begin{remark}
The feedback structure inherent in this solution provides a significant advantage when dealing with the open-loop system simulation in Step 3 of Newton's method for optimal control. The feedback nature of the controller helps maintain stability during the integration process.
\end{remark}

\subsection{Robustified step 2 update: closed-loop state-input trajectory update}
Idea: approximate $\Delta x_t \approx x_t^{k+1}-x_t^k$.\\ 
\textbf{Initialization}: consider an initial guess trajectory $(\mathbf{x}^0,\mathbf{u}^0)$\\
For each iter $k$, \\
\textbf{Step 1:} solve the affine LQR
\begin{align*}
    \min_{\mathbf{\Delta x, \Delta u}}  &\displaystyle\sum_{t=0}^{T-1}\begin{bmatrix}
        \nabla_1\ell_t(x_t^k,u_t^k)\\
        \nabla_1\ell_t(x_t^k,u_t^k)
    \end{bmatrix}^T \begin{bmatrix}
        \Delta x_t \\ \Delta u_t
    \end{bmatrix} + \displaystyle\frac{1}{2}\begin{bmatrix}
        \Delta x_t \\ \Delta u_t
    \end{bmatrix}^T \begin{bmatrix}
        Q_t^k & {S_t^k}^T \\ S_t^k & R_t^k
    \end{bmatrix}\begin{bmatrix}
        \Delta x_t \\ \Delta u_t
    \end{bmatrix} 
     + \nabla\ell_T(x_T^k)^T\Delta x_T + \displaystyle\frac{1}{2}\Delta x_T^TQ_T^k\Delta x_T\\
    \text{subj to } & \Delta x_{t+1} = \nabla_1f_t(x_t^k,u_t^k)^T\Delta x_t + \nabla_2f_t(x_t^k,u_t^k)^T\Delta u_t \quad t=0,\dots,T-1 \\ 
    & \Delta x_0 = 0
\end{align*}
and obtain gain matrices $K_t^k$ and feed-forward actions $\sigma_t^k$ for all $t=0,\dots,T-1$\\ 
\textbf{Step 2:} compute new state-input trajectory 

Forward integrate(closed-loop), for all $t=0,\dots,T-1$ with $x_0^{k+1}=x^0$
\begin{align*}
    &u_t^{k+1} = u_t^k + \gamma^k(\sigma_t^k + K_t^k\Delta x_t^k) + K_t^k(x_t^{k+1}-x_t^k - \gamma^k\Delta x_t)\\ 
    & x_{t+1}^{k+1} = f_t(x_t^{k+1},u_t^{k+1})
\end{align*}
which results as 
\begin{align*}
    &u_t^{k+1} = u_t^k + K_t^k(x_t^{k+1}-x_t^k ) + \gamma^k\sigma_t^k  \\ 
    & x_{t+1}^{k+1} = f_t(x_t^{k+1},u_t^{k+1})
\end{align*}



\chapter{Dynamic Programming}
Consider the optimal control problem 
\begin{gather*}
    \min_{\substack{x_0,x_1,\dots,x_T\\u_0,\dots,u_{T-1}}} \displaystyle\sum_{t=0}^{T-1}\ell_t(x_t,u_t)+\ell_T(x_T)\\
    \begin{array}{l l }
        \text{subj. to } & x_{t+1}=f_t(x_t,u_t), \quad t\in\{0,\dots,T-1\}\\
                         & x_0 = x_{\text{init}}
    \end{array}
\end{gather*}
Dynamic programming aims at solving optimal control problems by exploiting Bellman's principle of optimality: Each subtrajectory of an optimal trajectory is an optimal trajectory as well

\begin{figure}[ht]
    \centering
    \includegraphics[width=0.8\textwidth]{bellman.png}
    \caption{Bellman's Principle of Optimality}
    \label{fig:bellman}
\end{figure}


The optimal value function (or \emph{cost-to-go function}) $V_t^*:\R^n \to \R$ for each $t \in \{0,\dots,T-1\}$ is defined as:
\begin{align*}
    V_t^*(\bar{x})=  \min_{\substack{x_{t+1},\dots,x_T\\u_t,\dots,u_{T-1}}} &  \displaystyle\sum_{\tau=t}^{T-1}\ell_\tau(x_\tau,u_\tau)+\ell_T(x_T)\\
                            \text{subj. to } & x_{\tau+1}=f_\tau(x_\tau,u_\tau), \quad \tau = t,\dots,T-1\\
                                             & x_t = \bar{x}
\end{align*}

\begin{remark}
Notice that $V_T^*(\bar{x})=\ell_T(\bar{x})$
\end{remark}

To develop the Bellman equation, let's first write $V_{T-1}^*(\bar{x})$ explicitly:
\begin{align*}
    V_{T-1}^*(\bar{x}) &= \min_{u_{T-1}\in\R^m} \{\ell_{T-1}(x_{T-1},u_{T-1}) + \ell_T(x_T)\} \\
    \text{subj. to } & x_T = f_{T-1}(x_{T-1},u_{T-1}); \quad x_{T-1} = \bar{x}
\end{align*}

Setting $z = u_{T-1}\in\R^m$:
\[
    V_{T-1}^*(\bar{x}) = \min_{z\in\R^m} \{\ell_{T-1}(\bar{x},z) + V_T(f_{T-1}(\bar{x},z))\}
\]

From the definition of optimal value function, we can write:
\begin{align*}
    V_t^*(\bar{x}) &= \min_{\substack{x_{t+1},\dots,x_T \\ u_t,\dots,u_{T-1}}} \{\ell_t(x_t,u_t) + \sum_{\tau=t+1}^{T-1} \ell_\tau(x_\tau,u_\tau) + \ell_T(x_T)\} \\
    \text{subj. to } & x_{t+1} = f_t(x_t,u_t) \\
    & x_{\tau+1} = f_\tau(x_\tau,u_\tau), \quad \forall\tau=t+1,\dots,T-1 \\
    & x_t = \bar{x}
\end{align*}

Similarly for $V_{t+1}^*(\bar{x})$:
\begin{align*}
    V_{t+1}^*(\bar{x}) &= \min_{\substack{x_{t+2},\dots,x_T \\ u_{t+1},\dots,u_{T-1}}} \{\sum_{\tau=t+1}^{T-1} \ell_\tau(x_\tau,u_\tau) + \ell_T(x_T)\} \\
    \text{subj. to } & x_{\tau+1} = f_\tau(x_\tau,u_\tau), \quad \forall\tau=t+1,\dots,T-1 \\
    & x_{t+1} = \bar{x} = f_t(x_t,u_t)
\end{align*}

Merging these results we obtain:

\begin{theorem}[Bellman Equation]
For any $t \in \{0,\dots,T-1\}$ and any $\bar{x} \in \R^n$:
\[
    V_t^*(\bar{x}) = \min_{z\in\R^m} \{\ell_t(\bar{x},z) + V_{t+1}^*(f_t(\bar{x},z))\}
\]
where $z = u_t$.
\end{theorem}


\section{Derivation of the Recursion}
By isolating the first contribution in the cost, we have:
\begin{align*}
    V_t^*(\bar{x}) = \min_{\substack{x_{t+1},\dots,x_T \\ u_t,\dots,u_{T-1}}} & \ell_t(x_t,u_t) + \sum_{\tau=t+1}^{T-1} \ell_\tau(x_\tau,u_\tau) + \ell_T(x_T)\\
    \text{subject to } & x_{\tau+1} = f_t(x_\tau,u_\tau), \quad \tau=t,t+1,\dots,T-1\\
    & x_t = \bar{x}
\end{align*}

Take $\bar{x}_{t+1} = f_t(\bar{x}_t,u_t^*)$ with $u_t^*$ solution of the previous problem we can write:
\begin{align*}
    V_{t+1}^*(f_t(\bar{x}_t,u_t^*)) = \min_{\substack{x_{t+2},\dots,x_T \\ u_{t+1},\dots,u_{T-1}}} & \sum_{\tau=t+1}^{T-1} \ell_\tau(x_\tau,u_\tau) + \ell_T(x_T)\\
    \text{subject to } & x_{\tau+1} = f_t(x_\tau,u_\tau), \quad \tau=t+1,\dots,T-1\\
    & x_{t+1} = \bar{x}_{t+1}
\end{align*}

For $t=0,\dots,T-1$ the optimal value function satisfies:
\[
    V_t^*(\bar{x}) = \min_{u\in\R^m} \{\ell_t(\bar{x},u) + V_{t+1}^*(f_t(\bar{x},u))\}
\]
for any $\bar{x}\in\R^n$. This equation is known as Bellman's Equation.

\begin{remark}
The optimal cost for the original optimal control problem is $V_0^*(x_{\text{init}})$.
\end{remark}

\begin{remark}
If $V_{t+1}^*$ expression is known, this becomes a classical optimal control problem.
Meaning: if we know the optimal cost from time $t+1$ to the end, then we need to optimize only the "previous time stuff"!
\end{remark}

\subsection{Example: Graph Path Optimization}
Consider a directed weighted graph with vertices $\{V_0,V_1,\dots,V_6\}$ where we want to find the optimal path (cost: roads) between $V_0$ and $V_6$. The problem can be divided into three stages:
\begin{itemize}
    \item Stage T-2: vertices $\{V_1,V_2,V_3\}$
    \item Stage T-1: vertices $\{V_4,V_5\}$  
    \item Stage T: vertex $V_6$
\end{itemize}

Solving backwards from T-1:
\begin{align*}
    V_{T-1}^*(V_4) &= \min\{1,2+3\} = 1\\
    V_{T-1}^*(V_5) &= \min\{3,2+2\} = 3
\end{align*}

Therefore at stage T-1, the optimal path from $V_4$ to $V_6$ has cost 1.

For stage T-2:
\[
    V_{T-2}^*(V_3) = \min\{2+V_{T-1}^*(V_4), 5+V_{T-1}^*(V_5)\} = 3
\]

\subsection{Complexity Analysis}
For this example:
\begin{itemize}
    \item Exhaustive search complexity: $O(m^T n)$
    \item Dynamic Programming complexity: $O(nmT)$ for $t=0,\dots,T-1$
\end{itemize}
where:
\begin{itemize}
    \item $m = |U|$ (cardinality of input space)
    \item $n = |X|$ (cardinality of state space)
\end{itemize}

\section{Optimal Control Policy and Trajectory}

\subsection{Policy Definition and Computation}
A policy (feedback control law) $\pi_t(x)$ is a function that associates, at time $t$, to each state $x$ an input $u$, i.e., $\pi_t: \R^n \to \R^m$.

The optimal policy to apply at time $t$ when in a given state $x_t$ can be computed as:
\[
    \pi_t^*(x_t) = \argmin_u \{\ell_t(x_t,u) + V_{t+1}^*(f_t(x_t,u))\}
\]

Given this policy, an optimal trajectory can be computed by forward simulation as:
\begin{align*}
    u_t^* &= \pi_t^*(x_t^*) \\
    x_{t+1}^* &= f_t(x_t^*, u_t^*) \quad t = 0,\ldots,T-1 \\
    x_0^* &= x_{\text{init}}
\end{align*}

\begin{remark}
$u_t^*$ is not just a value in time, but rather a policy - a feedback control law!
\end{remark}

\section{Dynamic Programming: Advantages and Limitations}

\subsection{Advantages}
\begin{itemize}
    \item No need for differentiability or convexity assumptions on $\ell_t(\cdot)$, $\ell_T(\cdot)$, $f_t(\cdot)$
    \item Works well on discrete state-control spaces (where $x$ and $u$ have finite cardinality)
\end{itemize}

\subsection{Disadvantages}
\begin{itemize}
    \item Analytical solution not available on continuous spaces (e.g., $\R^n$) - course of dimensionality
    \item For continuous spaces, we need to parametrize the space with a finite set of parameters
\end{itemize}

\begin{remark}
A special case where DP can be performed exactly is Linear Quadratic optimal control, where we know the $V_{t+1}^*$ expression!
\end{remark}

\section{Linear Quadratic Optimal Control via Dynamic Programming}

Consider a linear quadratic optimal control problem as:
\begin{align*}
\min_{\substack{x_1,\dots,x_T \\ u_0,\dots,u_{T-1}}} & \sum_{t=0}^{T-1} \frac{1}{2}\begin{bmatrix}
x_t \\ u_t
\end{bmatrix}^T \begin{bmatrix}
Q_t & S_t^T \\ S_t & R_t
\end{bmatrix} \begin{bmatrix}
x_t \\ u_t
\end{bmatrix} + \frac{1}{2}x_T^T Q_T x_T\\
\text{subj. to } & x_{t+1} = A_tx_t + B_tu_t \quad t=0,\dots,T-1\\
&x_0 = x_{\text{init}}
\end{align*}
where:
\begin{itemize}
\item $x \in \mathbb{R}^n$ and $u \in \mathbb{R}^m$
\item $A_t \in \mathbb{R}^{n\times n}$
\item $B_t \in \mathbb{R}^{n\times m}$
\item $Q_t \in \mathbb{R}^{n\times n}$ and $Q_t = Q_t^T \geq 0$ for all $t=0,\dots,T$
\item $R_t \in \mathbb{R}^{m\times m}$ and $R_t = R_t^T > 0$ for all $t=0,\dots,T-1$
\item $S_t \in \mathbb{R}^{m\times n}$ such that the problem is convex\footnote{Namely, $Q_t - S_t^T R_t^{-1} S_t$ positive semi-definite.}
\end{itemize}

\subsection{Optimal Value Function}
The optimal value function can be defined as:
\begin{align*}
V_t^*(\bar{x}) = \min_{\substack{x_{t+1},\dots,x_T\\u_t,\dots,u_{T-1}}} \sum_{\tau=t}^{T-1} \frac{1}{2}\begin{bmatrix}
x_\tau \\ u_\tau
\end{bmatrix}^T \begin{bmatrix}
Q_\tau & S_\tau^T \\ S_\tau & R_\tau
\end{bmatrix} \begin{bmatrix}
x_\tau \\ u_\tau
\end{bmatrix} + \frac{1}{2}x_T^T Q_T x_T\\
\text{subj. to } x_{\tau+1} = A_\tau x_\tau + B_\tau u_\tau \quad \tau \in [t,T-1]\\
x_t = \bar{x}
\end{align*}

\noindent\textbf{Guess:} $V_t^*$ is a positive semi-definite quadratic function at any time, i.e., we can write:
\[
V_t^*(\bar{x}) = \frac{1}{2}\bar{x}^T P_t\bar{x}
\]
where $P_t = P_t^T \geq 0$.

At $t=T$, it trivially results that $P_T = Q_T$.

Then, try to compute $P_t$ recursively from $P_{t+1}$ based on the principle of optimality.

\subsection{Dynamic Programming Recursion}
Consider the dynamic programming recursion:
\[
V_t^*(x_t) = \min_{u\in\mathbb{R}^m} \frac{1}{2}\begin{bmatrix}
x_t \\ u
\end{bmatrix}^T \begin{bmatrix}
Q_t & S_t^T \\ S_t & R_t
\end{bmatrix} \begin{bmatrix}
x_t \\ u
\end{bmatrix} + V_{t+1}^*(A_tx_t + B_tu)
\]

The optimal input policy is the minimizer:
\begin{align*}
\pi_t^*(x_t) &= \argmin_{u\in\mathbb{R}^m} \frac{1}{2}\begin{bmatrix}
x_t \\ u
\end{bmatrix}^T \begin{bmatrix}
Q_t & S_t^T \\ S_t & R_t
\end{bmatrix} \begin{bmatrix}
x_t \\ u
\end{bmatrix} + \frac{1}{2}(A_tx_t + B_tu)^T P_{t+1}(A_tx_t + B_tu)\\
&= \argmin_{u\in\mathbb{R}^m} \frac{1}{2}\begin{bmatrix}
x_t \\ u
\end{bmatrix}^T \begin{bmatrix}
Q_t + A_t^T P_{t+1} A_t & S_t^T + A_t^T P_{t+1} B_t\\
S_t + B_t^T P_{t+1} A_t & R_t + B_t^T P_{t+1} B_t
\end{bmatrix} \begin{bmatrix}
x_t \\ u
\end{bmatrix}\\
&= -(R_t + B_t^T P_{t+1} B_t)^{-1}(S_t + B_t^T P_{t+1} A_t)x_t = K_t^*x_t
\end{align*}

Since $\pi_t^*(x_t) = K_t^*x_t$, we can write:
\begin{align*}
V_t^*(x_t) &= \frac{1}{2}\begin{bmatrix}
x_t \\ K_t^*x_t
\end{bmatrix}^T \begin{bmatrix}
Q_t + A_t^T P_{t+1} A_t & S_t^T + A_t^T P_{t+1} B_t\\
S_t + B_t^T P_{t+1} A_t & R_t + B_t^T P_{t+1} B_t
\end{bmatrix} \begin{bmatrix}
x_t \\ K_t^*x_t
\end{bmatrix}\\
\frac{1}{2}x_t^T P_tx_t &= \frac{1}{2}x_t^T(Q_t + A_t^T P_{t+1} A_t - (S_t^T + A_t^T P_{t+1} B_t)(R_t + B_t^T P_{t+1} B_t)^{-1}(S_t^T + B_t^T P_{t+1} A_t))x_t
\end{align*}

This should hold for all $x_t$, therefore we obtain the Difference Riccati Equation:
\[
P_t = Q_t + A_t^T P_{t+1} A_t - (S_t^T + A_t^T P_{t+1} B_t)(R_t + B_t^T P_{t+1} B_t)^{-1}(S_t^T + B_t^T P_{t+1} A_t)
\]

Setting $P_T = Q_T$, it is possible to obtain $P_t$ via backward integration for $t = T-1,\dots,0$.

\subsection{Summary}
Consider a linear quadratic optimal control problem:
\begin{align*}
\min_{\substack{x_1,\dots,x_T \\ u_0,\dots,u_{T-1}}} & \sum_{t=0}^{T-1} \frac{1}{2}\begin{bmatrix}
x_t \\ u_t
\end{bmatrix}^T \begin{bmatrix}
Q_t & S_t^T \\ S_t & R_t
\end{bmatrix} \begin{bmatrix}
x_t \\ u_t
\end{bmatrix} + \frac{1}{2}x_T^T Q_T x_T\\
\text{subj. to } & x_{t+1} = A_tx_t + B_tu_t \quad t=0,\dots,T-1\\
&x_0 = x_{\text{init}}
\end{align*}

The solution procedure is:

\begin{enumerate}
\item Set $P_T = Q_T$ and backward iterate $t = T-1,\dots,0$:
\[
P_t = Q_t + A_t^T P_{t+1} A_t - (S_t^T + A_t^T P_{t+1} B_t)(R_t + B_t^T P_{t+1} B_t)^{-1}(S_t^T + B_t^T P_{t+1} A_t)
\]

\item Define the feedback gain:
\[
K_t^* := -(R_t + B_t^T P_{t+1} B_t)^{-1}(S_t + B_t^T P_{t+1} A_t)
\]

\item Compute the optimal trajectory by forward integration $t = 0,\dots,T-1$:
\begin{align*}
u_t^* &= K_t^*x_t^*\\
x_{t+1}^* &= A_tx_t^* + B_tu_t^*, \quad x_0^* = x_{\text{init}}
\end{align*}
\end{enumerate}

\section{Infinite Horizon Linear Quadratic Problems}

\subsection{Problem Formulation}
Consider a linear quadratic optimal control problem:
\begin{align*}
\min_{\substack{x_1,\dots \\ u_0,\dots}} &\sum_{t=0}^{\infty} \frac{1}{2}\begin{bmatrix}
x_t \\ u_t
\end{bmatrix}^T \begin{bmatrix}
Q & S^T \\ S & R
\end{bmatrix} \begin{bmatrix}
x_t \\ u_t
\end{bmatrix}\\
\text{subj. to } & x_{t+1} = Ax_t + Bu_t \quad t=0,\dots,\infty\\
&x_0 = x_{\text{init}}
\end{align*}
where:
\begin{itemize}
\item $x \in \mathbb{R}^n$ and $u \in \mathbb{R}^m$
\item $A \in \mathbb{R}^{n\times n}$
\item $B \in \mathbb{R}^{n\times m}$
\item $Q \in \mathbb{R}^{n\times n}$ and $Q = Q^T \geq 0$  
\item $R \in \mathbb{R}^{m\times m}$ and $R = R^T > 0$
\item $S \in \mathbb{R}^{m\times n}$ such that the problem is convex\footnote{Namely, $Q - S^T R^{-1} S$ semi-positive definite.}
\end{itemize}

\textbf{Assumption:} The pair $(A,B)$ is controllable and the pair $(A,C)$, with $Q=C^TC$, is observable.

\subsection{Optimal Value Function}
The optimal value function can be defined as:
\begin{align*}
V_t^*(\bar{x}) = \min_{\substack{x_{t+1},\dots\\u_t,\dots}} &\sum_{\tau=t}^{\infty} \frac{1}{2}\begin{bmatrix}
x_\tau \\ u_\tau
\end{bmatrix}^T \begin{bmatrix}
Q & S^T \\ S & R
\end{bmatrix} \begin{bmatrix}
x_\tau \\ u_\tau
\end{bmatrix}\\
\text{subj. to } & x_{\tau+1} = Ax_\tau + Bu_\tau \quad \tau \in [t,\infty)\\
&x_t = \bar{x}
\end{align*}

This does not depend on time $t$ (the horizon is always $\infty$), i.e.,
\[
V_{t_1}^*(\bar{x}) = V_{t_2}^*(\bar{x}) \quad \forall t_1 \neq t_2, \forall \bar{x}
\]
Therefore, infinite horizon LQ is shift invariant and we can drop the subscript $t$:
\[
V^*(\bar{x}) := V_t^*(\bar{x})
\]

\textbf{Guess:} $V^*$ is a positive semi-definite quadratic function:
\[
V^*(\bar{x}) = \frac{1}{2}\bar{x}^T P\bar{x}
\]
where $P = P^T \geq 0$.

\subsection{Dynamic Programming Recursion}
Consider the dynamic programming recursion:
\[
V^*(x_t) = \min_u \frac{1}{2}\begin{bmatrix}
x_t \\ u
\end{bmatrix}^T \begin{bmatrix}
Q & S^T \\ S & R
\end{bmatrix} \begin{bmatrix}
x_t \\ u
\end{bmatrix} + V^*(Ax_t + Bu)
\]

The optimal input policy is:
\begin{align*}
\pi^*(x_t) &= \argmin_u \frac{1}{2}\begin{bmatrix}
x_t \\ u
\end{bmatrix}^T \begin{bmatrix}
Q & S^T \\ S & R
\end{bmatrix} \begin{bmatrix}
x_t \\ u
\end{bmatrix} + \frac{1}{2}(Ax_t + Bu)^T P(Ax_t + Bu)\\
&= -(R + B^T PB)^{-1}(S + B^T PA)x_t = K^*x_t
\end{align*}

Since $\pi^*(x_t) = K^*x_t$, with $K^* := -(R + B^T PB)^{-1}(S + B^T PA)$, we can write:
\begin{align*}
V^*(x_t) &= \frac{1}{2}\begin{bmatrix}
x_t \\ K^*x_t
\end{bmatrix}^T \begin{bmatrix}
Q + A^T PA & S^T + A^T PB\\
S + B^T PA & R + B^T PB
\end{bmatrix} \begin{bmatrix}
x_t \\ K^*x_t
\end{bmatrix}\\
\frac{1}{2}x_t^T Px_t &= \frac{1}{2}x_t^T(Q + A^T PA - (S^T + A^T PB)(R + B^T PB)^{-1}(S + B^T PA))x_t
\end{align*}

This should hold for all $x_t$, yielding the Algebraic Riccati Equation:
\[
P = Q + A^T PA - (S^T + A^T PB)(R + B^T PB)^{-1}(S + B^T PA)
\]

%%%%%%%%%%%%%%%%%%%%%%%%%%%%%%%%%%%%%%%%%%%%%%%%%%%%%%%%%%%%%%%%%%%%%%%%%%%%%%%%%%%%%%%%%%%%%%%%%%%%%%%%%%%%%%%%%%%%%%%%%%%%%%%%%%%%%%%%%%%%%%%%%%%%%%%%%%%%%%%%%%%%%%%%%%%%%%
\chapter{Model Predictive Control}
\section{Introductionn}
\subsubsection{Motivations}
We want to control a system 
\[
    x_{t+1} = f_t(x_t,u_t)
\]
via a \emph{stabilizing} controller, which
\begin{itemize}
    \item minimizes a certain cost function \[
            \displaystyle\sum_{t=0}^{\infty}\ell_t(x_t,u_t)
        \]
    \item enforces some constraints for all $t$ 
        \[
            x_t \in \mathcal{X}, u_t \in \mathcal{U}
        \]
    \item works \emph{online}
\end{itemize}
Idea: at each sampling time $t$ solve an optimal control problem and apply the first optimal input.
\subsubsection{Idea} % (fold)
For each $t$
\begin{enumerate}
    \item Measure the current state $x_t$ 
    \item Compute the optimal trajectory $x^*_{t|t},\dots,x^*_{t+T|t},u^*_{t|t},\dots,u^*_{t+T-1|t}$\footnote{$x^*_{\tau|t}$ signifies the optimal state trajectory at instant $\tau$ for the optimal control problem starting at instant $t$, similarly for $u^*_{\tau|t}$} with initial state $x_t$
    \item Apply the first control input $u^*_{t|t}$
    \item Measure $x_{t+1}$ and repeat
\end{enumerate}
\subsubsection{prediction horizon vs time horizon}
Two time-scales: 
\begin{itemize}
    \item time $t=0,\dots,\infty$ time instants in the real world 
    \item prediction iteration $\tau=t,\dots,t+T$, samples evaulated by the mpc algorithm at each time instant $t$
\end{itemize}
\subsubsection{optimal control problem to be solved at each t}
At each time instant $t$, solve 
\begin{align*}
    \min_{\substack{x_0,x_1,\dots,x_T\\u_0,\dots,u_{T-1}}} & \displaystyle\sum_{\tau=t}^{t+T-1}\ell_t(x_\tau,u_\tau)+\ell_{t+T}(x_{t+T})\\
        \text{subj.\ to } & x_{\tau+1}=f(x_\tau,u_\tau) \quad \tau=t,\dots,t+T-1\\
                         & x_\tau \in \mathcal{X}, u_\tau \in \mathcal{U} \qquad \tau=t,\dots,t+T\\
                         & x_t = x_t^{\text{meas}}\text{\footnotemark}
\end{align*}
\footnotetext{$x_t^{\text{meas}}$ is the measured state at time isntant $t$}

\section{MPC with Zero Terminal Constraint} 
At each time instant $t$, solve
\begin{align*}
    \min_{\substack{x_0,x_1,\dots,x_T\\u_0,\dots,u_{T-1}}} & \displaystyle\sum_{\tau=t}^{t+T-1}\ell_t(x_\tau,u_\tau) \\
        \text{subj.\ to } & x_{\tau+1}=f(x_\tau,u_\tau) \quad \tau=t,\dots,t+T-1\\
                          & x_\tau \in \mathcal{X}, u_\tau \in \mathcal{U} \qquad \tau=t,\dots,t+T-1\\ 
                          & x_t = x_t^{\text{meas}} \\
                          & x_{t+T}=0
\end{align*}
where 
\begin{itemize}
    \item $x_{\tau}$ and $u_\tau$ state and input predictions at future time $\tau$ computed at current time $t$ 
    \item $x_t^{\text{meas}}$ state (of the real system) measured at $t$ 
    \item $x=0$ equilibrium point for the system we want to stabilize 
    \item $\mathcal{X}$ and $\mathcal{U}$ state and input constraint sets, which satisfy $(0,0)\in \text{int}\{\mathcal{X}\times \mathcal{U}\}$.
    \item $\ell_\tau:\R^n\times \R^m \times \mathbb{Z}^+\to \R$ is a positive definite continous stage cost $\forall\tau\in\mathbb{Z}^+$.
\end{itemize}
Remark: in the more general case in which $(x^{eq},u^{eq})\neq (0,0)$, we can always perform a global change of coordinates $\psi:(x,u)\to(\bar{x},\bar{u})$ such that $(x^{eq},u^{eq})\to(0,0)$ which brings us back to the previous case 
\begin{theorem}[]
    Consider the discrete-time system
    \[
        x_{t+1}=f(x_t,u_t)
    \]
    with $f:\R^n\times\R^m\to\R^n$ Lipschitz continuous wrt $x$. If the stage cost is continuous, bounded and positive definite for all $\tau$, then the Zero Terminal Constraint MPC scheme is recursively feasible and the origin is asymptotically stable for the resulting closed-loop system.
    Assumptions: the optimal control problem at $t=0$ is feasible and some regularity on the constraint funcitinos ($g_t(x_t,u_t)\leq0$)
\end{theorem}
\begin{proof}[Proof (sketch of)]
    \begin{itemize}
        \item Recursive Feasibility
            \begin{itemize}
                \item Assume the problem is feasible at generic time $t$ and $\{u^*_{\tau|t}\}^{t+T-1}_{\tau=t}$ is the corresponding optimal input sequence, and assume $x_t^{\text{meas}}=x^*_{t|t},\dots,x^*_{t+T|t}$
                \item At time $t+1$ consider the following candidate input trajectory: 
                    \[
                        u_{\tau|t+1}:=\left\{\begin{array}{l l}
                                u_{\tau|t}^*, & \tau = t+1,\dots,t+T-1\\
                                0, & \tau = t+T
                        \end{array} \right.
                    \]
                \item As it can be easily verified, this new trajectory is still feasible for the MPC problem (though in general suboptimal).
            \end{itemize}
        \item Asympototic Stability 
            \begin{itemize}
                \item The idea is to use the optimal cost $J^*(x_t^{\text{meas}})$ as a Lyapunov function 
                \item First note that $J^*(x)\geq 0, \forall x\in\mathcal{X}$ and $J^*(x)=0 \iff x=0$
                    \item Then observe that 
                        \begin{align*}
                            J^*(x_{t+1}^{\text{meas}}) & =\displaystyle\sum_{\tau=t+1}^{t+T}\ell_{\tau}(x_{\tau|t+1}^*,u_{\tau|t+1})\\
                                                       &\leq \displaystyle\sum_{\tau=t+1}^{t+T-1}\ell_{\tau}(x_{\tau|t}^*,u_{\tau|t}^*)+\ell_{t+T}(0,0)\\
                                                       & =J^*(x_t^{\text{meas}})-\ell_t(x_{\tau|t}^*,u^*_{\tau|t})=J^*(x_t^{\text{meas}})-\ell_t(x_t^{\text{meas}},u_t^{\text{MPC}})
                        \end{align*}
                        Thus
                        \[
                            J^*(x_{t+1}^{\text{meas}})-J^*(x_t^{\text{meas}})<0, \quad \forall x_t^{\text{meas}}\neq 0
                        \]
            \end{itemize}
    \end{itemize}
\end{proof}

\section{Quasi-Infinte Horizon MPC}
Main idea: Since the terminal constraint introduces nuerical instabilities, relax the terminal condition by introducing a proper terminal cost $\ell_{t+T}$ and constraining the terminal state to be in a proper region $\mathcal{X}^f$ around the origin
\begin{align*}
    \min_{\substack{x_0,x_1,\dots,x_T\\u_0,\dots,u_{T-1}}} & \displaystyle\sum_{\tau=t}^{t+T-1}\ell_t(x_\tau,u_\tau)+\ell_{t+T}(x_{t+T})\\
        \text{subj.\ to } & x_{\tau+1}=f(x_\tau,u_\tau) \quad \tau=t,\dots,t+T-1\\
                          & x_\tau \in \mathcal{X}, u_\tau \in \mathcal{U} \qquad \tau=t,\dots,t+T-1\\ 
                          & x_t = x_t^{\text{meas}} \\
                          & x_{t+T}\in\mathcal{X}^f
\end{align*}
Assumption: The terminal region is forward invariant wrt a local controller, i.e. there exists $k^{\text{loc}}:\R^n\to\R^m$ such that 
\[
    x\in\mathcal{X}^f \implies f(x,k^{\text{loc}}(x))\in\mathcal{X}^f
\]
Result: Under this assumption it can be proven that for a proper terminal cost and terminal set $\mathcal{X}^f$ the MPC scheme is recursively feasible and asymptotically stable.
\subsection{Practical framework}
\begin{itemize}
    \item Quadratic stage cost and terminal cost: 
        \begin{align*}
            &\ell_\tau(x,u)=x^TQx+u^TRu\\
            &\ell_{t+T}(x) = x^TPx
        \end{align*}
        where $P$ is computed in a way that $x^TPx$ is an upper bound for the optimal infinite-horizon cost-to-go
    \item Ellipsoidal terminal region: 
        \[
            \mathcal{X}^f = \{x\in\R^n|x^TPx\leq\alpha\}
        \]
        for a proper $\alpha$ which is forward invariant wrt the LQR controller $K^{\text{lqr}}$
    \item Hence, at each sampling time $t$, we solve the following problem: 
        \begin{align*}
            \min_{\substack{x_0,x_1,\dots,x_T\\u_0,\dots,u_{T-1}}} & \displaystyle\sum_{\tau=t}^{t+T-1}x_\tau^TQx_\tau + u_\tau^TRu_\tau + x_{t+T}^TPx_{t+T}\\
            \text{subj.\ to } & x_{\tau+1}=f(x_\tau,u_\tau) \quad \tau=t,\dots,t+T-1\\
            & x_t = x_t^{\text{meas}} \\
            & u_\tau \in \mathcal{U} \qquad \tau=t,\dots,t+T-1\\ 
            & x_{t+T}^TPx_{t+T}\leq \alpha
        \end{align*}
        and we apply to the real system the first optimal input $u^*_{t|t}(x_t^{\text{meas}})$
\end{itemize}

\section{Optimal control with constraints}
when dealing with constrained optimal control, with constraints $g(x_t,u_t)\leq 0$ barrier functions can be used: 
\begin{align*}
    \min_{\substack{x_1,\dots,x_T\\u_0,...,u_{T-1}}} & \displaystyle\sum_{t=0}^{T-1}[\ell_t(x_t,u_t)-\varepsilon\log(-g(x_t,u_t))]+\ell_T(x_T)\\
    \text{subj to } &x_{t+1}=f_t(x_t,u_t) \quad t=0,\dots,T-1
\end{align*}
start with $\varepsilon=\varepsilon^0$, solve the relaxed problem above, decrease $\varepsilon>0$\\
Remarks: \begin{itemize}
    \item need to initialize with feasible initial trajectory 
        \item alternatively extend the $-\log(\cdot)$ function on the entire domain, i.e. use some function \[
            b(z)= \begin{cases}
                -\log(z) & z\leq -\delta\\
                \text{smooth function} & z\geq -\delta
            \end{cases} \qquad \delta>0
        \]
        \item use some heuristics to remain within the constraints when
\end{itemize}
% TODO ask someone what's missing


\chapter{Reinforcement Learning}
\section{Notation}
\begin{itemize}
    \item States and actions:
        \[
            (s,a)\in\mathcal{S}\times\mathcal{A}
        \]
        \item Transition Probability:
            \[
                s^+ \sim p(\cdot|s,a)
            \]
        \item Reward function at time $t$:
            \[
                R_t(s_t,a_t) \text{ or } R_t(s_{t-1},a_{t-1},s_t)
            \]
            
            \item Discounted Return: \begin{align*}
               &G_t(s_t):=\displaystyle\sum_{\tau=t}^{\infty}\gamma^\tau R_\tau(s_\tau,a_\tau) \\
               &\text{subj to } s_{\tau+1}\sim p(\cdot|s_\tau,a_\tau),a_\tau\sim \pi(\cdot|s_\tau),\forall \tau \geq 0
            \end{align*}
                
\end{itemize}
Reinforcement learning is generally data-driven, whereas optimal control is generally model-based.
\subsubsection{problem formulation}
The Reinforcement Learning (RL) problem can be written in optimal control language as 
% TODO slide 2






%%%%%%%%%%%%%%%%%%%%%%%%%%%%%%%%%%%%%%%%%%%%%%%%%%%%%%%%%%%%%%%%%%%%%%%%%%%%%%%%%%%%%%%%%%%%%%%%%%%%%%%%%%%%%%%%%%%%%%%%%%%%%%%%%%%%%%%%%%%%%%%%%%%%%%%%%%%%%%%%%%%%%%
%%%%%%%%%%%%%%%%%%%%%%%%%%%%%%%%%%%% CHECK %%%%%%%%%%%%%%%%%%%%%%%%%%%%%%%%%%%%%%%%%%%%%%%%%%%%%%%%%%%%%%%%%%%%%%%%%%%%%%%%%%%%%%%%%%%%%%%%%%%%%%%%%%%%%%%%%%%%%%%%%%%
%%%%%%%%%%%%%%%%%%%%%%%%%%%%%%%%%%%%%%%%%%%%%%%%%%%%%%%%%%%%%%%%%%%%%%%%%%%%%%%%%%%%%%%%%%%%%%%%%%%%%%%%%%%%%%%%%%%%%%%%%%%%%%%%%%%%%%%%%%%%%%%%%%%%%%%%%%%%%%%%%%%%%%

% \chapter{Optimal Control based trajectory generation and tracking}
% Task request: We want to control a (discrete-time) nonlinear system 
% \[
%     x_{t+1}=f_t(x_t,u_t)
% \]
% along a (possibly aggressive) evolution to perform a task while satisfying some performance criteria.
% \\Possible performance criteria:
% \begin{itemize}
%     \item reduce energy consumption
%     \item avoid excessive accelerations (due to e.g., a fragile payload)
% \end{itemize}
% \section{Main strategy idea over a finite horizon} 
% First, a trajectory generation task is reformulated into an optimal control problem such as 
% \begin{align*}
%     \min \sum_{t=0}^{T-1} \displaystyle\frac{1}{2} \| x_t-x_t^{des}\|_{Q_t}^2+ \displaystyle\frac{1}{2}\|u_t-u_t^{des}\|^2_{R_t}+\displaystyle\frac{1}{2}\|x_T-x_T^{des}\|^2_{P_f} \\
%     \text{s.t.} x_{t+1} = f(x_t,u_t) \quad t=0,\dots,T-1\\
%     x_0 = x_{\text{init}}
% \end{align*}
% Where $Q_t,R_t,P_f$ are suitably chosen cost matrices and $(\mathbf{x}^{des},\mathbf{u}^{des})$ is a "reference curve" describing a desired evolution. 

% Note: $(\mathbf{x}^{des},\mathbf{u}^{des})$ is NOT a trajectory. It is based, e.g., on geometric considerations

% Idea: by using an optimal control algorithm, compute an open loop (optimal) state-input trajectory $(\mathbf{x}^{opt},\mathbf{u}^{opt})$, i.e., such that $x_{t+1}^{opt}=f(x_t^{opt},u_t^{opt}),  t=0,\dots,T-1$. Then, a feedback controller can be used to track the system trajectory $(\mathbf{x}^{opt},\mathbf{u}^{opt})$.


% \section{LQR based trajectory tracking}
% Idea: track the generated (optimal) trajectory via a (stabilizing) feedback Linear Quadratic Regulator (LQR) on the linearization.

% \subsubsection{Step 1 - linearize the system}
% Linearize the dynamics about the (feasible) trajectory $(\mathbf{x}^{opt},\mathbf{u}^{opt})$, get the linear (time-varying) system 
% \[
%     \Delta x_{t+1}=A_t^{opt}\Delta x_t+B_t^{opt}\Delta u_t
% \]
% where $A_t^{opt}\in \R^{n\times n}$ and $B_t^{opt}\in\R^{n\times m}$ are defined as: 
% \begin{gather}
%     A_t^{opt}:=\nabla_1f_t{(x_t^{opt},u_t^{opt})}^T \\
%     B_t^{opt}:=\nabla_2f_t{(x_t^{opt},u_t^{opt})}^T
% \end{gather}
% for all $(x_t^{opt},u_t^{opt})$ with $t=0,\dots,T$, state-input pairs at time $t$ of trajectory $(\mathbf{x}^{opt},\mathbf{u}^{opt})$ with length $T$.

% \subsubsection{Step 2 - calculate the LQ optimal controller}
% Solve the optimal control problem 
% \begin{align*}
%     \min_{\substack{\Delta x_1,\dots \Delta, x_T \\ \Delta u_0u,\dots,\Delta u_{T-1}}} & \displaystyle\sum_{t=0}^{\infty}\Delta x_t^TQ_t^{\text{reg}}\Delta x_t+\Delta u_t^TR_t^{\text{reg}}\Delta u_t + \Delta x_T^TQ_T^{\text{reg}}\Delta x_T \\
%         \text{subj. to } & \Delta x_{t+1} = A_t^{opt} \Delta x_t + B_t^{opt}\Delta u_t \quad t=0,1,\dots\\
%                         &\Delta x_0 = 0
% \end{align*}
% for some cost matrices $Q_t^{reg}\geq 0 \in \R^{n\times R},\ R_t^{reg}>0 \in \R^{n\times m}$ and $Q_T^{reg}\geq 0 \in \R^{n\times n}$ (DoF). \\
% Set $P_T=Q_T^{reg}$ and backward iterate $t=T-1,\dots,0$: 
% \[
%     P_t=Q_t^{reg}+{A_t^{opt}}^T P_{t+1}A_t^{opt}-({A_t^{opt}}^T P_{t+1}B_t^{opt})(R_t^{reg}+{B_t^{opt}}^TP_{t+1}B_t^{opt})^{-1}({B_t^{opt}}^TP_{t+1}A_t^{opt})
% \]
% and define for all $t=0,\dots,T-1$, the feedback gain $K_t^{reg}\in\R^{m\times n}$
% \[
%     K_t^{reg} := -(R_t^{reg}+{B_t^{opt}}^TP_{t+1}B_t^{opt})^{-1}({B_t^{opt}}^TP_{t+1}A_t^{opt})
% \]

% \subsubsection{Step 3 - track the generated (optimal) trajectory}
% Apply the feedback controller designed on the linearization to the nonlinear system to track   $(\mathbf{x}^{opt},\mathbf{u}^{opt})$. Namely, for all $t=0,\dots,T-1$, we apply 
% \begin{align*}
%     u_t &= u_t^{opt}+K_t^{reg}(x_t-x_t^{opt})\\ 
%     x_{t+1} &= f_t(x_t,u_t)
% \end{align*}
% with $x_0$ given

% Remark: Under suitable assumptions, it can be shown that an infinite horizon trajectory of a nonlinear system, $(x_t,u_t)$ with $t=0,\dots$ is (locally) exponentially stable if and only if the system linearization about the trajectory is exponentially stable. (this can be viewed as a time-varying version of the Lyapunov indirect theorem)

% % TODO slide 6

% \section{Affine LQR for trajectory tracking}
% The general trajectory tracking problem for a linear system can be recast into an affine LQR problem, with the affine part being generated by the trajectory.
% % idk get notes from someone for math or figure it out







\end{document}
