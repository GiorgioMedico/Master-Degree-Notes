\documentclass[openany]{book}

% Basic packages
\usepackage{amsmath, amsthm, graphicx, amsfonts, float, bm}
\usepackage[english]{babel}
\usepackage[utf8]{inputenc}
\usepackage[T1]{fontenc}
\usepackage{amssymb}
\usepackage{enumitem}
\usepackage{listings}
\usepackage{xcolor}
\usepackage[most]{tcolorbox}

% Page geometry
\usepackage{geometry}
\geometry{
 a4paper,
 total={170mm,237mm},
 left=20mm,
 top=30mm,
}

% Hyperlinks and headers
\usepackage[hidelinks]{hyperref}
\usepackage{fancyhdr}
\usepackage{tikz}
\pagestyle{fancy}
\fancyhf{}
\renewcommand{\headrulewidth}{0pt}
\fancyfoot[C]{\thepage}
\setlength{\footskip}{50pt}

% Image path
\graphicspath{ {./images/} }

% Custom commands and operators
\newcommand\at[2]{\left.#1\right|_{#2}}
\DeclareMathOperator{\sgn}{sgn}
\DeclareMathOperator{\col}{col}
\DeclareMathOperator{\des}{des}
\DeclareMathOperator*{\argmax}{arg\,max}
\DeclareMathOperator*{\argmin}{arg\,min}
\newcommand{\notimplies}{%
\mathrel{{\ooalign{\hidewidth$\not\phantom{=}$\hidewidth\cr$\implies$}}}}
\newcommand{\R}{\mathbb{R}}
\newcommand{\N}{\mathbb{N}}
\newcommand{\deriv}[1]{\displaystyle\frac{d}{d #1}}
\newcommand{\traj}{(\bar{\mathbf{x}},\bar{\mathbf{u}})}

% Theorem environments
\theoremstyle{definition}
\newtheorem{definition}{Definition}[section]
\newtheorem{theorem}{Theorem}[section]
\newtheorem{proposition}{Proposition}[section]

\theoremstyle{remark}
\newtheorem*{remark}{Remark}
\newtheorem*{notation}{Notation}
\newtheorem*{corollary}{Corollary}

% Custom boxes for definitions and notes
\newcommand{\definitionbox}[1]{
\begin{tcolorbox}[colback=blue!5,colframe=blue!40!black,title=Definition]
 #1
\end{tcolorbox}
}
\newcommand{\note}[1]{
\begin{tcolorbox}[colback=green!5,colframe=green!40!black,title=Note]
 #1
\end{tcolorbox}
}

\title{Autonomous and Mobile Robotics}
\author{Giorgio Medico}
\date{fall semester 2024}


\begin{document}
\maketitle
\tableofcontents

\chapter{Control of Mobile Robots}

\section{Configuration Space}
\begin{definition}
The configuration space of a mobile robot has dimensions equal to the number of parameters needed to uniquely describe the configuration of the robot. It is heavily dependent on the structure of the considered robot and is equivalent to the Joint Space for manipulators.
\end{definition}

For different robot types, we have:
\begin{itemize}
    \item Unicycle: $q = [x \; y \; \theta]^T \in \R^2 \times S$
    \item Bicycle: $q = [x \; y \; \theta \; \gamma]^T \in \R^2 \times S^2$
\end{itemize}

\section{Constraints}

\subsection{Fundamental Definitions}

\begin{definition}[Constraint]
A constraint is any condition imposed on a material system that prevents it from assuming a generic position and/or act of motion.
\end{definition}

\begin{definition}[Holonomic Constraint]
A material system is subject to holonomic constraints if finite relations between the coordinates of the system are present (position constraints) or if differentiable/integrable relations between the coordinates of the system are present.
\end{definition}

\begin{definition}[Non-holonomic Constraint]
A constraint is said to be non-holonomic if the differential relation between the coordinates is not reducible to finite form.
\end{definition}

\subsection{Non-Holonomic Constraints}

Under the simplifying assumption that each wheel rolls without slipping, we can state that:
\begin{itemize}
    \item Each wheel introduces a non-holonomic constraint since it does not allow normal translations to the rolling direction
    \item The wheel constraints the instant robot mobility, without typically reducing the configuration space (e.g., parallel parking)
\end{itemize}

Without constraints, the system is described by:
\begin{equation}
    \begin{cases}
    \dot{x} = v_t \cos\theta + v_n \cos(\theta + \frac{\pi}{2}) \\
    \dot{y} = v_t \sin\theta + v_n \sin(\theta + \frac{\pi}{2})
    \end{cases}
\end{equation}

Since there is no slipping in normal direction ($v_n = 0$):
\begin{equation}
    \begin{cases}
    \dot{x} = v_t \cos\theta \\
    \dot{y} = v_t \sin\theta
    \end{cases}
    \Leftrightarrow \tan\theta = \frac{\dot{y}}{\dot{x}}
    \Leftrightarrow [\dot{x}\sin\theta - \dot{y}\cos\theta = 0]
\end{equation}

This last equation represents the mobility constraint.

\section{Pfaffian Form of Constraints}

The constraints can be expressed in different forms:
\begin{itemize}
    \item Constraint vector equation: $a(q)\dot{q} = 0$ (1 wheel)
    \item Constraints matrix equation: $A(q)\dot{q} = 0$ ($N$ wheels)
\end{itemize}

A constraint that can be written as $A(q)\dot{q} = 0$ is said to be in Pfaffian form.

\begin{definition}[Non-Holonomic Constraint Properties]
A non-holonomic constraint:
\begin{itemize}
    \item Cannot be fully integrated
    \item Cannot be written in the configuration space
    \item Does not restrict the space of configurations but the instant robot mobility
\end{itemize}
\end{definition}

\subsection{Admissible Speeds}
Admissible speeds may be generated by a matrix $G(q)$ such that:
\begin{equation}
    \text{Im}(G(q)) = \text{Ker}(A(q)), \quad \forall q \in C
\end{equation}
where $C$ is the Configuration Space.

\section{Kinematic Model of a WMR}

\subsection{General Formulation}
The general kinematic model can be written as:
\begin{equation}
    \dot{q} = G(q)v
\end{equation}

This formulation:
\begin{itemize}
    \item Represents the allowable directions of motion in the configuration space
    \item Binds speeds in the operational space with speeds in the configuration space
    \item Is required to deal with common problems in mobile robotics:
    \begin{itemize}
        \item Trajectory planning
        \item Control (High level)
        \item Robot localization
    \end{itemize}
\end{itemize}

\section{Unicycle Model}

\begin{definition}
A unicycle is a vehicle with a single adjustable wheel with configuration described by $q = [x \; y \; \theta]^T$
\end{definition}

The constraint equation is:
\begin{equation}
    \dot{x}\sin\theta - \dot{y}\cos\theta = 0
\end{equation}

In Pfaffian Form:
\begin{equation}
    A(q)\dot{q} = 0 \text{ with } A(q) = [\sin\theta, -\cos\theta, 0]
\end{equation}

The kernel of $A(q)$ is:
\begin{equation}
    \text{Ker}(A(q)) = \text{span}\left\{\begin{bmatrix}\cos\theta \\ \sin\theta \\ 0\end{bmatrix}, \begin{bmatrix}0 \\ 0 \\ 1\end{bmatrix}\right\} = \text{Im}(G(q))
\end{equation}

\subsection{Kinematic Model}
The unicycle kinematic model is:
\begin{equation}
    \dot{q} = \begin{bmatrix}\cos\theta \\ \sin\theta \\ 0\end{bmatrix}v + \begin{bmatrix}0 \\ 0 \\ 1\end{bmatrix}\omega = \begin{bmatrix}\cos\theta & 0 \\ \sin\theta & 0 \\ 0 & 1\end{bmatrix}\begin{bmatrix}v \\ \omega\end{bmatrix}
\end{equation}

Where:
\begin{itemize}
    \item $v$: linear velocity of the contact point between wheel and ground
    \item $\omega$: angular velocity of the robot
\end{itemize}

\section{Control Architecture}

\subsection{Control Scheme Components}
The control architecture consists of:

\begin{itemize}
    \item \textbf{Actuators:} DC motors, stepper motors
    \item \textbf{End-Effector:} General purpose tool, gripper, hand
    \item \textbf{Sensors:}
    \begin{itemize}
        \item Proprioceptive: encoder, gyro
        \item Exteroceptive: bumpers, rangefinders (infrared, ultrasonic), laser, vision (mono, stereo)
    \end{itemize}
    \item \textbf{Control:}
    \begin{itemize}
        \item Low-level control
        \item High-level control
    \end{itemize}
\end{itemize}

\subsection{Low-Level Control}
Key characteristics:
\begin{itemize}
    \item Uses high-gain PI controllers to control robot motors
    \item Deals only with robot actuators control according to high-level control instructions
    \item Makes the robot a purely kinematic system when gains are high enough
\end{itemize}

\subsection{High-Level Control}
Main features:
\begin{itemize}
    \item Processes and computes signals for low-level controller using sensor data
    \item Views robot as a purely kinematic system
    \item Uses speed control signals for mobile robots
\end{itemize}

\section{Motion Control}

\subsection{Problem Statement}
Given a trajectory or desired configuration, design a control law that leads the robot to:
\begin{itemize}
    \item Reach the desired configuration
    \item Follow the trajectory (high-level control)
\end{itemize}

Key aspects:
\begin{itemize}
    \item Uses kinematic model for motion control
    \item Assumes kinematic inputs act directly on configuration variables
    \item For unicycle and bicycle, control inputs are $v$ and $\omega$
\end{itemize}

\subsection{Control Problems}

\subsubsection{Configuration Regulation}
The robot must reach a desired configuration:
\begin{equation}
    q_d = [x_d, y_d, \theta_d]^T
\end{equation}
starting from an initial configuration:
\begin{equation}
    q_0 = [x_0, y_0, \theta_0]^T
\end{equation}

\subsubsection{Trajectory Following and Tracking}
The robot must asymptotically follow a desired Cartesian trajectory $[x_d(t), y_d(t)]^T$ from initial configuration $q_0 = [x_0, y_0, \theta_0]^T$ with:
\begin{itemize}
    \item \textbf{Trajectory Following:} depends on parameter $s$ (geometric specifications)
    \item \textbf{Trajectory Tracking:} depends on time $t$ (timing specifications)
\end{itemize}

\section{Trajectory Planning}

\subsection{Space-time Separation}
For a trajectory $q(t), t \in [t_i, t_f]$ from initial configuration $q(t_i) = q_i$ to final configuration $q(t_f) = q_f$, we can decompose into:
\begin{itemize}
    \item A path $q(s)$, with $\frac{dq(s)}{ds} \neq 0, \forall s$
    \item A motion law $s = s(t)$, with $s_i \leq s \leq s_f$
\end{itemize}

Where:
\begin{equation}
    \begin{cases}
    s(t_i) = s_i \\
    s(t_f) = s_f
    \end{cases}
\end{equation}

And $s$ is monotonic: $\dot{s}(t) \geq 0$

\subsection{Differential Flatness Planning}
\begin{definition}
A nonlinear dynamic system $\dot{x} = f(x) + g(x)u$ has the property of differential flatness if there exists a set of outputs $y$ (called flat) such that the system's state $x$ and input $u$ can be expressed algebraically as functions of $y$ and its derivatives:
\begin{equation}
    \begin{cases}
    x = x(y, \dot{y}, \ddot{y}, ..., y^{(r)}) \\
    u = u(y, \dot{y}, \ddot{y}, ..., y^{(r)})
    \end{cases}
\end{equation}
\end{definition}

For the unicycle model, the Cartesian coordinates are flat outputs, and:
\begin{equation}
    \begin{cases}
    x' = \tilde{v}\cos\theta \\
    y' = \tilde{v}\sin\theta \\
    \theta' = \tilde{\omega}
    \end{cases}
\end{equation}

The orientation is given by:
\begin{equation}
    \theta = \theta(x', y') = \arctan(y'/x') + k\pi, \quad k = 0,1
\end{equation}


\end{document}
