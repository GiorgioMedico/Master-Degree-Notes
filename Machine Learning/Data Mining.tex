\chapter{Introduzione alla Business Intelligence}

La Business Intelligence rappresenta un insieme completo di metodologie, processi e tecnologie finalizzate alla trasformazione dei dati aziendali in informazioni utili per il processo decisionale. 

\section{Definizioni Fondamentali}
\begin{description}
\item[Definizione Gartner:] ``La Business Intelligence è un termine ombrello che include applicazioni, infrastrutture, strumenti e best practice che abilitano l'accesso e l'analisi delle informazioni per migliorare decisioni e performance.''
\item[Definizione Forrester:] ``La Business Intelligence è un insieme di metodologie, processi, architetture e tecnologie che trasformano i dati grezzi in informazioni significative e utili.''
\end{description}

\section{Obiettivi Principali}
Gli obiettivi fondamentali della BI sono:
\begin{itemize}
\item Trasformare i dati grezzi in informazioni utili
\item Supportare strategie aziendali consapevoli
\item Fornire le informazioni giuste alle persone giuste
\item Garantire la tempestività dell'informazione
\item Ottimizzare i canali di distribuzione delle informazioni
\end{itemize}

\chapter{Il Data Warehouse}

\section{Caratteristiche Fondamentali}
Un Data Warehouse è una collezione di dati che supporta i processi decisionali e presenta le seguenti caratteristiche distintive:

\begin{description}
\item[Orientamento ai soggetti] Focus su concetti specifici dell'impresa come clienti, prodotti, vendite
\item[Integrazione] Unificazione di dati provenienti da fonti eterogenee
\item[Non volatilità] I dati, una volta inseriti, non vengono modificati
\item[Variazione temporale] Mantenimento della storicità dei dati
\end{description}

\section{Vantaggi dei Sistemi DWH}
I principali vantaggi offerti da un sistema DWH sono:
\begin{itemize}
\item Gestione efficiente dei dati storici
\item Esecuzione di analisi multidimensionali
\item Semplicità di apprendimento per gli utenti
\item Supporto al calcolo di indicatori prestazionali
\end{itemize}

\chapter{La Piramide della BI}
La struttura della Business Intelligence può essere rappresentata come una piramide con quattro livelli:

\begin{description}
\item[Dati] Database operazionali e fonti dati
\item[Informazione] OLAP e data warehouse
\item[Conoscenza] Data Mining e modelli di apprendimento
\item[Decisioni] Analisi what-if e modelli di simulazione
\end{description}

\chapter{Campi di Applicazione}

I sistemi DWH trovano applicazione in numerosi settori:

\section{Commercio}
\begin{itemize}
\item Analisi delle vendite e dei reclami
\item Controllo spedizioni e inventario
\item Customer care e relazioni pubbliche
\end{itemize}

\section{Servizi Finanziari}
\begin{itemize}
\item Analisi del rischio
\item Gestione carte di credito
\item Rilevamento frodi
\end{itemize}

\section{Altri Settori}
\begin{itemize}
\item \textbf{Trasporti:} Gestione veicoli
\item \textbf{Telecomunicazioni:} Analisi profilo clienti e performance di rete
\item \textbf{Sanità:} Analisi ammissioni e dimissioni pazienti
\end{itemize}

\chapter{Infrastruttura BI}

Un'efficace piattaforma BI richiede:

\begin{itemize}
\item Hardware dedicato
\item Infrastruttura di rete
\item Database ottimizzati
\item Data Warehouse
\item Software di front-end per la visualizzazione
\end{itemize}