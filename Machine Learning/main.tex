
% Document class for a book with a4paper and openany options
\documentclass[openany]{book}

% Packages for mathematical symbols, theorems, graphics, fonts, and floating objects
\usepackage{amsmath, amsthm, graphicx, amsfonts, float, bm}

% Package for language-specific rules (English)
\usepackage[english]{babel}

% Path for images
\graphicspath{ {./images/} }

% Package for setting page geometry
\usepackage{geometry}
\geometry{
 a4paper, % Paper size
 total={170mm,237mm}, % Total usable area
 left=20mm, % Left margin
 top=30mm, % Top margin
}

% Package for hyperlinks without colored borders
\usepackage[hidelinks]{hyperref}

% Package for custom headers and footers
\usepackage{fancyhdr}

\usepackage{tikz}

% Set page style to fancy
\pagestyle{fancy}

% Clear all header and footer fields
\fancyhf{}

% Remove header rule
\renewcommand{\headrulewidth}{0pt}

% Center the page number in the footer
\fancyfoot[C]{\thepage}

% Increase the distance between the footer and the text
\setlength{\footskip}{50pt}

% Custom commands and operators
\newcommand\at[2]{\left.#1\right|_{#2}} % Evaluation of a function at a point
\DeclareMathOperator{\sgn}{sgn} % Sign function
\DeclareMathOperator{\col}{col} % Column operator
\DeclareMathOperator{\des}{des} % Descent operator
\DeclareMathOperator*{\argmax}{arg\,max} % Argument of the maximum
\DeclareMathOperator*{\argmin}{arg\,min} % Argument of the minimum
\newcommand{\notimplies}{%
    \mathrel{{\ooalign{\hidewidth$\not\phantom{=}$\hidewidth\cr$\implies$}}}} % Not implies symbol
\newcommand{\R}{\mathbb{R}} % Real numbers symbol
\newcommand{\N}{\mathbb{N}} % Natural numbers symbol
\newcommand{\deriv}[1]{\displaystyle\frac{d}{d #1}} % Derivative operator
\newcommand{\traj}{(\bar{\mathbf{x}},\bar{\mathbf{u}})} % Trajectory notation

% Theorem styles and definitions
\theoremstyle{definition}
\newtheorem{definition}{Definition}[section] % Definition environment
\newtheorem{theorem}{Theorem}[section] % Theorem environment
\newtheorem{proposition}{Proposition}[section] % Proposition environment

% Remark styles
\theoremstyle{remark}
\newtheorem*{remark}{Remark} % Remark environment
\newtheorem*{notation}{Notation} % Notation environment
\newtheorem*{corollary}{Corollary} % Corollary environment

\title{Machine Learning and Data Mining}
\author{Giorgio Medico and Carlo Gervasi}
\date{fall semester 2024}


\begin{document}
\maketitle
\tableofcontents

\part{Part 1 : Data Mining}
\chapter{Introduction to Business Intelligence and Data Warehouse}

\section{Basic Concepts and Definitions}
\definitionbox{
Business Intelligence (BI) is a set of methodologies, processes, architectures, and technologies that:
\begin{itemize}
    \item Transform raw data into useful information
    \item Support effective business strategies
    \item Deliver the right information to the right people, at the right time, through the right channel
\end{itemize}
}

\subsection{Authoritative Definitions}
According to Gartner: "Business intelligence is an umbrella term that includes the applications, infrastructure and tools, and best practices that enable access to and analysis of information to improve and optimize decisions and performance."

According to Forrester Research: "Business Intelligence is a set of methodologies, processes, architectures, and technologies that transform raw data into meaningful and useful information used to enable more effective strategic, tactical, and operational insights and decision-making."

\subsection{Data Warehouse Characteristics}
\definitionbox{
The Data Warehouse (DWH) serves as an optimized repository for decision-making and represents a specific type of Decision Support System (DSS).
}

Key features:
\begin{itemize}
    \item Subject-oriented: 
        \begin{itemize}
            \item Focuses on enterprise-specific concepts
            \item Examples: customers, products, sales
        \end{itemize}
    \item Integrated and consistent:
        \begin{itemize}
            \item Integrates data from different and heterogeneous sources
            \item Provides unified view of all data
        \end{itemize}
    \item Time variant and non-volatile:
        \begin{itemize}
            \item Tracks and records changes over time
            \item Data is static and read-only once committed
            \item Retained for future reporting
        \end{itemize}
\end{itemize}

\section{BI Infrastructure and Architecture}

\subsection{The BI Pyramid}
Business Intelligence is structured according to a pyramid representing different levels of data processing:

\begin{center}
\begin{tabular}{|c|l|}
    \hline
    \textbf{Level} & \textbf{Components} \\
    \hline
    DECISIONS & What-if Analysis and simulation modeling \\
    \hline
    KNOWLEDGE & Data Mining and learning models \\
    \hline
    INFORMATION & OLAP and data warehouse \\
    \hline
    DATA & Operational databases and data sources \\
    \hline
\end{tabular}
\end{center}

\subsection{Platform Requirements}
An effective BI platform requires:
\begin{itemize}
    \item Ad-hoc Hardware
    \item Network infrastructure
    \item Databases
    \item Data Warehouse
    \item Front-end software (Data Visualization)
\end{itemize}

\section{Application Fields}

\subsection{Commercial Applications}
\begin{itemize}
    \item Trade:
        \begin{itemize}
            \item Sales and claims analyses
            \item Shipment and inventory control
            \item Customer care and public relations
        \end{itemize}
    \item Financial Services:
        \begin{itemize}
            \item Risk analysis
            \item Fraud detection
            \item Credit card management
        \end{itemize}
\end{itemize}

\subsection{Service Applications}
\begin{itemize}
    \item Transport: Vehicle management
    \item Telecommunications:
        \begin{itemize}
            \item Customer profile analysis
            \item Network performance analysis
        \end{itemize}
    \item Healthcare: Patient admission and discharge analysis
\end{itemize}

\chapter{Data Processing and OLAP}

\section{OLAP Fundamentals}
\definitionbox{
Online Analytical Processing (OLAP) allows users to interactively navigate data warehouse information by exploiting the multidimensional model. Data is analyzed at different levels of aggregation through subsequent OLAP operators.
}

\subsection{Typical Queries}
Common OLAP analytical questions include:
\begin{itemize}
    \item Which products maximize profit?
    \item What is the total revenue per product category and state?
    \item What is the relationship between profits gained by different products?
    \item What is the revenue trend in the last three years?
\end{itemize}

\subsection{OLTP vs OLAP Comparison}
\begin{center}
\begin{tabular}{|p{3.5cm}|p{5cm}|p{5cm}|}
    \hline
    \textbf{Feature} & \textbf{OLTP} & \textbf{OLAP} \\
    \hline
    System Type & Transaction processing & Analytical processing \\
    \hline
    Data Scope & Few records per transaction & Millions of records per query \\
    \hline
    Processing Pattern & Fixed transactions & Dynamic analyses \\
    \hline
    Data Access & Read and write & Mainly read-only \\
    \hline
    Update Frequency & Continuous & Periodic \\
    \hline
    Optimization & For transaction speed & For query performance \\
    \hline
\end{tabular}
\end{center}

\section{Multidimensional Analysis}

\subsection{Data Cube Example}
Here's an example of how data is organized in a multidimensional view:

\begin{center}
\begin{tabular}{|l|l|l|r|r|r|r|}
    \hline
    \textbf{Category} & \textbf{Type} & \textbf{Product} & \multicolumn{2}{c|}{\textbf{2015}} & \multicolumn{2}{c|}{\textbf{2014}} \\
    \cline{4-7}
    & & & \textbf{Jan} & \textbf{Feb} & \textbf{Jan} & \textbf{Feb} \\
    \hline
    Food and & Dairy & White milk & 90 & 90 & 60 & 80 \\
    Beverages & products & Chocolate milk & 60 & 80 & 70 & 70 \\
    & & Yogurt XY & 20 & 30 & 30 & 35 \\
    \cline{2-7}
    & Beverages & Cola & 20 & 10 & 35 & 30 \\
    & & Orange Juice X & 50 & 60 & 60 & 45 \\
    \hline
\end{tabular}
\end{center}

\subsection{OLAP Operators}
\begin{itemize}
    \item \textbf{Roll-up:} 
        \begin{itemize}
            \item Increases data aggregation
            \item Removes detail level from hierarchy
            \item Example: Product → Type → Category
        \end{itemize}
    \item \textbf{Drill-down:}
        \begin{itemize}
            \item Reduces data aggregation
            \item Adds detail level to hierarchy
            \item Example: Category → Type → Product
        \end{itemize}
    \item \textbf{Slice-and-dice:}
        \begin{itemize}
            \item Slice: Sets one dimension to specific value
            \item Dice: Reduces data set by selection criteria
        \end{itemize}
    \item \textbf{Pivot:}
        \begin{itemize}
            \item Changes data presentation layout
            \item Reorganizes multidimensional view
        \end{itemize}
    \item \textbf{Drill-across:}
        \begin{itemize}
            \item Links concepts between related cubes
            \item Enables cross-cube analysis
        \end{itemize}
    \item \textbf{Drill-through:}
        \begin{itemize}
            \item Accesses detailed operational data
            \item Connects aggregates to source data
        \end{itemize}
\end{itemize}
\section{Data Mart}
\definitionbox{
A Data Mart is a subset or an aggregation of data stored in a primary data warehouse, focused on a specific business area, corporate department, or category of users.
}

\subsection{Key Characteristics}
\begin{itemize}
    \item Serves as building block for incremental DWH development
    \item Focuses on specific business area or user group
    \item Provides better query performance due to smaller size
    \item Enables departmental data management
\end{itemize}

\note{
Data Marts can be dependent (sourced from enterprise DWH) or independent (sourced directly from operational systems).
}
\chapter{ETL Process}

\section{Overview and Components}
\definitionbox{
The ETL (Extraction, Transformation, Loading) process extracts, integrates, and cleans data from operational sources to feed the Data Warehouse layer. This process ensures data quality and consistency in the DWH.
}

\section{Extraction Phase}
\subsection{Extraction Methods}
\begin{itemize}
    \item \textbf{Static Extraction:}
        \begin{itemize}
            \item Initial DWH population
            \item Complete snapshot of operational data
            \item Used for first-time setup
        \end{itemize}
    \item \textbf{Incremental Extraction:}
        \begin{itemize}
            \item Regular DWH updates
            \item Captures changes since last extraction
            \item Based on timestamps or triggers
        \end{itemize}
\end{itemize}

\subsection{Data Source Types}
\begin{itemize}
    \item \textbf{Structured Data:}
        \begin{itemize}
            \item Relational databases
            \item Fixed format files
            \item Well-defined schemas
        \end{itemize}
    \item \textbf{Unstructured Data:}
        \begin{itemize}
            \item No pre-defined data model
            \item Variable formats
            \item Requires more complex processing
        \end{itemize}
\end{itemize}

\section{Data Cleansing}
\subsection{Common Quality Issues}
\begin{enumerate}
    \item \textbf{Data Duplication:}
        \begin{itemize}
            \item Multiple customer registrations
            \item Redundant records
            \item Inconsistent representations
        \end{itemize}
    \item \textbf{Missing or Incomplete Data:}
        \begin{itemize}
            \item Null values
            \item Partial information
            \item Empty required fields
        \end{itemize}
    \item \textbf{Format Issues:}
        \begin{itemize}
            \item Inconsistent date formats
            \item Mixed number representations
            \item Character encoding problems
        \end{itemize}
    \item \textbf{Value Problems:}
        \begin{itemize}
            \item Invalid entries (e.g., "30th Feb 2016")
            \item Out-of-range values
            \item Logical inconsistencies
        \end{itemize}
\end{enumerate}
\subsection{Cleansing Solutions}
\begin{itemize}
    \item \textbf{Dictionary-based Techniques:}
        \begin{itemize}
            \item Lookup tables for standardization
            \item Abbreviation resolution
            \item Format normalization
        \end{itemize}
    \item \textbf{Approximate Matching:}
        \begin{itemize}
            \item Similarity functions
            \item Fuzzy matching
            \item Pattern recognition
        \end{itemize}
    \item \textbf{Custom Algorithms:}
        \begin{itemize}
            \item Business rule validation
            \item Domain-specific checks
            \item Complex data verification
        \end{itemize}
\end{itemize}

\note{
Example of lookup table for standardization:
\begin{center}
\begin{tabular}{|l|l|}
    \hline
    \textbf{Short Form} & \textbf{Standard Form} \\
    \hline
    IT & Italy \\
    FR & France \\
    DE & Germany \\
    GR & Greece \\
    ES & Spain \\
    \hline
\end{tabular}
\end{center}

}\section{Transformation and Loading}

\subsection{Transformation Types}
\begin{enumerate}
    \item \textbf{Basic Conversion:}
        \begin{itemize}
            \item Data type changes
            \item Format standardization
            \item Unit conversion
            \item Example: Date formatting (12/11/2018 → 20181112)
        \end{itemize}
    \item \textbf{Data Enrichment:}
        \begin{itemize}
            \item Derived calculations
            \item Attribute combination
            \item Value enhancement
            \item Example: Profit = Revenue - Cost
        \end{itemize}
    \item \textbf{Structural Transformation:}
        \begin{itemize}
            \item Denormalization
            \item Key generation
            \item Hierarchy building
            \item Example: Combining customer name fields
        \end{itemize}
\end{enumerate}

\subsection{Loading Strategies}
\begin{itemize}
    \item \textbf{Full Refresh:}
        \begin{itemize}
            \item Complete DWH rewrite
            \item Suitable for initial loads
            \item Ensures data consistency
        \end{itemize}
    \item \textbf{Incremental Update:}
        \begin{itemize}
            \item Adds only changed data
            \item Preserves existing records
            \item More efficient for regular updates
        \end{itemize}
\end{itemize}

\subsection{Loading Considerations}
\begin{itemize}
    \item Performance optimization
    \item Data integrity maintenance
    \item Recovery procedures
    \item Load monitoring and validation
\end{itemize}

\chapter{Data Warehouse Architectures}

\section{Architectural Requirements}
\definitionbox{
DWH architectures must satisfy specific requirements to ensure effective data management, analysis capabilities, and system maintainability.
}

\subsection{Core Requirements}
\begin{enumerate}
    \item \textbf{Separation}
        \begin{itemize}
            \item Analytical and transactional processing kept separate
            \item Dedicated resources for each type of operation
            \item Optimized performance for both workloads
        \end{itemize}
    \item \textbf{Scalability}
        \begin{itemize}
            \item Easy hardware/software upgrades
            \item Handles increasing data volumes
            \item Accommodates growing user base
            \item Supports additional applications
        \end{itemize}
    \item \textbf{Extensibility}
        \begin{itemize}
            \item Integrates new technologies
            \item Adds functionality without redesign
            \item Supports evolving business needs
        \end{itemize}
    \item \textbf{Security}
        \begin{itemize}
            \item Access control and monitoring
            \item Data protection mechanisms
            \item Audit trail capabilities
        \end{itemize}
    \item \textbf{Administrability}
        \begin{itemize}
            \item Simplified management procedures
            \item Efficient maintenance processes
            \item Clear operational procedures
        \end{itemize}
\end{enumerate}

\section{Architectural Models}

\subsection{Single-Layer Architecture}
\definitionbox{
A simple architecture that minimizes data storage by removing redundancies, where the source layer is the only physical layer available.
}

\begin{itemize}
    \item \textbf{Characteristics:}
        \begin{itemize}
            \item DWH implemented as virtual view
            \item Uses middleware for data access
            \item Minimal data duplication
        \end{itemize}
    \item \textbf{Advantages:}
        \begin{itemize}
            \item Reduced storage requirements
            \item Simpler maintenance
            \item Always current data
        \end{itemize}
    \item \textbf{Disadvantages:}
        \begin{itemize}
            \item Performance limitations
            \item No separation of workloads
            \item Complex query processing
        \end{itemize}
\end{itemize}


\subsection{Two-Layer Architecture}
\definitionbox{
Provides clear separation between source systems and the data warehouse through distinct layers for different functionalities.
}

\begin{itemize}
    \item \textbf{Layer Structure:}
        \begin{itemize}
            \item Source Layer (operational data)
            \item Data Staging Area (ETL processing)
            \item Data Warehouse Layer (analytical data)
            \item Analysis Layer (user access)
        \end{itemize}
    \item \textbf{Key Features:}
        \begin{itemize}
            \item Separate physical storage
            \item Dedicated processing areas
            \item Clear functional separation
        \end{itemize}
\end{itemize}

\note{
The two-layer architecture represents a balance between complexity and functionality, making it a popular choice for many organizations.
}

\subsection{Three-Layer Architecture}
\definitionbox{
Extends the two-layer architecture by adding a reconciled layer between sources and the data warehouse, providing additional data integration capabilities.
}

\begin{itemize}
    \item \textbf{Layer Components:}
        \begin{enumerate}
            \item \textbf{Source Layer:}
                \begin{itemize}
                    \item Internal data sources
                    \item External data sources
                    \item Operational systems
                \end{itemize}
            \item \textbf{Reconciled Layer:}
                \begin{itemize}
                    \item Integrated operational data
                    \item Common data model
                    \item Cleansed and standardized data
                \end{itemize}
            \item \textbf{Data Warehouse Layer:}
                \begin{itemize}
                    \item Enterprise data warehouse
                    \item Data marts
                    \item Analytical databases
                \end{itemize}
        \end{enumerate}
\end{itemize}

\subsection{Architecture Comparison}
\begin{center}
\begin{tabular}{|p{3cm}|p{4cm}|p{4cm}|p{4cm}|}
    \hline
    \textbf{Feature} & \textbf{Single-Layer} & \textbf{Two-Layer} & \textbf{Three-Layer} \\
    \hline
    Complexity & Low & Medium & High \\
    \hline
    Data Redundancy & Minimal & Moderate & Higher \\
    \hline
    Performance & Limited & Good & Excellent \\
    \hline
    Maintenance & Simple & Moderate & Complex \\
    \hline
    Scalability & Limited & Good & Excellent \\
    \hline
    Data Integration & Basic & Good & Advanced \\
    \hline
\end{tabular}
\end{center}

\section{Implementation Considerations}

\subsection{Technology Selection}
\begin{itemize}
    \item Hardware infrastructure requirements
    \item Database management systems
    \item ETL tools and middleware
    \item Analysis and reporting tools
\end{itemize}

\subsection{Performance Optimization}
\begin{itemize}
    \item Data partitioning strategies
    \item Index design and management
    \item Query optimization techniques
    \item Cache management
\end{itemize}

\subsection{Management and Monitoring}
\begin{itemize}
    \item System health monitoring
    \item Performance metrics tracking
    \item Capacity planning
    \item Disaster recovery procedures
\end{itemize}
\chapter{Dimensional Fact Model}

\section{Conceptual Design Approaches}
\definitionbox{
The DFM (Dimensional Fact Model) is a conceptual model specifically created to support data mart design. It is graphic-based and founded on the multidimensional model.
}

\subsection{Design Methodologies}
\begin{itemize}
    \item \textbf{Requirement-driven Approach:}
        \begin{itemize}
            \item Based on user analysis
            \item Extracts information about facts, measures, hierarchies
            \item Used when source data is complex or unavailable
        \end{itemize}
    \item \textbf{Data-driven Approach:}
        \begin{itemize}
            \item Based on operational source structure
            \item Direct translation from source schemas
            \item Used when sources are well-defined
        \end{itemize}
    \item \textbf{Mixed Approach:}
        \begin{itemize}
            \item Combines both methodologies
            \item Balances user needs with data availability
            \item Most common in practice
        \end{itemize}
\end{itemize}

\section{Core Components}

\subsection{Facts and Measures}
\begin{itemize}
    \item \textbf{Facts:}
        \begin{itemize}
            \item Business events or processes
            \item Examples: sales, orders, shipments
            \item Center of analysis interest
        \end{itemize}
    \item \textbf{Measures:}
        \begin{itemize}
            \item Numerical properties of facts
            \item Quantitative aspects
            \item Examples: quantity, revenue, profit
        \end{itemize}
\end{itemize}

\subsection{Dimensions and Attributes}
\begin{itemize}
    \item \textbf{Dimensions:}
        \begin{itemize}
            \item Analysis perspectives
            \item Finite domains
            \item Examples: time, product, location
        \end{itemize}
    \item \textbf{Dimensional Attributes:}
        \begin{itemize}
            \item Descriptive properties
            \item Hierarchy components
            \item Examples: product category, month name
        \end{itemize}
\end{itemize}

\section{Advanced Concepts}

\subsection{Hierarchies and Relationships}
\begin{itemize}
    \item \textbf{Standard Hierarchies:}
        \begin{itemize}
            \item Many-to-one relationships
            \item Regular structure
            \item Example: Day → Month → Year
        \end{itemize}
    \item \textbf{Specialized Hierarchies:}
        \begin{itemize}
            \item Recursive relationships
            \item Multiple paths
            \item Optional relationships
        \end{itemize}
\end{itemize}

\subsection{Advanced Features}
\begin{itemize}
    \item \textbf{Cross-dimensional Attributes:}
        \begin{itemize}
            \item Derived from multiple dimensions
            \item Complex relationships
            \item Advanced analysis capabilities
        \end{itemize}
    \item \textbf{Convergence:}
        \begin{itemize}
            \item Multiple paths to same attribute
            \item Alternative drill-down paths
            \item Complex hierarchical structures
        \end{itemize}
    \item \textbf{Optional Elements:}
        \begin{itemize}
            \item Conditional relationships
            \item Partial hierarchies
            \item Flexible structures
        \end{itemize}
\end{itemize}

\section{Logical Implementation}

\subsection{Star Schema}
\definitionbox{
A denormalized schema structure where a central fact table is connected to dimension tables in a star-like pattern.
}

\begin{itemize}
    \item \textbf{Components:}
        \begin{itemize}
            \item Central fact table with measures
            \item Surrounding dimension tables
            \item Simple join paths
        \end{itemize}
    \item \textbf{Characteristics:}
        \begin{itemize}
            \item Denormalized dimensions
            \item Optimized for queries
            \item Simple to understand and use
        \end{itemize}
\end{itemize}


\subsection{Snowflake Schema}
\bigskip

\definitionbox{
A normalized variant of the star schema where dimension tables are split into multiple related tables.
}

\begin{itemize}
    \item \textbf{Structure:}
        \begin{itemize}
            \item Normalized dimensions
            \item Multiple related tables
            \item Hierarchical organization
        \end{itemize}
    \item \textbf{Trade-offs:}
        \begin{itemize}
            \item Reduced storage space
            \item More complex queries
            \item Better data consistency
        \end{itemize}
\end{itemize}

\subsection{Schema Comparison}
\begin{center}
\begin{tabular}{|p{3cm}|p{5.5cm}|p{5.5cm}|}
    \hline
    \textbf{Aspect} & \textbf{Star Schema} & \textbf{Snowflake Schema} \\
    \hline
    Structure & Simple, denormalized dimensions & Complex, normalized dimensions \\
    \hline
    Storage Space & Higher due to redundancy & Lower due to normalization \\
    \hline
    Query Performance & Better (fewer joins) & May be slower (more joins) \\
    \hline
    Maintenance & Easier to maintain & More complex maintenance \\
    \hline
    Data Integrity & Possible redundancy issues & Better referential integrity \\
    \hline
\end{tabular}
\end{center}

\part{Part 2 : Machine Learning}
\chapter{Machine Learning}


\end{document}
